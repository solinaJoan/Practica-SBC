% 03_formalitzacio.tex
% Capítol 3: Formalització

\section{Construcció de l'Ontologia}

\subsection{Procés de Construcció}

L'ontologia s'ha desenvolupat seguint una metodologia incremental:

\begin{enumerate}
    \item \textbf{Identificació de conceptes principals}: A partir de l'anàlisi del problema
    \item \textbf{Establiment de jerarquies}: Especialització de conceptes generals
    \item \textbf{Definició d'atributs}: Per cada classe, identificar propietats rellevants
    \item \textbf{Definició de relacions}: Connexions entre conceptes
    \item \textbf{Validació}: Comprovar cobertura de casos d'ús
    \item \textbf{Iteració}: Refinament basat en proves
\end{enumerate}

\subsection{Jerarquia de Classes}

\subsubsection{Jerarquia d'Habitatge}

\begin{verbatim}
Habitatge (classe abstracta)
├── Pis
├── Àtic
├── Dúplex
├── Estudi
└── HabitatgeUnifamiliar
\end{verbatim}

\textbf{Justificació}: Cada tipus té característiques específiques (àtics tenen terrassa gran, estudis són petits, etc.) que afecten l'adequació.

\subsubsection{Jerarquia de Sol·licitant}

\begin{verbatim}
Sol·licitant (classe abstracta)
├── Individu
├── Parella
│   ├── ParellaSenseFills
│   ├── ParellaAmbFills
│   └── ParellaFutursFills
├── Família
│   ├── FamiliaBiparental
│   └── FamiliaMonoparental
├── GrupEstudiants
└── PersonaGran
\end{verbatim}

\textbf{Justificació}: Les necessitats varien significativament segons la tipologia familiar i demogràfica.

\subsubsection{Jerarquia de Servei}

\begin{verbatim}
Servei (classe abstracta)
├── TransportPublic
│   ├── EstacioMetro
│   ├── ParadaBus
│   └── EstacioTren
├── ServeiEducatiu
│   ├── LlarInfants
│   ├── Escola
│   ├── Institut
│   └── Universitat
├── ServeiSalut
│   ├── Hospital
│   ├── CentreSalut
│   └── Farmàcia
├── ServeiComercial
│   ├── Supermercat
│   ├── Hipermercat
│   ├── CentreComercial
│   └── Mercat
├── ZonaVerda
│   ├── Parc
│   ├── Jardí
│   └── ZonaEsportiva
├── ServeiOci
│   ├── Gimnàs
│   ├── Biblioteca
│   ├── CentreCultural
│   └── ZonaNocturna
└── ServeiMolest
    ├── Discoteca
    ├── Estadi
    ├── Aeroport
    ├── ZonaIndustrial
    └── Autopista
\end{verbatim}

\textbf{Justificació}: La categorització permet regles específiques (ex: famílies necessiten escoles, no estadis).

\subsection{Atributs Principals per Classe}

\subsubsection{Classe Habitatge}

\begin{table}[h]
\centering
\small
\begin{tabular}{|l|l|p{6cm}|}
\hline
\textbf{Atribut} & \textbf{Tipus} & \textbf{Justificació} \\
\hline
superficieHabitable & Float & Determina capacitat per nombre de persones \\
numeroDormitoris & Integer & Essencial per famílies \\
numeroDormitorisDobles & Integer & Afecta capacitat real \\
numeroDormitorisSimples & Integer & Afecta capacitat real \\
numeroBanys & Integer & Comoditat, necessari per famílies grans \\
plantaPis & Integer & Accessibilitat per persones grans \\
teAscensor & Boolean & Crític per accessibilitat \\
permetMascotes & Boolean & Restricció obligatòria \\
moblat & Boolean & Necessari per estudiants \\
teTerrassaOBalco & Boolean & Plus valorat \\
superficieTerrassa & Float & Diferencia qualitat \\
orientacioSolar & String & Afecta qualitat de vida \\
teVistes & Boolean & Plus valorat \\
tipusVistes & String & Mar, muntanya, ciutat \\
teCalefaccio & Boolean & Comoditat \\
teAireCondicionat & Boolean & Comoditat \\
tePlacaAparcament & Boolean & Necessari si tens vehicle \\
consumEnergetic & String & A, B, C, D, E (eficiència) \\
nivellSoroll & String & Baix, Mitjà, Alt \\
estatConservacio & String & Nou, BonEstat, Reformar \\
\hline
\end{tabular}
\caption{Atributs de la classe Habitatge}
\end{table}

\subsubsection{Classe Sol·licitant}

\begin{table}[h]
\centering
\small
\begin{tabular}{|l|l|p{6cm}|}
\hline
\textbf{Atribut} & \textbf{Tipus} & \textbf{Justificació} \\
\hline
nom & String & Identificació \\
edat & Integer & Determina categoria (jove, gran) \\
numeroPersones & Integer & Espai necessari \\
pressupostMaxim & Float & Restricció crítica \\
pressupostMinim & Float & Detecta ofertes sospitoses \\
margeEstricte & Boolean & Flexibilitat en preu \\
numeroFills & Integer & Necessitats específiques \\
edatsFills & List<Integer> & Afecta serveis necessaris \\
teAvis & Boolean & Accessibilitat \\
teVehicle & Boolean & Parking rellevant \\
teMascotes & Boolean & Restricció obligatòria \\
numeroMascotes & Integer & - \\
tipusMascota & String & - \\
prefereixTransportPublic & Boolean & Proximitat transport \\
necessitaAccessibilitat & Boolean & Ascensor obligatori \\
treballaACiutat & Boolean & Transport necessari \\
estudiaACiutat & Boolean & Transport necessari \\
\hline
\end{tabular}
\caption{Atributs de la classe Sol·licitant}
\end{table}

\subsubsection{Classe Oferta}

\begin{table}[h]
\centering
\small
\begin{tabular}{|l|l|p{6cm}|}
\hline
\textbf{Atribut} & \textbf{Tipus} & \textbf{Justificació} \\
\hline
preuMensual & Float & Restricció principal \\
disponible & Boolean & Filtrar no disponibles \\
dataPublicacio & String & Informació addicional \\
grauRecomanacio & String & Calculat pel sistema \\
motiusRecomanacio & List<String> & Explicabilitat \\
\hline
\end{tabular}
\caption{Atributs de la classe Oferta}
\end{table}

\subsection{Relacions (Object Properties)}

\subsubsection{Relació teLocalitzacio}

\begin{itemize}
    \item \textbf{Domini}: Habitatge $\cup$ Servei
    \item \textbf{Rang}: Localitzacio
    \item \textbf{Cardinalitat}: Funcional (1:1)
    \item \textbf{Justificació}: Tot habitatge i servei té exactament una ubicació
\end{itemize}

\subsubsection{Relació aPropDe}

\begin{itemize}
    \item \textbf{Domini}: Localitzacio
    \item \textbf{Rang}: Servei
    \item \textbf{Subpropietats}:
    \begin{itemize}
        \item moltAPropDe: < 500m
        \item aDistanciaMitjana: 500-1000m
        \item llunyde: > 1000m
    \end{itemize}
    \item \textbf{Justificació}: Permet inferir proximitat a serveis sense calcular sempre
\end{itemize}

\subsubsection{Relació teHabitatge}

\begin{itemize}
    \item \textbf{Domini}: Oferta
    \item \textbf{Rang}: Habitatge
    \item \textbf{Cardinalitat}: Funcional (1:1)
    \item \textbf{Justificació}: Cada oferta fa referència a un habitatge concret
\end{itemize}

\subsubsection{Relacions de Preferència del Sol·licitant}

\begin{table}[h]
\centering
\begin{tabular}{|l|l|l|p{4cm}|}
\hline
\textbf{Relació} & \textbf{Domini} & \textbf{Rang} & \textbf{Significat} \\
\hline
requereixServei & Sol·licitant & Servei & Necessita obligatòriament \\
prefereixServei & Sol·licitant & Servei & Millora valoració \\
evitaServei & Sol·licitant & ServeiMolest & Descarta si massa a prop \\
\hline
\end{tabular}
\caption{Relacions de preferències}
\end{table}

\subsection{Documentació de l'Ontologia}

L'ontologia s'ha creat amb \textbf{Protégé 5.6} i exportada a:
\begin{itemize}
    \item \textbf{OWL/XML} (\texttt{ontologiaSBC.owl}): Format estàndard
    \item \textbf{Turtle} (\texttt{ontologiaSBC.ttl}): Format llegible
    \item \textbf{CLIPS} (\texttt{ontologiaSBC.clp}): Mitjançant owl2clips
\end{itemize}

La jerarquia completa es pot visualitzar a l'Annex B.

\section{Metodologia de Resolució de Problemes}

\subsection{Paradigma Escollit: Forward Chaining amb Fases}

S'utilitza \textbf{encadenament cap endavant (forward chaining)} organitzat en \textbf{tres fases seqüencials}:

\begin{enumerate}
    \item \textbf{Abstracció}: Inferència de requisits
    \item \textbf{Resolució}: Avaluació i filtratge
    \item \textbf{Refinació}: Classificació final
\end{enumerate}

\subsection{Justificació de la Metodologia}

\subsubsection{Per què Forward Chaining?}

\begin{itemize}
    \item \textbf{Dades disponibles des del principi}: Tenim tots els sol·licitants i ofertes
    \item \textbf{Avaluació exhaustiva}: Volem avaluar totes les combinacions
    \item \textbf{Explicabilitat}: El traç de regles és natural
    \item \textbf{Eficiència}: CLIPS està optimitzat per FC amb algoritme RETE
\end{itemize}

\subsubsection{Per què Organització en Fases?}

\begin{itemize}
    \item \textbf{Separació de responsabilitats}: Cada fase té objectiu clar
    \item \textbf{Control de flux}: Evita conflictes entre regles de fases diferents
    \item \textbf{Mantenibilitat}: Facilita afegir/modificar regles
    \item \textbf{Debugging}: Errors localitzats per fase
\end{itemize}

Control de fases mitjançant:
\begin{lstlisting}
(deftemplate fase-completada
    (slot nom (type SYMBOL)))
    
;; Les regles de cada fase comproven:
(declare (salience 40))
(fase-completada (nom abstraccio))
...
\end{lstlisting}

\subsection{Templates Auxiliars}

A més de les classes de l'ontologia, s'utilitzen templates CLIPS per raonament:

\subsubsection{proximitat}
Emmagatzema distància entre habitatge i servei:
\begin{lstlisting}
(deftemplate proximitat
    (slot habitatge (type INSTANCE))
    (slot servei (type INSTANCE))
    (slot categoria (type SYMBOL))
    (slot distancia (type SYMBOL))  ; MoltAProp, DistanciaMitjana, Lluny
    (slot metres (type FLOAT)))
\end{lstlisting}

\subsubsection{requisit-inferit}
Necessitats detectades automàticament:
\begin{lstlisting}
(deftemplate requisit-inferit
    (slot solicitant (type INSTANCE))
    (slot categoria-servei (type SYMBOL))
    (slot obligatori (type SYMBOL))  ; si/no
    (slot motiu (type STRING)))
\end{lstlisting}

\subsubsection{oferta-descartada}
Ofertes eliminades amb justificació:
\begin{lstlisting}
(deftemplate oferta-descartada
    (slot solicitant (type INSTANCE))
    (slot oferta (type INSTANCE))
    (slot motiu (type STRING)))
\end{lstlisting}

\subsubsection{criteri-no-cumplit}
Aspectes negatius de ofertes parcials:
\begin{lstlisting}
(deftemplate criteri-no-cumplit
    (slot solicitant (type INSTANCE))
    (slot oferta (type INSTANCE))
    (slot criteri (type STRING))
    (slot gravetat (type SYMBOL)))  ; Lleu, Moderat, Greu
\end{lstlisting}

\subsubsection{punt-positiu}
Aspectes destacables:
\begin{lstlisting}
(deftemplate punt-positiu
    (slot solicitant (type INSTANCE))
    (slot oferta (type INSTANCE))
    (slot descripcio (type STRING)))
\end{lstlisting}

\subsubsection{recomanacio}
Resultat final:
\begin{lstlisting}
(deftemplate recomanacio
    (slot solicitant (type INSTANCE))
    (slot oferta (type INSTANCE))
    (slot grau (type SYMBOL))  ; MoltRecomanable, Adequat, Parcialment
    (slot puntuacio (type INTEGER)))
\end{lstlisting}

\section{Disseny de Regles}

\subsection{Estratègia de Resolució per Fase}

\subsubsection{FASE 1: Abstracció (Salience 95-100)}

\textbf{Objectiu}: Inferir requisits implícits del perfil

\textbf{Regles exemple}:

\begin{lstlisting}
(defrule abstraccio-familia-amb-fills
    "Les families amb fills necessiten escoles i zones verdes"
    (declare (salience 95))
    ?sol <- (object (is-a Solicitant) (numeroFills ?fills))
    (test (> ?fills 0))
    (not (requisit-inferit (solicitant ?sol) (categoria-servei ServeiEducatiu)))
    =>
    (assert (requisit-inferit (solicitant ?sol) 
            (categoria-servei ServeiEducatiu)
            (obligatori si) 
            (motiu "Familia amb fills necessita escoles")))
    (assert (requisit-inferit (solicitant ?sol) 
            (categoria-servei ZonaVerda)
            (obligatori no) 
            (motiu "Familia amb fills prefereix zones verdes"))))
\end{lstlisting}

\textbf{Altres regles d'aquesta fase}:
\begin{itemize}
    \item \texttt{abstraccio-calcular-proximitats}: Calcula distàncies
    \item \texttt{abstraccio-persona-gran}: Salut + comerç
    \item \texttt{abstraccio-estudiants}: Transport
    \item \texttt{abstraccio-prefereix-transport}: Si ho indica explícitament
\end{itemize}

\subsubsection{FASE 2: Resolució (Salience 30-45)}

\textbf{Objectiu}: Avaluar ofertes, descartar inadequades, detectar pros/contres

\paragraph{Regles de descart (Salience 40-45)}:

\begin{lstlisting}
(defrule resolucio-descartar-preu-excessiu
    (declare (salience 40))
    (fase-completada (nom abstraccio))
    ?sol <- (object (is-a Solicitant) 
            (pressupostMaxim ?max) (margeEstricte si))
    ?of <- (object (is-a Oferta) (preuMensual ?preu) (disponible si))
    (test (> ?preu ?max))
    (not (oferta-descartada (solicitant ?sol) (oferta ?of)))
    =>
    (assert (oferta-descartada (solicitant ?sol) (oferta ?of)
            (motiu "Preu supera pressupost maxim (estricte)"))))
\end{lstlisting}

Altres descarts:
\begin{itemize}
    \item No permet mascotes
    \item No accessible
    \item Preu sospitós (massa baix)
    \item Servei moles molt a prop
    \item Falta requisit obligatori
    \item Superfície insuficient
\end{itemize}

\paragraph{Regles de criteris no complerts (Salience 35)}:

Per ofertes que no es descarten però tenen aspectes negatius:

\begin{lstlisting}
(defrule resolucio-criteri-preu-alt
    (declare (salience 35))
    (fase-completada (nom abstraccio))
    ?sol <- (object (is-a Solicitant) 
            (pressupostMaxim ?max) (margeEstricte no))
    ?of <- (object (is-a Oferta) (preuMensual ?preu))
    (test (and (> ?preu ?max) (<= ?preu (* ?max 1.15))))
    (not (oferta-descartada (solicitant ?sol) (oferta ?of)))
    =>
    (assert (criteri-no-cumplit (solicitant ?sol) (oferta ?of)
            (criteri "Preu lleugerament superior") 
            (gravetat Lleu))))
\end{lstlisting}

\paragraph{Regles de punts positius (Salience 30)}:

\begin{lstlisting}
(defrule resolucio-punt-bon-preu
    (declare (salience 30))
    ?sol <- (object (is-a Solicitant) (pressupostMaxim ?max))
    ?of <- (object (is-a Oferta) (preuMensual ?preu))
    (test (< ?preu (* ?max 0.8)))
    (not (oferta-descartada (solicitant ?sol) (oferta ?of)))
    =>
    (assert (punt-positiu (solicitant ?sol) (oferta ?of) 
            (descripcio "Preu molt bo (>20% estalvi)"))))
\end{lstlisting}

Altres punts positius:
\begin{itemize}
    \item Terrassa/balcó
    \item Molt assolellat
    \item Alta eficiència energètica
    \item Exterior i silenciós
    \item Bones vistes
    \item Parking (si té vehicle)
    \item Transport molt a prop
\end{itemize}

\subsubsection{FASE 3: Refinació (Salience 3-5)}

\textbf{Objectiu}: Classificar ofertes segons punts positius/negatius

\begin{lstlisting}
(defrule refinacio-molt-recomanable
    (declare (salience 5))
    (fase-completada (nom resolucio))
    ?sol <- (object (is-a Solicitant))
    ?of <- (object (is-a Oferta))
    (not (oferta-descartada (solicitant ?sol) (oferta ?of)))
    (not (criteri-no-cumplit (solicitant ?sol) (oferta ?of)))
    (punt-positiu (solicitant ?sol) (oferta ?of) (descripcio ?d1))
    (punt-positiu (solicitant ?sol) (oferta ?of) (descripcio ?d2&~?d1))
    (punt-positiu (solicitant ?sol) (oferta ?of) (descripcio ?d3&~?d1&~?d2))
    =>
    (assert (recomanacio (solicitant ?sol) (oferta ?of) 
            (grau MoltRecomanable) (puntuacio 100))))
\end{lstlisting}

Criteris de classificació:
\begin{itemize}
    \item \textbf{Molt Recomanable}: 0 negatius, 3+ positius
    \item \textbf{Adequat}: 0 negatius, <3 positius
    \item \textbf{Parcialment}: 1-2 negatius
    \item \textbf{Descartat}: (ja filtrat en fase 2)
\end{itemize}

\subsection{Control de Salience}

\begin{table}[h]
\centering
\begin{tabular}{|l|c|l|}
\hline
\textbf{Fase} & \textbf{Salience} & \textbf{Objectiu} \\
\hline
Càlcul proximitats & 100 & Precalcular distàncies \\
Inferència requisits & 95 & Deduir necessitats \\
Fi abstracció & 50 & Marcar fase completa \\
Descart obligatori & 40-45 & Eliminar inadequades \\
Criteris negatius & 35 & Detectar febleses \\
Punts positius & 30-32 & Detectar fortaleses \\
Fi resolució & 10 & Marcar fase completa \\
Classificació final & 3-5 & Assignar grau \\
Presentació & -10 a -100 & Mostrar resultats \\
\hline
\end{tabular}
\caption{Estratègia de salience per controlar flux}
\end{table}

\section{Justificació de Decisions de Disseny}

\subsection{Per què no Backward Chaining?}

\begin{itemize}
    \item BC és goal-driven: "És adequat l'habitatge X per usuari Y?"
    \item Nosaltres volem avaluar TOTES les ofertes per TOTS els usuaris
    \item FC avalua exhaustivament totes les combinacions
\end{itemize}

\subsection{Per què Templates en lloc de només Objects?}

\begin{itemize}
    \item \textbf{Objects (COOL)}: Per conceptes del domini (ontologia)
    \item \textbf{Templates}: Per fets de raonament temporal
    \begin{itemize}
        \item Més eficients per pattern matching
        \item Faciliten acumulació d'evidència
        \item Simplifiquen regles
    \end{itemize}
\end{itemize}

\subsection{Per què Tres Fases?}

Alternatives considerades:

\begin{enumerate}
    \item \textbf{Una sola fase}: Massa complex, regles interfereixen
    \item \textbf{Dues fases} (Avaluació + Classificació): No separa inferència
    \item \textbf{Tres fases} (escollit): Equilibri òptim
\end{enumerate}

\subsection{Per què Proximitat Pre-calculada?}

\begin{itemize}
    \item \textbf{Alternativa}: Calcular distància a cada regla
    \item \textbf{Problema}: Ineficient, codi duplicat
    \item \textbf{Solució}: Pre-calcular i assertar com a fets
\end{itemize}

\section{Cobertura de l'Ontologia}

L'ontologia cobreix:

\begin{itemize}
    \item \textbf{Tipus d'habitatge}: 5 categories principals
    \item \textbf{Perfils sol·licitant}: 7 tipologies
    \item \textbf{Serveis}: 6 categories, 20+ subclasses
    \item \textbf{Atributs}: 40+ propietats rellevants
    \item \textbf{Regles}: 30+ regles de raonament
\end{itemize}

Casos no coberts (conscientment fora d'abast):
\begin{itemize}
    \item Habitatges comercials (oficines, locals)
    \item Lloguers temporals (vacances)
    \item Co-housing o lloguers compartits parcials
\end{itemize}