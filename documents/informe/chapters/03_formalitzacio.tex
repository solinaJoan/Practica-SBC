% Capítol 3: Formalització

\section{Formalització}
\label{sec:formalitzacio}

En aquesta secció es presenta el procés de formalització del problema mitjançant la construcció d'una ontologia que representa tot el coneixement necessari per al sistema de recomanació d'habitatges. A més, s'explica detalladament la metodologia de resolució de problemes escollida i com s'estructura el raonament del sistema.

\subsection{Construcció de l'ontologia}

La construcció de l'ontologia és una de les fases més crítiques del desenvolupament d'un sistema basat en el coneixement. Una ontologia ben dissenyada no només facilita la implementació de les regles, sinó que també garanteix que el sistema pugui cobrir tots els casos rellevants del domini.

\subsubsection{Procés de construcció}

El procés de construcció de l'ontologia ha seguit una metodologia incremental basada en les millors pràctiques d'enginyeria del coneixement. Els passos seguits han estat:

\begin{enumerate}
    \item \textbf{Identificació de conceptes principals}: A partir de l'anàlisi del domini realitzada en la fase de conceptualització, s'han identificat els conceptes clau que intervenen en el problema: habitatges, sol·licitants, ofertes, serveis i localitzacions.
    
    \item \textbf{Establiment de jerarquies}: Per cada concepte principal s'ha analitzat si calia crear subclasses especialitzades. Per exemple, s'ha determinat que no tots els habitatges són iguals i que les necessitats varien significativament segons el tipus de sol·licitant.
    
    \item \textbf{Definició d'atributs}: Per cada classe s'han identificat les propietats rellevants que permeten caracteritzar adequadament les instàncies i que són necessàries per al raonament del sistema.
    
    \item \textbf{Definició de relacions}: S'han establert les connexions entre conceptes mitjançant object properties, especificant domini, rang i cardinalitat quan és rellevant.
    
    \item \textbf{Validació amb casos d'ús}: L'ontologia s'ha validat comprovant que permet representar tots els escenaris identificats durant la conceptualització.
    
    \item \textbf{Iteració i refinament}: L'ontologia s'ha refinat múltiples vegades durant el desenvolupament, afegint conceptes o atributs quan les proves revelaven mancances.
\end{enumerate}

Aquest procés iteratiu ha estat fonamental per arribar a una ontologia completa i coherent que cobreix tots els aspectes rellevants del problema.

\subsubsection{Jerarquia de classes}

L'ontologia s'estructura en diverses jerarquies de classes que representen els diferents aspectes del domini. A continuació es presenten les jerarquies principals i es justifica per què s'han dissenyat d'aquesta manera.

\paragraph{Jerarquia d'Habitatge}

La jerarquia d'habitatges distingeix entre diferents tipus d'habitatge, ja que cadascun té característiques específiques que afecten la seva adequació per a diferents perfils:

\begin{verbatim}
Habitatge (classe abstracta)
|-- Pis
|-- Àtic
|-- Dúplex
|-- Estudi
`-- HabitatgeUnifamiliar
\end{verbatim}

\textbf{Justificació de la jerarquia}: Aquesta distinció és necessària perquè cada tipus d'habitatge té característiques intrínsecament diferents. Els àtics solen tenir terrasses més grans i millors vistes però poden tenir problemes d'aïllament tèrmic. Els estudis són adequats per a persones soles o parelles sense fills però no per a famílies. Els habitatges unifamiliars ofereixen més privacitat i espai però requereixen més manteniment. Aquestes diferències són rellevants per al raonament del sistema i justifiquen la necessitat de subclasses específiques.

\paragraph{Jerarquia de Sol·licitant}

La jerarquia de sol·licitants és potser la més important, ja que el perfil del sol·licitant determina completament quines són les seves necessitats:

\begin{verbatim}
Sol·licitant (classe abstracta)
|-- Individu
|-- Parella
|   |-- ParellaSenseFills
|   |-- ParellaAmbFills
|   `-- ParellaFutursFills
|-- Família
|   |-- FamiliaBiparental
|   `-- FamiliaMonoparental
|-- GrupEstudiants
`-- PersonaGran
\end{verbatim}

\textbf{Justificació de la jerarquia}: Les necessitats d'habitatge varien radicalment segons la tipologia demogràfica del sol·licitant. Una família amb fills necessita escoles properes i zones verdes, mentre que un grup d'estudiants prioritza el transport públic i el preu baix. Una persona gran necessita accessibilitat i serveis de salut propers. Aquesta jerarquia permet al sistema aplicar regles específiques per a cada perfil sense haver de codificar totes les combinacions possibles de característiques. A més, la distinció entre parelles amb fills, sense fills i amb plans de tenir fills permet anticipar necessitats futures.

\paragraph{Jerarquia de Servei}

Els serveis del voltant de l'habitatge són un factor clau en la decisió. La jerarquia de serveis permet classificar-los segons la seva funció:

\begin{verbatim}
Servei (classe abstracta)
|-- TransportPublic
|   |-- EstacioMetro
|   |-- ParadaBus
|   `-- EstacioTren
|-- ServeiEducatiu
|   |-- LlarInfants
|   |-- Escola
|   |-- Institut
|   `-- Universitat
|-- ServeiSalut
|   |-- Hospital
|   |-- CentreSalut
|   `-- Farmàcia
|-- ServeiComercial
|   |-- Supermercat
|   |-- Hipermercat
|   |-- CentreComercial
|   `-- Mercat
|-- ZonaVerda
|   |-- Parc
|   |-- Jardí
|   `-- ZonaEsportiva
|-- ServeiOci
|   |-- Gimnàs
|   |-- Biblioteca
|   |-- CentreCultural
|   `-- ZonaNocturna
`-- ServeiMolest
    |-- Discoteca
    |-- Estadi
    |-- Aeroport
    |-- ZonaIndustrial
    `-- Autopista
\end{verbatim}

\textbf{Justificació de la jerarquia}: Aquesta categorització permet que les regles del sistema facin referència a categories completes de serveis en lloc d'haver de llistar serveis individuals. Per exemple, una regla pot expressar que "les famílies amb fills necessiten serveis educatius propers" sense haver d'enumerar escoles, instituts, llars d'infants, etc. A més, la distinció entre serveis desitjables i serveis molestos (ServeiMolest) permet que el sistema pugui descartar ofertes que estan massa a prop d'elements que poden resultar problemàtics com discoteques o estadis.

\subsubsection{Atributs de les classes}

Els atributs definits per a cada classe són els que permeten caracteritzar les instàncies i fer raonaments sobre elles. A continuació es detallen els atributs més rellevants de les classes principals.

\paragraph{Atributs de la classe Habitatge}

La classe Habitatge té un conjunt extens d'atributs que descriuen tant característiques físiques com qualitatives de l'habitatge:

\begin{itemize}
    \item \textbf{superficieHabitable} (Float): La superfície en metres quadrats és fonamental per determinar si un habitatge és adequat per al nombre de persones que hi viuran. S'utilitza per descartar habitatges massa petits.
    
    \item \textbf{numeroDormitoris, numeroDormitorisDobles, numeroDormitorisSimples} (Integer): El nombre i tipus de dormitoris determina la capacitat real de l'habitatge. No és el mateix tenir 3 dormitoris simples que 2 dobles i 1 simple per a una família de 4 persones.
    
    \item \textbf{numeroBanys} (Integer): El nombre de banys és rellevant especialment per a famílies grans o habitatges compartits. Un sol bany pot ser un problema per a més de 3 persones.
    
    \item \textbf{plantaPis} (Integer): La planta de l'habitatge és rellevant per a l'accessibilitat. Plantes baixes són ideals per a persones grans, mentre que plantes altes amb vistes són valorades positivament.
    
    \item \textbf{teAscensor} (Boolean): La presència d'ascensor és crítica per a persones amb necessitats d'accessibilitat. Un habitatge sense ascensor en una planta alta pot ser un motiu de descart automàtic per a persones grans.
    
    \item \textbf{permetMascotes} (Boolean): Aquest és un criteri de descart obligatori. Si un sol·licitant té mascotes i l'habitatge no les permet, l'oferta es descarta immediatament.
    
    \item \textbf{moblat} (Boolean): Important especialment per a estudiants o persones que busquen estades temporals. Un habitatge moblat estalvia la inversió inicial en mobiliari.
    
    \item \textbf{teTerrassaOBalco} (Boolean) i \textbf{superficieTerrassa} (Float): La presència d'espai exterior és valorada positivament, especialment per a famílies amb fills o persones que treballen des de casa.
    
    \item \textbf{orientacioSolar} (String: "Mati", "Tarda", "TotElDia", "Nord"): La quantitat de llum natural afecta significativament la qualitat de vida. Habitatges assolellats tot el dia són valorats positivament.
    
    \item \textbf{teVistes} (Boolean) i \textbf{tipusVistes} (String): Les bones vistes són un plus que pot compensar altres aspectes menys favorables.
    
    \item \textbf{teCalefaccio, teAireCondicionat} (Boolean): La climatització afecta el confort i els costos d'ús de l'habitatge.
    
    \item \textbf{tePlacaAparcament} (Boolean): Fonamental per a sol·licitants que tenen vehicle. Pot ser un criteri de descart si el sol·licitant té cotxe i la zona no té aparcament públic.
    
    \item \textbf{consumEnergetic} (String: "A", "B", "C", "D", "E"): L'eficiència energètica determina els costos d'ús i és cada vegada més valorada.
    
    \item \textbf{nivellSoroll} (String: "Baix", "Mitja", "Alt"): El soroll és un factor de qualitat de vida important. Habitatges amb soroll alt poden ser descartats o penalitzats.
    
    \item \textbf{estatConservacio} (String: "Nou", "BonEstat", "Reformar"): L'estat de conservació afecta els costos inicials i de manteniment.
\end{itemize}

Cadascun d'aquests atributs s'ha inclòs perquè és rellevant per a alguna de les regles del sistema. No hi ha atributs superflus, però tampoc hi manca informació necessària per fer un raonament adequat.

\paragraph{Atributs de la classe Sol·licitant}

Els atributs del sol·licitant descriuen tant les seves característiques demogràfiques com les seves preferències i restriccions:

\begin{itemize}
    \item \textbf{nom} (String): Identificador per a la presentació de resultats.
    
    \item \textbf{edat} (Integer): L'edat determina la categoria demogràfica i influeix en necessitats específiques (persones joves vs. persones grans).
    
    \item \textbf{numeroPersones} (Integer): El nombre de persones determina l'espai mínim necessari i el nombre de dormitoris adequat.
    
    \item \textbf{pressupostMaxim, pressupostMinim} (Float): El pressupost és la restricció més important. El màxim determina què es pot pagar, el mínim permet detectar ofertes sospitosament barates.
    
    \item \textbf{margeEstricte} (Boolean): Indica si el pressupost màxim és inamovible o hi ha una mica de flexibilitat. Si és estricte, qualsevol oferta per sobre es descarta. Si no, ofertes lleugerament per sobre poden considerar-se com a parcialment adequades.
    
    \item \textbf{numeroFills} (Integer) i \textbf{edatsFills} (List<Integer>): La presència i edat dels fills és fonamental per inferir necessitats (escoles, zones verdes) i determinar l'espai necessari.
    
    \item \textbf{teAvis} (Boolean): La presència de persones grans a la llar afegeix necessitats d'accessibilitat i proximitat a serveis de salut.
    
    \item \textbf{teVehicle} (Boolean): Determina si cal valorar la presència de parking.
    
    \item \textbf{teMascotes} (Boolean), \textbf{numeroMascotes} (Integer), \textbf{tipusMascota} (String): Les mascotes són una restricció obligatòria que pot descartar ofertes.
    
    \item \textbf{prefereixTransportPublic} (Boolean): Indica si el sol·licitant vol tenir transport públic molt a prop.
    
    \item \textbf{necessitaAccessibilitat} (Boolean): Restricció obligatòria per a persones amb mobilitat reduïda. Descarta habitatges sense ascensor en plantes altes.
    
    \item \textbf{treballaACiutat, estudiaACiutat} (Boolean): Si el sol·licitant treballa o estudia a la ciutat, la proximitat a transport públic és més important.
\end{itemize}

Aquests atributs permeten al sistema inferir automàticament necessitats i aplicar les regles adequades per a cada perfil.

\paragraph{Atributs de la classe Oferta}

L'oferta és la classe que connecta un habitatge amb les condicions de lloguer:

\begin{itemize}
    \item \textbf{preuMensual} (Float): El preu de l'oferta, que es compara amb el pressupost del sol·licitant.
    
    \item \textbf{disponible} (Boolean): Permet filtrar ofertes que ja no estan disponibles.
    
    \item \textbf{dataPublicacio} (String): Informació addicional que pot ser rellevant per a l'usuari.
    
    \item \textbf{grauRecomanacio} (String): Atribut que el sistema emplenarà amb la classificació final (MoltRecomanable, Adequat, Parcialment).
    
    \item \textbf{motiusRecomanacio} (List<String>): Llista de justificacions de per què es recomana o no l'oferta.
\end{itemize}

\paragraph{Atributs de la classe Localitzacio}

\begin{itemize}
    \item \textbf{coordenadaX, coordenadaY} (Float): Coordenades en un sistema cartesià simplificat. En un sistema real es farien servir coordenades GPS (latitud/longitud).
    
    \item \textbf{districte} (String): Districte de la ciutat per a informació contextual.
    
    \item \textbf{adreca} (String): Adreça completa per a presentació.
\end{itemize}

\subsubsection{Relacions entre classes}

Les relacions (object properties) connecten les diferents classes i permeten navegar per l'ontologia durant el raonament.

\paragraph{Relació teLocalitzacio}

\begin{itemize}
    \item \textbf{Domini}: Habitatge $\cup$ Servei
    \item \textbf{Rang}: Localitzacio
    \item \textbf{Cardinalitat}: Funcional (1:1)
    \item \textbf{Justificació}: Tot habitatge i tot servei té exactament una ubicació. Aquesta relació és fonamental per poder calcular distàncies i determinar la proximitat entre habitatges i serveis.
\end{itemize}

\paragraph{Relació aPropDe}

\begin{itemize}
    \item \textbf{Domini}: Localitzacio
    \item \textbf{Rang}: Servei
    \item \textbf{Subpropietats}:
    \begin{itemize}
        \item moltAPropDe: distància $< 500$m
        \item aDistanciaMitjana: $500\text{m} \le \text{distància} < 1000\text{m}$
        \item llunyDe: $\text{distància} \ge 1000$m
    \end{itemize}
    \item \textbf{Justificació}: Aquesta jerarquia de relacions permet expressar proximitat de manera qualitativa. Les regles poden fer referència a "serveis molt a prop" sense haver de comparar distàncies numèriques en cada regla, simplificant el codi i millorant la llegibilitat.
\end{itemize}

\paragraph{Relació teHabitatge}

\begin{itemize}
    \item \textbf{Domini}: Oferta
    \item \textbf{Rang}: Habitatge
    \item \textbf{Cardinalitat}: Funcional (1:1)
    \item \textbf{Justificació}: Cada oferta fa referència a un habitatge concret. Aquesta relació permet accedir a les característiques de l'habitatge des de l'oferta.
\end{itemize}

\paragraph{Relacions de preferència del sol·licitant}

\begin{itemize}
    \item \textbf{requereixServei} (Sol·licitant $\to$ Servei): El sol·licitant necessita obligatòriament aquest servei a prop. Si no està disponible, pot ser motiu de descart.
    
    \item \textbf{prefereixServei} (Sol·licitant $\to$ Servei): El sol·licitant valora positivament aquest servei però no és obligatori.
    
    \item \textbf{evitaServei} (Sol·licitant $\to$ ServeiMolest): El sol·licitant prefereix no tenir aquest tipus de servei molt a prop. Si està molt a prop, pot ser motiu de descart.
\end{itemize}

Aquestes relacions permeten que els usuaris expressin preferències específiques més enllà de les que s'infereixen automàticament del seu perfil.

\subsection{Metodologia de resolució de problemes}

La metodologia de resolució escollida determina com s'estructura el raonament del sistema. Aquesta decisió és crítica perquè afecta tant l'eficiència com la mantenibilitat del codi.

\subsubsection{Paradigma escollit: Forward Chaining amb fases}

El sistema utilitza \textbf{encadenament cap endavant (forward chaining)} organitzat en tres fases seqüencials ben diferenciades. Cada fase té un objectiu específic i genera fets que són consumits per la fase següent.

Les tres fases són:

\begin{enumerate}
    \item \textbf{Fase d'Abstracció}: Infereix requisits implícits a partir del perfil del sol·licitant i calcula proximitats entre habitatges i serveis.
    
    \item \textbf{Fase de Resolució}: Avalua cada combinació sol·licitant-oferta, descartant ofertes inadequades i detectant punts positius i negatius.
    
    \item \textbf{Fase de Refinació}: Classifica les ofertes no descartades en categories (Molt Recomanable, Adequat, Parcialment Adequat) segons els punts positius i negatius detectats.
\end{enumerate}

\subsubsection{Justificació del Forward Chaining}

La decisió d'utilitzar forward chaining en lloc de backward chaining es basa en diversos factors:

\begin{itemize}
    \item \textbf{Dades disponibles des del principi}: En el nostre problema, tenim totes les dades (sol·licitants i ofertes) des de l'inici. No estem buscant demostrar un objectiu concret sinó avaluar exhaustivament totes les combinacions possibles.
    
    \item \textbf{Avaluació exhaustiva requerida}: Volem avaluar totes les ofertes per a cada sol·licitant i presentar-li un llistat complet de recomanacions ordenades. El backward chaining seria adequat per a consultes específiques com "És adequada l'oferta X per a l'usuari Y?", però no per a la nostra necessitat d'avaluació global.
    
    \item \textbf{Explicabilitat natural}: Amb forward chaining, el traç d'execució mostra clarament com s'han inferit requisits, per què s'han descartat ofertes i quins criteris s'han avaluat. Això facilita la generació d'explicacions comprensibles per a l'usuari.
    
    \item \textbf{Eficiència de CLIPS}: L'algoritme RETE que implementa CLIPS està altament optimitzat per a forward chaining, fent que l'avaluació de centenars de combinacions sigui molt eficient.
\end{itemize}

\subsubsection{Justificació de l'organització en fases}

L'organització del raonament en tres fases separades ofereix múltiples avantatges respecte a un sistema monolític:

\begin{itemize}
    \item \textbf{Separació de responsabilitats}: Cada fase té un objectiu clar i específic. Les regles d'inferència de requisits són conceptualment diferents de les regles de descart, que al seu torn són diferents de les regles de classificació. Aquesta separació fa que cada regla sigui més senzilla i fàcil d'entendre.
    
    \item \textbf{Control de flux predictible}: Les fases s'executen sempre en el mateix ordre (abstracció $\to$ resolució $\to$ refinació), evitant conflictes entre regles de diferents fases. Això fa que el comportament del sistema sigui més predictible i fàcil de depurar.
    
    \item \textbf{Mantenibilitat}: Si cal modificar una regla o afegir-ne una de nova, és fàcil identificar a quina fase pertany i on s'ha de col·locar. Això redueix el risc d'introduir errors en afegir funcionalitat nova.
    
    \item \textbf{Facilitat de depuració}: Si hi ha un error en els resultats, és fàcil identificar en quina fase s'ha produït. Per exemple, si una oferta s'hauria d'haver descartat però no s'ha fet, sabem que el problema està en la fase de resolució.
    
    \item \textbf{Reusabilitat}: Els resultats d'una fase (per exemple, els requisits inferits) poden ser reutilitzats per múltiples regles de fases posteriors sense haver de recalcular-los.
\end{itemize}

El control entre fases s'implementa mitjançant fets de control que indiquen quan una fase ha acabat:

\begin{lstlisting}
(deftemplate fase-completada
    (slot nom (type SYMBOL)))
\end{lstlisting}

Les regles de cada fase comproven que la fase anterior s'ha completat abans d'executar-se:

\begin{lstlisting}
(defrule resolucio-exemple
    (declare (salience 40))
    (fase-completada (nom abstraccio))  ; Comprova fase anterior
    ; ... resta de condicions ...
    =>
    ; ... accions ...
)
\end{lstlisting}

\subsubsection{Templates auxiliars per al raonament}

A més de les classes de l'ontologia (implementades com a COOL objects), el sistema utilitza templates CLIPS per representar fets intermedis del raonament. Aquesta distinció entre objects i templates és intencionada i justificada:

\begin{itemize}
    \item \textbf{Objects (COOL)}: S'utilitzen per representar conceptes del domini (habitatges, sol·licitants, serveis). Aquestes són entitats estables que existeixen independentment del raonament.
    
    \item \textbf{Templates}: S'utilitzen per representar fets derivats del raonament, que són específics d'una execució concreta i no tenen sentit fora del context del procés de decisió.
\end{itemize}

Els templates principals són:

\paragraph{proximitat}

Representa la distància entre un habitatge i un servei:

\begin{lstlisting}
(deftemplate proximitat
    (slot habitatge (type INSTANCE))
    (slot servei (type INSTANCE))
    (slot categoria (type SYMBOL))
    (slot distancia (type SYMBOL))  ; MoltAProp, DistanciaMitjana, Lluny
    (slot metres (type FLOAT)))
\end{lstlisting}

\textbf{Justificació}: Calcular distàncies és costós computacionalment. En lloc de recalcular la distància cada vegada que una regla la necessita, es calcula una sola vegada al principi i s'emmagatzema com a fet. Això millora enormement l'eficiència.

\paragraph{requisit-inferit}

Representa necessitats detectades automàticament pel sistema:

\begin{lstlisting}
(deftemplate requisit-inferit
    (slot solicitant (type INSTANCE))
    (slot categoria-servei (type SYMBOL))
    (slot obligatori (type SYMBOL))  ; si/no
    (slot motiu (type STRING)))
\end{lstlisting}

\textbf{Justificació}: Els requisits inferits són conclusions del raonament, no dades inicials. Representar-los com a fets separats permet que múltiples regles puguin consultar-los i que s'incloguin en les explicacions al usuari.

\paragraph{oferta-descartada}

Indica que una oferta s'ha eliminat de la consideració per a un sol·licitant:

\begin{lstlisting}
(deftemplate oferta-descartada
    (slot solicitant (type INSTANCE))
    (slot oferta (type INSTANCE))
    (slot motiu (type STRING)))
\end{lstlisting}

\textbf{Justificació}: En comptes d'eliminar realment les ofertes de la memòria, es marquen com a descartades amb una justificació. Això permet presentar a l'usuari per què certes ofertes no són adequades, millorant la transparència del sistema.

\paragraph{criteri-no-cumplit}

Representa aspectes negatius d'ofertes que no es descarten completament:

\begin{lstlisting}
(deftemplate criteri-no-cumplit
    (slot solicitant (type INSTANCE))
    (slot oferta (type INSTANCE))
    (slot criteri (type STRING))
    (slot gravetat (type SYMBOL)))  ; Lleu, Moderat, Greu
\end{lstlisting}

\textbf{Justificació}: No totes les ofertes són perfectes o totalment inadequades. Moltes tenen aspectes negatius que no justifiquen un descart total però que l'usuari ha de conèixer. Aquest template permet acumular aquests aspectes i utilitzar-los tant per a la classificació com per a l'explicació.

\paragraph{punt-positiu}

Representa aspectes destacables d'una oferta:

\begin{lstlisting}
(deftemplate punt-positiu
    (slot solicitant (type INSTANCE))
    (slot oferta (type INSTANCE))
    (slot descripcio (type STRING)))
\end{lstlisting}

\textbf{Justificació}: Simètricament als criteris negatius, els punts positius s'acumulen per determinar si una oferta és "Molt Recomanable" i per explicar a l'usuari què fa que l'oferta destaqui.

\paragraph{recomanacio}

Representa la classificació final d'una oferta:

\begin{lstlisting}
(deftemplate recomanacio
    (slot solicitant (type INSTANCE))
    (slot oferta (type INSTANCE))
    (slot grau (type SYMBOL))  ; MoltRecomanable, Adequat, Parcialment
    (slot puntuacio (type INTEGER)))
\end{lstlisting}

\textbf{Justificació}: La recomanació final és el resultat del raonament complet. Tenir-la com a fet separat permet presentar els resultats de manera ordenada i permet futures extensions com ordenar ofertes per puntuació.

\subsection{Justificació de decisions de disseny}

Durant el procés de formalització s'han pres diverses decisions de disseny que cal justificar adequadament.

\subsubsection{Per què no Backward Chaining?}

S'ha considerat utilitzar backward chaining però s'ha descartat pels següents motius:

\begin{itemize}
    \item \textbf{Backward chaining és goal-driven}: És adequat quan es vol demostrar un objectiu específic, per exemple "És adequat l'habitatge X per a l'usuari Y?". En el nostre cas, no tenim un objectiu específic sinó que volem avaluar exhaustivament totes les ofertes per a cada usuari.
    
    \item \textbf{Avaluació selectiva vs. exhaustiva}: Amb backward chaining, el sistema començaria des d'un objectiu i només buscaria la informació necessària per confirmar-lo o rebutjar-lo. Això és eficient per a consultes puntuals però inadequat quan volem avaluar totes les ofertes disponibles.
    
    \item \textbf{Generació d'explicacions}: Forward chaining genera naturalment un traç d'execució que mostra com s'han inferit els requisits i per què s'han pres les decisions. Aquest traç és la base de les explicacions que es presenten a l'usuari.
\end{itemize}

\subsubsection{Per què Templates en lloc de només Objects?}

CLIPS ofereix dos mecanismes per representar informació: templates (fets simples) i objects (programació orientada a objectes). S'ha optat per utilitzar ambdós de manera complementària:

\begin{itemize}
    \item \textbf{Objects per conceptes estables del domini}: Les classes de l'ontologia (Habitatge, Sol·licitant, Servei, etc.) s'implementen com a objects perquè representen entitats del món real amb estructura complexa i jerarquies d'herència.
    
    \item \textbf{Templates per fets derivats del raonament}: Els requisits inferits, proximitats, criteris no complerts, etc. s'implementen com a templates perquè són fets temporals específics d'una execució concreta.
    
    \item \textbf{Eficiència del pattern matching}: CLIPS està altament optimitzat per fer pattern matching sobre templates. Per a fets que es consulten freqüentment en condicions de regles, els templates són més eficients que els objects.
    
    \item \textbf{Simplicitat de les regles}: Els templates permeten escriure condicions més concises. Per exemple, comprovar l'existència d'un requisit inferit és més simple amb templates que amb objects.
\end{itemize}

\subsubsection{Per què tres fases en lloc de dues o més?}

S'han considerat diferents alternatives d'organització:

\begin{itemize}
    \item \textbf{Una sola fase}: Massa complex. Les regles d'inferència, descart i classificació es barrejarien, fent el sistema difícil d'entendre i mantenir. El risc de conflictes entre regles seria alt.
    
    \item \textbf{Dues fases} (Avaluació + Classificació): No separa adequadament la inferència de coneixement de l'avaluació d'ofertes. La inferència de requisits hauria de fer-se juntament amb el descart d'ofertes, cosa conceptualment confusa.
    
    \item \textbf{Tres fases} (escollit): Ofereix un equilibri òptim. Cada fase té una responsabilitat clara i ben definida. El flux és natural: primer s'entén què necessita el sol·licitant (abstracció), després s'avaluen les ofertes segons aquestes necessitats (resolució), i finalment es classifiquen els resultats (refinació).
    
    \item \textbf{Quatre o més fases}: Fragmentaria massa el procés sense guanys clars. Per exemple, separar la inferència de requisits del càlcul de proximitats afegiria complexitat innecessària.
\end{itemize}

\subsubsection{Per què pre-calcular les proximitats?}

Una alternativa seria calcular la distància entre habitatge i servei cada vegada que una regla ho necessita. S'ha optat per pre-calcular totes les distàncies per diversos motius:

\begin{itemize}
    \item \textbf{Eficiència}: El càlcul de distàncies (fins i tot amb distància de Manhattan simplificada) és computacionalment costós si s'ha de fer repetidament. Pre-calcular totes les distàncies una sola vegada i emmagatzemar-les com a fets permet que les regles posteriors les consultin de manera instantània.
    
    \item \textbf{Evita duplicació de codi}: Sense pre-càlcul, cada regla que necessita saber si un servei està a prop hauria d'incloure el càlcul de distància, duplicant codi i incrementant el risc d'errors.
    
    \item \textbf{Classificació qualitativa}: Pre-calcular permet classificar les distàncies en categories qualitatives (Molt a Prop, Distància Mitjana, Lluny) una sola vegada. Les regles poden després fer referència a aquestes categories de manera natural sense preocupar-se dels valors numèrics exactes.
    
    \item \textbf{Facilita extensions futures}: Si en el futur es volgués utilitzar distàncies més complexes (per exemple, temps de desplaçament en transport públic en lloc de distància euclidiana), només caldria modificar la regla de càlcul de proximitats, no totes les regles que utilitzen aquesta informació.
\end{itemize}

\subsection{Cobertura de l'ontologia}

L'ontologia dissenyada cobreix adequadament el domini definit en el problema. A continuació es detalla l'abast de la cobertura:

\begin{itemize}
    \item \textbf{Tipus d'habitatge}: 5 categories principals (Pis, Àtic, Dúplex, Estudi, Habitatge Unifamiliar) que cobreixen la gran majoria de casos reals d'habitatges de lloguer urbà.
    
    \item \textbf{Perfils de sol·licitant}: 7 tipologies diferents (Individu, Parella amb diferents situacions, Família, Grup d'Estudiants, Persona Gran) que representen els principals perfils demogràfics que busquen habitatge de lloguer.
    
    \item \textbf{Serveis}: 6 categories principals amb més de 20 subclasses específiques. Això permet representar tots els serveis rellevants del voltant d'un habitatge que poden influir en la decisió.
    
    \item \textbf{Atributs}: Més de 40 propietats diferents entre totes les classes, cobriment tots els aspectes rellevants identificats durant la fase de conceptualització.
    
    \item \textbf{Regles de raonament}: Més de 30 regles que cobreixen la inferència de requisits, el descart d'ofertes inadequades, la detecció de punts positius i negatius, i la classificació final.
\end{itemize}

\textbf{Casos conscientment no coberts}:

S'ha decidit deixar fora de l'abast del sistema alguns casos que, tot i ser reals, són menys freqüents o requeririen una complexitat desproporcionada:

\begin{itemize}
    \item \textbf{Habitatges comercials}: Oficines, locals comercials, naus industrials. El sistema es centra en habitatge residencial.
    
    \item \textbf{Lloguers temporals o vacacionals}: El sistema assumeix lloguers estables de llarga durada (mínim un any).
    
    \item \textbf{Co-housing o lloguers compartits parcials}: El sistema assumeix que es lloga l'habitatge complet, no habitacions individuals en habitatges compartits amb desconeguts.
    
    \item \textbf{Necessitats mèdiques molt específiques}: Més enllà de l'accessibilitat general i la proximitat a serveis de salut, no es consideren necessitats mèdiques específiques com diàlisi, rehabilitació, etc.
\end{itemize}

Aquestes exclusions estan justificades perquè els casos coberts representen la gran majoria de situacions reals i permeten mantenir el sistema en una complexitat raonable.