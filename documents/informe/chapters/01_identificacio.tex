% 01_identificacio.tex
% Capítol 1: Identificació del problema

\section{Descripció del Problema}

El problema aborda la necessitat de trobar l'habitatge de lloguer més adequat per a persones amb diferents perfils i necessitats. En un mercat immobiliari amb milers d'ofertes i característiques diverses, trobar l'opció òptima requereix avaluar múltiples factors simultàniament:

\begin{itemize}
    \item Característiques de l'habitatge (preu, superfície, nombre d'habitacions, equipament)
    \item Localització i serveis propers (transport, escoles, centres de salut, zones verdes)
    \item Perfil del sol·licitant (família, estudiants, persones grans, parelles)
    \item Necessitats específiques (accessibilitat, mascotes, vehicle propi)
\end{itemize}

Aquest procés implica coneixement expert sobre criteris de qualitat d'habitatge, necessitats segons perfils demogràfics i sentit comú sobre preferències habitacionals.

\section{Anàlisi de Viabilitat}

\subsection{És adequat un Sistema Basat en Coneixement?}

Un Sistema Basat en Coneixement (SBC) és adequat per aquest problema per les següents raons:

\begin{enumerate}
    \item \textbf{Existeix coneixement expert}: Agents immobiliaris experimentats utilitzen regles heurístiques per recomanar habitatges segons perfils.
    
    \item \textbf{El problema no té solució algorítmica simple}: La recomanació òptima depèn de múltiples factors qualitatius que interactuen de manera complexa.
    
    \item \textbf{Raonable complexitat}: El problema és prou complex per necessitar IA però no tant com per requerir aprenentatge automàtic.
    
    \item \textbf{Coneixement estructurable}: Es poden identificar clarament conceptes (habitatge, sol·licitant, servei) i relacions entre ells.
    
    \item \textbf{Explicabilitat important}: Els usuaris necessiten entendre per què se'ls recomana o descarta un habitatge.
\end{enumerate}

\subsection{Limitacions d'altres aproximacions}

\begin{itemize}
    \item \textbf{Cerca simple per filtres}: No captura preferències complexes ni coneixement expert sobre adequació.
    \item \textbf{Machine Learning}: Requereix grans quantitats de dades d'usuaris reals i no proporciona explicabilitat transparent.
    \item \textbf{Sistemes de puntuació numèrics}: No capturen adequadament les restriccions obligatòries vs. preferències opcionals.
\end{itemize}

\section{Identificació de Fonts de Coneixement}

\subsection{Fonts Primàries}

\begin{enumerate}
    \item \textbf{Experts del domini}:
    \begin{itemize}
        \item Agents immobiliaris amb experiència en lloguer d'habitatges
        \item Consultors d'habitatge social
        \item Gestors de propietats
    \end{itemize}
    
    \item \textbf{Lleis i normatives}:
    \begin{itemize}
        \item Llei d'Arrendaments Urbans (LAU)
        \item Normatives d'accessibilitat
        \item Regulacions locals de lloguer
    \end{itemize}
\end{enumerate}

\subsection{Fonts Secundàries}

\begin{enumerate}
    \item \textbf{Plataformes immobiliàries}:
    \begin{itemize}
        \item Idealista.com
        \item Fotocasa.es
        \item Habitaclia.com
    \end{itemize}
    Anàlisi de característiques ofertes i sistemes de cerca.
    
    \item \textbf{Estudis demogràfics i sociològics}:
    \begin{itemize}
        \item Necessitats habitacionals per grups d'edat
        \item Patrons de mobilitat urbana
        \item Preferències per tipologia familiar
    \end{itemize}
    
    \item \textbf{Models de llenguatge (LLM)}:
    \begin{itemize}
        \item ChatGPT, Claude, Gemini
        \item Utilitzats com a "expert virtual" per elicitació de coneixement
        \item Validació posterior amb fonts reals
    \end{itemize}
\end{enumerate}

\section{Objectius del Sistema}

\subsection{Objectius Funcionals}

\begin{enumerate}
    \item \textbf{Recomanar habitatges adequats}: Generar una llista ordenada d'ofertes segons grau d'adequació (Molt Recomanable, Adequat, Parcialment Adequat).
    
    \item \textbf{Descartar ofertes inadequades}: Identificar i filtrar habitatges que no compleixen requisits obligatoris.
    
    \item \textbf{Justificar decisions}: Proporcionar explicacions clares sobre per què un habitatge és recomanat o descartat.
    
    \item \textbf{Inferir necessitats}: Deduir requisits no explícits basant-se en el perfil del sol·licitant.
\end{enumerate}

\subsection{Objectius No Funcionals}

\begin{enumerate}
    \item \textbf{Cobertura exhaustiva}: Considerar la majoria de casos reals d'habitatge i perfils d'usuari.
    
    \item \textbf{Mantenibilitat}: Estructura modular que permeti afegir noves regles fàcilment.
    
    \item \textbf{Rendiment adequat}: Processar desenes d'ofertes i sol·licitants en temps raonable.
    
    \item \textbf{Transparència}: Regles comprensibles i traçables.
\end{enumerate}

\section{Abast del Sistema}

\subsection{Dins de l'abast}

\begin{itemize}
    \item Recomanació d'habitatges de lloguer a Barcelona (ciutat fictícia simplificada)
    \item Avaluació de proximitat a serveis urbans
    \item Perfils: individus, parelles, famílies, estudiants, persones grans
    \item Restriccions de preu, accessibilitat, mascotes
    \item Inferència de necessitats segons perfil demogràfic
\end{itemize}

\subsection{Fora de l'abast}

\begin{itemize}
    \item Habitatge en propietat (compra)
    \item Anàlisi financer detallat (hipoteques, impostos)
    \item Predicció de preus futurs
    \item Recomanacions personalitzades amb aprenentatge d'usuari
    \item Integració amb bases de dades reals d'ofertes
    \item Gestió de reserves o contractes
\end{itemize}

\section{Resultats Esperats}

El sistema ha de produir per cada sol·licitant:

\begin{enumerate}
    \item \textbf{Llista d'ofertes recomanades} amb tres nivells:
    \begin{itemize}
        \item Molt Recomanable: Compleix tot i té característiques excepcionals
        \item Adequat: Compleix tots els requisits
        \item Parcialment Adequat: Acceptable amb alguns criteris no complerts
    \end{itemize}
    
    \item \textbf{Llista d'ofertes descartades} amb justificació de cada descart.
    
    \item \textbf{Explicacions detallades} per cada recomanació:
    \begin{itemize}
        \item Punts positius destacables
        \item Criteris no complerts (per ofertes parcials)
        \item Serveis propers rellevants
    \end{itemize}
\end{enumerate}

\section{Beneficis Esperats}

\begin{itemize}
    \item \textbf{Per usuaris}: Estalvi de temps, recomanacions més adequades, transparència en decisions
    \item \textbf{Per agents immobiliaris}: Automatització de pre-filtratge, més eficiència
    \item \textbf{Per propietaris}: Millor matching amb llogaters adequats
\end{itemize}