\documentclass[12pt]{article} 
\usepackage[a4paper, left = 1in, right = 1in, top = 1in, bottom = 1in]{geometry}
\usepackage[utf8]{inputenc}
\usepackage[T1]{fontenc}
\usepackage[catalan]{babel}
\usepackage{graphicx}
\usepackage{hyperref}
\usepackage{listings}
\usepackage{xcolor}
\usepackage{fancyhdr}
\usepackage{titlesec}
\usepackage{amsmath}
\usepackage{float}        
\usepackage{caption}       
\usepackage{subcaption}    
\usepackage{tikz}
\usepackage{forest}
\usepackage{multirow}
\usepackage{longtable}
\usepackage{booktabs}
\setlength{\parskip}{1.5cm}
\usepackage{amssymb}


\title{
    \includegraphics[width=0.3\textwidth]{fotos/upc-logo.png}\\[1cm]
    {\LARGE Pràctica de Sistemes Basats en el Coneixement}\\[0.5cm]
    {\large Sistema Expert de Recomanació d'Habitatges de Lloguer}\\[2cm]
}

\author{
\large
\textsc{Grau IA -- Q1 Curs 2025-2026}\\[2mm] 
\normalsize Departament de Ciències de la Computació \\
\normalsize Universitat Politècnica de Catalunya \\[1cm]
\begin{tabular}{cc}
{\textsc{Anel Ademovic}} & {\textsc{Aleix Pitarch}} \\
\normalsize \texttt{anel.suljic@estudiantat.upc.edu} & \normalsize \texttt{aleix.pitarch@estudiantat.upc.edu} \\[4mm]
\multicolumn{2}{c}{\textsc{Joan Solina}} \\
\multicolumn{2}{c}{\normalsize \texttt{joan.solina@estudiantat.upc.edu}}\\
\end{tabular}
}

\date{14 de desembre de 2025}

\pagestyle{plain}
\setlength{\parindent}{0pt}
\setlength{\parskip}{0.5em}

\begin{document}
\section{Identificació del problema}

\subsection{Descripció del problema}

El mercat immobiliari de lloguer a Barcelona presenta una complexitat creixent que dificulta la cerca d'habitatge adequat. La regidoria d'habitatge disposa de nombroses ofertes, però connectar-les eficientment amb els ciutadans requereix un sistema intel·ligent que vagi més enllà de filtres simples.

El repte consisteix a entendre les necessitats reals de cada tipus de sol·licitant considerant el seu context vital complet. Una família amb fills petits necessita proximitat a escoles i zones verdes, mentre que una persona gran prioritza accessibilitat i serveis sanitaris propers. Aquestes preferències, sovint implícites, han de ser inferides pel sistema.

A més de les característiques intrínseques de l'habitatge (superfície, habitacions, preu), cal considerar l'entorn urbà: transport públic, comerços, centres educatius i sanitaris. Aquest coneixement territorial és complex: la distància acceptable varia segons el perfil (500m per a una persona gran, 1000m per a un jove), i certs serveis poden ser desitjables o inacceptables segons qui busca (una discoteca propera pot ser atractiva per a estudiants però inadequada per a famílies).

El sistema ha de gestionar restriccions de diferent naturalesa: absolutes (permetre mascotes, accessibilitat), preferències fortes (pressupost màxim) i desitjables (orientació, vistes). Cal distingir entre aquests nivells per generar recomanacions adaptades.

Finalment, les ofertes s'han de classificar en diferents graus de recomanació (des de parcialment adequades fins a molt recomanables) amb explicacions clares que permetin al sol·licitant entendre els avantatges i aspectes a considerar de cada opció.
\subsection{Anàlisi de viabilitat}

Per determinar si aquest problema és adequat per ser resolt mitjançant un sistema basat en el coneixement, hem analitzat diverses dimensions de viabilitat.

\subsubsection{Viabilitat tècnica}

El problema reuneix les característiques fonamentals que fan viable la construcció d'un SBC:

\textbf{Existència de coneixement expert.} Existeix un coneixement expert clar en el domini immobiliari. Els agents immobiliaris experimentats desenvolupen una comprensió profunda de com emparedar perfils de clients amb habitatges adequats, considerant factors que van més enllà de les especificacions tècniques. Aquest coneixement inclou heurístiques del tipus ``les famílies amb fills petits valoren especialment la proximitat a escoles i parcs'' o ``les persones grans necessiten accessibilitat i serveis de salut propers'', que es poden formalitzar en regles.

\textbf{Domini delimitat.} Hem acotat el problema a la ciutat de Barcelona, amb tipus d'habitatge i serveis ben definits. Aquesta delimitació fa el problema tractable sense perdre la seva essència. No intentem resoldre el problema general de recomanació d'habitatges a nivell mundial, sinó que ens centrem en un context urbà específic amb característiques ben conegudes.

\textbf{Problema de classificació i recomanació.} El problema s'ajusta perfectament a una metodologia de classificació heurística: hem de classificar ofertes en categories de recomanació (parcialment adequades, adequades, molt recomanables) basant-nos en l'avaluació de múltiples criteris. Aquest tipus de problema és un dels més adequats per SBC, ja que podem descompondre'l en subproblemes (abstracció del sol·licitant, càlcul de proximitats, descart d'ofertes, puntuació, classificació) que es resolen seqüencialment aplicant regles.

\textbf{Espai de solucions finit.} Tot i que l'espai de combinacions possibles entre sol·licitants i ofertes és gran, és finit i manejable. Amb N sol·licitants i M ofertes, tenim N×M parells a avaluar, però cada avaluació és independent i es pot resoldre aplicant un conjunt de regles ben definides.

\subsubsection{Viabilitat de desenvolupament}

El projecte és viable des del punt de vista del desenvolupament per diverses raons:

\textbf{Disponibilitat d'eines.} CLIPS proporciona un entorn robust per implementar sistemes basats en regles, amb suport per a programació orientada a objectes (COOL) que ens permet representar l'ontologia del domini de manera natural. Protégé ens ha permès dissenyar i documentar l'ontologia de forma sistemàtica abans de la implementació.

\textbf{Desenvolupament incremental.} El problema es pot abordar de manera incremental, començant amb un conjunt bàsic de regles i ampliant progressivament la cobertura. Hem pogut començar amb la gestió de restriccions dures (preu, mascotes, accessibilitat) i anar afegint regles més sofisticades per a la puntuació i classificació.

\textbf{Prototipatge ràpid.} La naturalesa declarativa de CLIPS permet fer prototips ràpidament i iterar sobre el disseny. Modificar o afegir regles no requereix reescriure grans porcions de codi, cosa que facilita l'experimentació i refinament.

\subsubsection{Limitacions identificades}

Tot i la viabilitat general, hem identificat algunes limitacions que cal tenir presents:

\textbf{Coneixement incomplet del domini.} No som experts immobiliaris reals, per la qual cosa el nostre sistema es basa principalment en coneixement de sentit comú i en patrons generals que hem pogut inferir. Un sistema de producció requeriria la participació activa d'experts del sector per afinar les regles i els pesos de puntuació.

\textbf{Dades sintètiques.} Les instàncies que utilitzem són simulades. Un sistema real necessitaria integrar-se amb bases de dades reals d'ofertes i disposar d'informació actualitzada sobre serveis urbans. També caldria considerar la dinàmica temporal (ofertes que deixen d'estar disponibles, preus que canvien).

\textbf{Absència de retroalimentació.} El sistema actual no aprèn de les decisions dels usuaris. No sabem si les recomanacions que fem són realment útils o si els usuaris acaben escollint opcions diferents de les que el sistema proposa com a millors. Un sistema real hauria d'incorporar mecanismes de feedback per ajustar els pesos i les regles.

\subsection{Fonts de coneixement}

Per construir el sistema hem identificat i utilitzat diverses fonts de coneixement:

\subsubsection{Fonts primàries}

\textbf{Coneixement de sentit comú.} La base principal del nostre sistema prové del sentit comú sobre necessitats habitacionals segons perfils demogràfics. És coneixement bàsic com per exemple que les persones grans necessiten accessibilitat o que les famílies amb fills petits valoren la proximitat a escoles i zones verdes. Són àmpliament acceptades i no requereixen expertesa específica.

\textbf{Webs d'anuncis immobiliaris.} Plataformes com Idealista ens han servit per identificar les característiques rellevants que es descriuen en les ofertes reals (superfície, nombre d'habitacions, si permet mascotes, si té terrassa, consum energètic, etc.). També ens han permès veure quines són les categories de serveis que es mencionen habitualment com a punts forts de les ubicacions.

\textbf{Preguntes a l'usuari} Les preguntes a l'usuari també formen part del coneixement del sistema. Tot seguit les expliquarem amb extensió.

\subsubsection{Fonts secundàries}

\textbf{Models de llenguatge.} Hem fet servir LLM's com a experts per demanar-los informació, perquè ens ajudi a pensar com un expert immobiliari i puguem fer una inferència més precisa.

\textbf{Experiència personal.} Els membres de l'equip hem aplicat la nostra pròpia experiència en la cerca d'habitatge i coneixement de la ciutat de Barcelona per definir proximitats raonables, serveis rellevants i preferències típiques.

\subsection{Obtenció de la informació de l'usuari}

Finalment, el sistema basat en coneixement obté la informació necessària mitjançant un procés interactiu de preguntes a l'usuari. Aquest procés permet construir un perfil complet del sol·licitant que posteriorment serà utilitzat pel motor d'inferència per generar recomanacions adequades.

La recollida de dades es duu a terme mitjançant la funció \texttt{crear-perfil-solicitant}, la qual guia l'usuari a través d'una sèrie de preguntes estructurades en diferents blocs temàtics. Cada bloc respon a un conjunt de variables rellevants per a la presa de decisions del sistema.

\subsubsection{Dades personals}
En primer lloc, el sistema demana informació bàsica del sol·licitant, com ara:
\begin{itemize}
    \item Nom o identificador
    \item Edat
    \item Si es vol comprar un habitatge com a segona residència
\end{itemize}

Aquestes dades permeten contextualitzar el perfil i activar determinades preguntes condicionals en fases posteriors.

\subsubsection{Situació familiar i convivència}
A continuació, es recull informació sobre la composició de la llar:
\begin{itemize}
    \item Nombre total de persones que conviuran a l'habitatge
    \item Existència de fills menors i les seves edats
    \item Presència de persones grans
    \item Intenció de tenir fills en un futur proper
\end{itemize}

Aquest bloc permet detectar necessitats futures i requisits especials d'espai o accessibilitat.

\subsubsection{Pressupost}
El sistema sol·licita informació econòmica clau:
\begin{itemize}
    \item Pressupost màxim mensual
    \item Pressupost mínim desitjat
    \item Si hi ha flexibilitat de pressupost
\end{itemize}

Aquestes dades s'utilitzen per filtrar les opcions d'habitatge compatibles amb la capacitat econòmica del sol·licitant.

\subsubsection{Ubicació i mobilitat}
Per adaptar les recomanacions a la mobilitat de l'usuari, el sistema pregunta:
\begin{itemize}
    \item Si el lloc de treball o estudi es troba a la ciutat
    \item Disponibilitat de vehicle propi
    \item Necessitat de transport públic proper
\end{itemize}

Aquestes respostes influeixen directament en la selecció de zones i accessibilitat.

\subsubsection{Accessibilitat}
En funció de l'edat del sol·licitant o la presència de persones grans, el sistema determina:
\begin{itemize}
    \item Necessitat d'ascensor
    \item Requisits d'accessibilitat per a mobilitat reduïda
\end{itemize}

\subsubsection{Mascotes}
El sistema també considera la convivència amb animals:
\begin{itemize}
    \item Existència de mascotes
    \item Tipus de mascota
    \item Nombre total
\end{itemize}

\subsubsection{Preferències i exclusions}
Finalment, es recullen preferències respecte a serveis i entorns:
\begin{itemize}
    \item Serveis propers que es consideren desitjables
    \item Serveis o instal·lacions que es volen evitar
\end{itemize}

Aquestes últimes respostes són clau al nostre sistema. Dos usuaris que siguin categoritzats amb la mateixa categoria, tindràn ofertes diferents en gran part per el gran esforç del sistema per complir aquests dos condicionants

\subsubsection{Generació del perfil}
Un cop recollides totes les respostes, el sistema les guarda com a part del domini del problema. Constitueix la base de coneixement inicial sobre la qual treballarà el sistema d'inferència.




\subsection{Objectius del sistema}

Els objectius principals que ha d'assolir el nostre sistema són:

\subsubsection{Objectius funcionals}

\begin{enumerate}
    \item \textbf{Classificar sol·licitants automàticament}: A partir de les característiques bàsiques del sol·licitant (edat, nombre de persones, fills, situació laboral, etc.), el sistema ha d'inferir el seu perfil (persona gran, família amb fills, estudiants, parella jove, etc.) sense requerir que el propi usuari s'auto-classifiqui.
    
    \item \textbf{Inferir necessitats implícites}: El sistema ha de deduir requeriments que el sol·licitant potser no ha expressat explícitament. Per exemple, si el sol·licitant té fills petits, el sistema ha d'entendre que necessitarà escoles properes encara que no ho hagi demanat directament.
    
    \item \textbf{Descartar ofertes inadequades}: Abans de puntuar, el sistema ha d'aplicar filtres durs per eliminar ofertes que clarament no són adequades (fora de pressupost estricte, no permeten mascotes quan és imprescindible, no són accessibles quan cal, etc.).
    
    \item \textbf{Puntuar ofertes segons adequació}: Les ofertes que superen els filtres han de rebre una puntuació que reflecteixi el seu grau d'adequació global, considerant tant aspectes de l'habitatge com de l'entorn.
    
    \item \textbf{Classificar ofertes en graus de recomanació}: Basant-se en la puntuació, assignar cada oferta a una categoria: parcialment adequada, adequada o molt recomanable.
    
    \item \textbf{Explicar les recomanacions}: Per a cada oferta recomanada, el sistema ha de generar explicacions que indiquin per què és adequada (punts forts) i, en cas d'ofertes parcialment adequades, quins criteris no compleix plenament.
\end{enumerate}

\subsubsection{Objectius de qualitat}

\begin{enumerate}
    \item \textbf{Transparència}: Les decisions del sistema han de ser comprensibles i justificables. L'usuari ha de poder entendre per què una oferta està recomanada i una altra no.
    
    \item \textbf{Equitat}: El sistema no ha de discriminar de manera injustificada cap perfil de sol·licitant. Les regles han de reflectir preferències raonables, no biaixos arbitraris.
    
    \item \textbf{Cobertura}: El sistema ha de ser capaç de gestionar una àmplia varietat de perfils i ofertes, no només casos ideals o trivials.
    
    \item \textbf{Consistència}: Aplicat a situacions similars, el sistema ha de produir recomanacions coherents.
\end{enumerate}

\subsection{Resultats del sistema}

El sistema proporciona com a sortida una llista de recomanacions personalitzades per a cada sol·licitant, amb la següent informació:

\subsubsection{Per a cada sol·licitant}

\begin{itemize}
    \item \textbf{Top 3 d'ofertes recomanades}: El sistema presenta les tres millors ofertes ordenades per puntuació, facilitant la presa de decisió sense saturar l'usuari amb massa opcions.
    
    \item \textbf{Grau de recomanació}: Per a cada oferta, s'indica si és ``Parcialment adequada'', ``Adequada'' o ``Molt recomanable''.
    
    \item \textbf{Puntuació numèrica}: Tot i que l'usuari final veu principalment la classificació qualitativa, el sistema genera internament una puntuació numèrica que permet ordenar les ofertes amb precisió.
\end{itemize}

\subsubsection{Per a cada oferta recomanada}

\begin{itemize}
    \item \textbf{Característiques bàsiques}: Tipus d'habitatge, superfície, nombre de dormitoris i banys, preu mensual, adreça i districte.
    
    \item \textbf{Punts forts}: Llista de característiques positives que fan l'oferta especialment adequada per al sol·licitant concret. Per exemple: ``Té terrassa o balcó (+10p)'', ``Molt assolellat (+20p)'', ``Proximitat a escoles (inferida) (+20p)''.
    
    \item \textbf{Aspectes a considerar}: Per a ofertes parcialment adequades o adequades, s'indiquen els criteris que no es compleixen completament. Per exemple: ``Preu lleugerament superior al pressupost màxim (Moderat)'', ``Planta alta sense ascensor (Lleu)''.
\end{itemize}

\subsubsection{Informació complementària}

El sistema també genera informació interna útil per a depuració i anàlisi:

\begin{itemize}
    \item \textbf{Ofertes descartades}: Registre de quines ofertes s'han descartat i per quin motiu per a cada sol·licitant.
    
    \item \textbf{Requisits inferits}: Documentació de quines necessitats s'han deduït automàticament per a cada perfil.
    
    \item \textbf{Proximitats calculades}: Taula de distàncies entre cada habitatge i cada servei, classificades en molt a prop, distància mitjana o lluny.
\end{itemize}

Aquest disseny de sortida equilibra la utilitat per a l'usuari final (informació clara i accionable) amb la necessitat de transparència i explicabilitat que són fonamentals en un sistema basat en coneixement.
\end{document}