\section{Identificació del problema}

\subsection{Descripció del problema}

El mercat immobiliari de lloguer a Barcelona presenta una complexitat creixent que dificulta la cerca d'habitatge adequat per part dels ciutadans. La regidoria d'habitatge de l'ajuntament de Barcelona disposa d'un gran nombre d'ofertes de lloguer, però connectar aquestes ofertes amb les persones que busquen habitatge de manera eficient requereix un coneixement expert que tingui en compte múltiples factors simultàniament.

El problema que ens proposem resoldre va més enllà d'una simple cerca parametritzada. No es tracta només de filtrar ofertes per preu o nombre d'habitacions, sinó d'entendre les necessitats reals de cada tipus de sol·licitant i fer recomanacions intel·ligents que considerin el context complet de la seva situació vital. Per exemple, una família amb fills petits no només necessita un pis amb suficients habitacions, sinó que valora especialment la proximitat a escoles, zones verdes i serveis pediàtrics, mentre que evita zones amb soroll excessiu. Una persona gran, per contra, prioritza l'accessibilitat, la proximitat a centres de salut i comerços de proximitat. Aquestes preferències no sempre són explícites en la cerca inicial del sol·licitant, però un expert immobiliari les tindria en compte.

A més, el sistema ha d'anar més enllà de les característiques intrínseques de l'habitatge (superfície, nombre d'habitacions, preu) i considerar el seu entorn urbà. La localització de l'habitatge en relació amb serveis com transport públic, comerços, centres educatius, sanitaris i zones d'oci és crucial per determinar la seva adequació. Aquest coneixement territorial no és trivial: cal saber què significa estar "a prop" d'un servei segons el context (una persona gran considera prop el que està a 200m, mentre que un jove pot considerar acceptable 500m), i entendre quins serveis són molestos per a qui (una discoteca propera pot ser desitjable per a estudiants però inacceptable per a famílies amb nens petits).

El repte també inclou gestionar restriccions de diferent naturalesa. Algunes són absolutes (permetre mascotes quan el sol·licitant en té, accessibilitat per a persones amb mobilitat reduïda), altres són preferències fortes (pressupost màxim), i d'altres són desitjables però no imprescindibles (orientació solar, vistes). El sistema ha de ser capaç de distingir entre aquests nivells i generar recomanacions que s'adaptin al grau de flexibilitat de cada sol·licitant.

Finalment, el sistema no només ha de trobar habitatges que compleixin uns criteris mínims, sinó que ha de classificar-los en diferents graus de recomanació: des d'ofertes parcialment adequades (que compleixen la majoria de requisits però fallen en algun aspecte menor) fins a ofertes molt recomanables (que no només compleixen tots els requisits sinó que ofereixen avantatges addicionals). Aquesta classificació ha de venir acompanyada d'explicacions clares que permetin al sol·licitant entendre per què una oferta està recomanada i què la fa destacar o quins aspectes caldria considerar abans de prendre una decisió.

\subsection{Anàlisi de viabilitat}

Per determinar si aquest problema és adequat per ser resolt mitjançant un sistema basat en el coneixement, hem analitzat diverses dimensions de viabilitat.

\subsubsection{Viabilitat tècnica}

El problema reuneix les característiques fonamentals que fan viable la construcció d'un SBC:

\textbf{Existència de coneixement expert.} Existeix un coneixement expert clar en el domini immobiliari. Els agents immobiliaris experimentats desenvolupen una comprensió profunda de com emparedar perfils de clients amb habitatges adequats, considerant factors que van més enllà de les especificacions tècniques. Aquest coneixement inclou heurístiques del tipus ``les famílies amb fills petits valoren especialment la proximitat a escoles i parcs'' o ``les persones grans necessiten accessibilitat i serveis de salut propers'', que es poden formalitzar en regles.

\textbf{Domini delimitat.} Hem acotat el problema a la ciutat de Barcelona, amb tipus d'habitatge i serveis ben definits. Aquesta delimitació fa el problema tractable sense perdre la seva essència. No intentem resoldre el problema general de recomanació d'habitatges a nivell mundial, sinó que ens centrem en un context urbà específic amb característiques ben conegudes.

\textbf{Problema de classificació i recomanació.} El problema s'ajusta perfectament a una metodologia de classificació heurística: hem de classificar ofertes en categories de recomanació (parcialment adequades, adequades, molt recomanables) basant-nos en l'avaluació de múltiples criteris. Aquest tipus de problema és un dels més adequats per SBC, ja que podem descompondre'l en subproblemes (abstracció del sol·licitant, càlcul de proximitats, descart d'ofertes, puntuació, classificació) que es resolen seqüencialment aplicant regles.

\textbf{Espai de solucions finit.} Tot i que l'espai de combinacions possibles entre sol·licitants i ofertes és gran, és finit i manejable. Amb N sol·licitants i M ofertes, tenim N×M parells a avaluar, però cada avaluació és independent i es pot resoldre aplicant un conjunt de regles ben definides.

\subsubsection{Viabilitat de desenvolupament}

El projecte és viable des del punt de vista del desenvolupament per diverses raons:

\textbf{Disponibilitat d'eines.} CLIPS proporciona un entorn robust per implementar sistemes basats en regles, amb suport per a programació orientada a objectes (COOL) que ens permet representar l'ontologia del domini de manera natural. Protégé ens ha permès dissenyar i documentar l'ontologia de forma sistemàtica abans de la implementació.

\textbf{Desenvolupament incremental.} El problema es pot abordar de manera incremental, començant amb un conjunt bàsic de regles i ampliant progressivament la cobertura. Hem pogut començar amb la gestió de restriccions dures (preu, mascotes, accessibilitat) i anar afegint regles més sofisticades per a la puntuació i classificació.

\textbf{Prototipatge ràpid.} La naturalesa declarativa de CLIPS permet fer prototips ràpidament i iterar sobre el disseny. Modificar o afegir regles no requereix reescriure grans porcions de codi, cosa que facilita l'experimentació i refinament.

\subsubsection{Limitacions identificades}

Tot i la viabilitat general, hem identificat algunes limitacions que cal tenir presents:

\textbf{Coneixement incomplet del domini.} No som experts immobiliaris reals, per la qual cosa el nostre sistema es basa principalment en coneixement de sentit comú i en patrons generals que hem pogut inferir. Un sistema de producció requeriria la participació activa d'experts del sector per afinar les regles i els pesos de puntuació.

\textbf{Dades sintètiques.} Les instàncies que utilitzem són simulades. Un sistema real necessitaria integrar-se amb bases de dades reals d'ofertes i disposar d'informació actualitzada sobre serveis urbans. També caldria considerar la dinàmica temporal (ofertes que deixen d'estar disponibles, preus que canvien).

\textbf{Absència de retroalimentació.} El sistema actual no aprèn de les decisions dels usuaris. No sabem si les recomanacions que fem són realment útils o si els usuaris acaben escollint opcions diferents de les que el sistema proposa com a millors. Un sistema real hauria d'incorporar mecanismes de feedback per ajustar els pesos i les regles.

\textbf{Factors subjectius.} Alguns factors que influeixen en la decisió d'escollir un habitatge són altament subjectius i difícils de codificar en regles (l'estètica del barri, la "sensació" que transmet un habitatge, factors culturals o personals molt específics). El nostre sistema se centra en factors objectius o preferències generals, però no pot capturar aquestes nuances individuals.

\subsection{Fonts de coneixement}

Per construir el sistema hem identificat i utilitzat diverses fonts de coneixement:

\subsubsection{Fonts primàries}

\textbf{Coneixement de sentit comú.} La base principal del nostre sistema prové del sentit comú sobre necessitats habitacionals segons perfils demogràfics. Aquest coneixement inclou afirmacions com "les persones grans necessiten accessibilitat" o "els estudiants valoren la proximitat al transport públic", que són àmpliament acceptades i no requereixen expertesa específica.

\textbf{Normativa d'accessibilitat.} Hem consultat els requisits bàsics d'accessibilitat (presència d'ascensor en plantes altes, accés sense barreres) que són estàndards regulats i objectius.

\textbf{Webs d'anuncis immobiliaris.} Plataformes com Idealista ens han servit per identificar les característiques rellevants que es descriuen en les ofertes reals (superfície, nombre d'habitacions, si permet mascotes, si té terrassa, consum energètic, etc.). També ens han permès veure quines són les categories de serveis que es mencionen habitualment com a punts forts de les ubicacions.

\subsubsection{Fonts secundàries}

\textbf{Models de llenguatge (opcional).} Tal com s'indica a l'enunciat, podríem haver utilitzat models de llenguatge com a "experts" per fer elicitació de coneixement sobre criteris de decisió en la recomanació d'habitatges. Tot i que no ho hem documentat extensament en aquest apartat, seria una via legítima per obtenir coneixement estructurat sobre el domini.

\textbf{Experiència personal.} Els membres de l'equip hem aplicat la nostra pròpia experiència en la cerca d'habitatge i coneixement de la ciutat de Barcelona per definir proximitats raonables, serveis rellevants i preferències típiques.

\textbf{Ontologies existents.} Hem consultat exemples d'ontologies del domini immobiliari i de recomanació de serveis per estructurar adequadament els conceptes i relacions del nostre sistema.

\subsection{Objectius del sistema}

Els objectius principals que ha d'assolir el nostre sistema són:

\subsubsection{Objectius funcionals}

\begin{enumerate}
    \item \textbf{Classificar sol·licitants automàticament}: A partir de les característiques bàsiques del sol·licitant (edat, nombre de persones, fills, situació laboral, etc.), el sistema ha d'inferir el seu perfil (persona gran, família amb fills, estudiants, parella jove, etc.) sense requerir que el propi usuari s'auto-classifiqui.
    
    \item \textbf{Inferir necessitats implícites}: El sistema ha de deduir requeriments que el sol·licitant potser no ha expressat explícitament. Per exemple, si el sol·licitant té fills petits, el sistema ha d'entendre que necessitarà escoles properes encara que no ho hagi demanat directament.
    
    \item \textbf{Descartar ofertes inadequades}: Abans de puntuar, el sistema ha d'aplicar filtres durs per eliminar ofertes que clarament no són adequades (fora de pressupost estricte, no permeten mascotes quan és imprescindible, no són accessibles quan cal, etc.).
    
    \item \textbf{Puntuar ofertes segons adequació}: Les ofertes que superen els filtres han de rebre una puntuació que reflecteixi el seu grau d'adequació global, considerant tant aspectes de l'habitatge com de l'entorn.
    
    \item \textbf{Classificar ofertes en graus de recomanació}: Basant-se en la puntuació, assignar cada oferta a una categoria: parcialment adequada, adequada o molt recomanable.
    
    \item \textbf{Explicar les recomanacions}: Per a cada oferta recomanada, el sistema ha de generar explicacions que indiquin per què és adequada (punts forts) i, en cas d'ofertes parcialment adequades, quins criteris no compleix plenament.
\end{enumerate}

\subsubsection{Objectius de qualitat}

\begin{enumerate}
    \item \textbf{Transparència}: Les decisions del sistema han de ser comprensibles i justificables. L'usuari ha de poder entendre per què una oferta està recomanada i una altra no.
    
    \item \textbf{Equitat}: El sistema no ha de discriminar de manera injustificada cap perfil de sol·licitant. Les regles han de reflectir preferències raonables, no biaixos arbitraris.
    
    \item \textbf{Cobertura}: El sistema ha de ser capaç de gestionar una àmplia varietat de perfils i ofertes, no només casos ideals o trivials.
    
    \item \textbf{Consistència}: Aplicat a situacions similars, el sistema ha de produir recomanacions coherents.
\end{enumerate}

\subsection{Resultats del sistema}

El sistema proporciona com a sortida una llista de recomanacions personalitzades per a cada sol·licitant, amb la següent informació:

\subsubsection{Per a cada sol·licitant}

\begin{itemize}
    \item \textbf{Top 3 d'ofertes recomanades}: El sistema presenta les tres millors ofertes ordenades per puntuació, facilitant la presa de decisió sense saturar l'usuari amb massa opcions.
    
    \item \textbf{Grau de recomanació}: Per a cada oferta, s'indica si és "Parcialment adequada", "Adequada" o "Molt recomanable".
    
    \item \textbf{Puntuació numèrica}: Tot i que l'usuari final veu principalment la classificació qualitativa, el sistema genera internament una puntuació numèrica que permet ordenar les ofertes amb precisió.
\end{itemize}

\subsubsection{Per a cada oferta recomanada}

\begin{itemize}
    \item \textbf{Característiques bàsiques}: Tipus d'habitatge, superfície, nombre de dormitoris i banys, preu mensual, adreça i districte.
    
    \item \textbf{Punts forts}: Llista de característiques positives que fan l'oferta especialment adequada per al sol·licitant concret. Per exemple: "Té terrassa o balcó (+10p)", "Molt assolellat (+20p)", "Proximitat a escoles (inferida) (+20p)".
    
    \item \textbf{Aspectes a considerar}: Per a ofertes parcialment adequades o adequades, s'indiquen els criteris que no es compleixen completament. Per exemple: "Preu lleugerament superior al pressupost màxim (Moderat)", "Planta alta sense ascensor (Lleu)".
\end{itemize}

\subsubsection{Informació complementària}

El sistema també genera informació interna útil per a depuració i anàlisi:

\begin{itemize}
    \item \textbf{Ofertes descartades}: Registre de quines ofertes s'han descartat i per quin motiu per a cada sol·licitant.
    
    \item \textbf{Requisits inferits}: Documentació de quines necessitats s'han deduït automàticament per a cada perfil.
    
    \item \textbf{Proximitats calculades}: Taula de distàncies entre cada habitatge i cada servei, classificades en molt a prop, distància mitjana o lluny.
\end{itemize}

Aquest disseny de sortida equilibra la utilitat per a l'usuari final (informació clara i accionable) amb la necessitat de transparència i explicabilitat que són fonamentals en un sistema basat en coneixement.