% 04_implementacio.tex
% Capítol 4: Implementació

\section{Arquitectura del Sistema}

\subsection{Estructura de Fitxers}

El sistema està organitzat en múltiples fitxers per modularitat:

\begin{verbatim}
src/
├── ontologiaSBC.clp          # Classes de l'ontologia (generat)
├── ontologiaSBC.owl          # Ontologia en OWL/XML
├── ontologiaSBC.ttl          # Ontologia en Turtle
├── regles.clp                # Regles del sistema expert
├── instancies.clp            # Instàncies bàsiques de prova
├── instancies_extended.clp   # Instàncies addicionals (100+)
├── main.clp                  # Interfície interactiva
└── run.clp                   # Script de càrrega
\end{verbatim}

\subsection{Flux d'Execució}

\begin{enumerate}
    \item \textbf{Càrrega}: \texttt{(load "run.clp")} carrega tots els fitxers
    \item \textbf{Reset}: \texttt{(reset)} crea instàncies inicials
    \item \textbf{Execució}:
    \begin{itemize}
        \item Opció A: \texttt{(main)} - Interfície interactiva
        \item Opció B: \texttt{(run)} - Processa instàncies predefinides
    \end{itemize}
    \item \textbf{Processament}: Sistema avalua automàticament totes les combinacions
    \item \textbf{Sortida}: Recomanacions amb justificacions
\end{enumerate}

\section{Implementació de l'Ontologia}

\subsection{Generació amb Protégé i owl2clips}

L'ontologia s'ha creat visualment amb Protégé 5.6 i s'ha convertit a CLIPS:

\begin{lstlisting}[language=bash]
# Proces de conversio
1. Diseñar ontologia en Protege (ontologiaSBC.owl)
2. Exportar a Turtle (ontologiaSBC.ttl)  
3. Convertir con owl2clips:
   java -jar owl2clips.jar ontologiaSBC.ttl > ontologiaSBC.clp
\end{lstlisting}

\subsection{Exemple de Classe Generada}

\begin{lstlisting}
(defclass Habitatge
    (is-a USER)
    (role concrete)
    (pattern-match reactive)
    (slot superficieHabitable
        (type FLOAT)
        (create-accessor read-write))
    (slot numeroDormitoris
        (type INTEGER)
        (create-accessor read-write))
    (slot permetMascotes
        (type SYMBOL)
        (create-accessor read-write))
    (slot teLocalitzacio
        (type INSTANCE)
        (create-accessor read-write))
    ; ... més slots
)

(defclass Pis
    (is-a Habitatge)
    (role concrete)
    (pattern-match reactive))
\end{lstlisting}

\section{Metodologia de Desenvolupament}

\subsection{Prototipatge Ràpid i Disseny Incremental}

S'ha seguit un procés iteratiu:

\subsubsection{Iteració 1: Prototip Bàsic (Setmana 3)}

\textbf{Objectiu}: Sistema mínim funcional

\textbf{Característiques}:
\begin{itemize}
    \item 3 classes: Habitatge, Sol·licitant, Oferta
    \item 5 regles bàsiques (preu, mascotes, superfície)
    \item 3 instàncies de prova
    \item Sortida: Recomanat/No Recomanat
\end{itemize}

\textbf{Lliçons apreses}:
\begin{itemize}
    \item Necessitat de jerarquies (no tots els habitatges són iguals)
    \item Falta explicabilitat (per què es recomana?)
    \item Necessitat de fases separades
\end{itemize}

\subsubsection{Iteració 2: Ampliació Ontologia (Setmana 4)}

\textbf{Millores}:
\begin{itemize}
    \item Jerarquia completa de Sol·licitant (7 subclasses)
    \item Jerarquia de Servei (6 categories)
    \item Relació de proximitat
    \item 10 instàncies de prova
    \item Sistema de tres fases inicial
\end{itemize}

\textbf{Problemes detectats}:
\begin{itemize}
    \item Càlcul de distàncies ineficient
    \item Regles massa específiques (difícils de mantenir)
\end{itemize}

\subsubsection{Iteració 3: Refinament Regles (Setmana 5)}

\textbf{Millores}:
\begin{itemize}
    \item Templates auxiliars (proximitat, requisit-inferit, etc.)
    \item 30+ regles organitzades per fase
    \item Sistema de salience per control de flux
    \item Explicacions detallades
    \item 20+ instàncies de prova
\end{itemize}

\subsubsection{Iteració 4: Proves Exhaustives (Setmana 6)}

\textbf{Millores}:
\begin{itemize}
    \item 100+ instàncies de prova diverses
    \item Interfície interactiva (\texttt{main.clp})
    \item Detecció i correcció de casos extrems
    \item Optimització de regles
    \item Documentació completa
\end{itemize}

\subsection{Planificació i Divisió de Tasques}

\begin{table}[h]
\centering
\small
\begin{tabular}{|l|l|l|}
\hline
\textbf{Setmana} & \textbf{Responsable} & \textbf{Tasca} \\
\hline
1 & Tots & Anàlisi i comprensió enunciat \\
2 & Est. 1 & Disseny ontologia Protégé \\
  & Est. 2 & Elicitació coneixement (LLM) \\
  & Est. 3 & Recerca webs immobiliàries \\
\hline
3 & Est. 1 & Implementació classes CLIPS \\
  & Est. 2 & Regles fase abstracció \\
  & Est. 3 & Instàncies de prova \\
\hline
4 & Est. 1 & Regles fase resolució \\
  & Est. 2 & Regles fase refinació \\
  & Est. 3 & Més instàncies + proves \\
\hline
5 & Est. 1 & Interfície interactiva \\
  & Est. 2 & Jocs de prova finals \\
  & Est. 3 & Inici documentació \\
\hline
6 & Tots & Documentació final i memòria \\
\hline
\end{tabular}
\caption{Distribució de tasques}
\end{table}

\section{Modularització del Sistema}

\subsection{Separació en Mòduls Conceptuals}

Tot i que CLIPS no té mòduls explícits com altres llenguatges, s'ha aconseguit modularitat mitjançant:

\begin{enumerate}
    \item \textbf{Separació per fitxers}: Ontologia, regles, instàncies
    \item \textbf{Separació per fases}: Abstracció, Resolució, Refinació
    \item \textbf{Convenció de noms}: \texttt{fase-funcio-descripcio}
    \item \textbf{Templates especialitzats}: Per cada tipus de raonament
\end{enumerate}

\subsection{Mòdul d'Abstracció}

\textbf{Responsabilitat}: Inferir requisits implícits

\textbf{Regles} (prefixades amb \texttt{abstraccio-}):
\begin{itemize}
    \item \texttt{abstraccio-calcular-proximitats}
    \item \texttt{abstraccio-familia-amb-fills}
    \item \texttt{abstraccio-persona-gran}
    \item \texttt{abstraccio-estudiants}
    \item \texttt{abstraccio-prefereix-transport}
    \item \texttt{abstraccio-parella-futurs-fills}
\end{itemize}

\textbf{Entrada}: Instàncies de Sol·licitant, Habitatge, Servei

\textbf{Sortida}: Fets \texttt{proximitat} i \texttt{requisit-inferit}

\subsection{Mòdul de Resolució}

\textbf{Responsabilitat}: Avaluar ofertes i detectar pros/contres

\textbf{Sub-mòduls}:

\paragraph{Filtratge (Descart)}:
\begin{itemize}
    \item \texttt{resolucio-descartar-preu-excessiu}
    \item \texttt{resolucio-descartar-no-mascotes}
    \item \texttt{resolucio-descartar-no-accessible}
    \item \texttt{resolucio-descartar-servei-evitat}
    \item \texttt{resolucio-descartar-falta-requisit-inferit}
\end{itemize}

\paragraph{Detecció de Criteris Negatius}:
\begin{itemize}
    \item \texttt{resolucio-criteri-preu-alt}
    \item \texttt{resolucio-criteri-soroll-alt}
    \item \texttt{resolucio-criteri-sense-ascensor}
\end{itemize}

\paragraph{Detecció de Punts Positius}:
\begin{itemize}
    \item \texttt{resolucio-punt-bon-preu}
    \item \texttt{resolucio-punt-terrassa}
    \item \texttt{resolucio-punt-assolellat}
    \item \texttt{resolucio-punt-transport-aprop}
    \item (10+ regles més)
\end{itemize}

\subsection{Mòdul de Refinació}

\textbf{Responsabilitat}: Classificar ofertes en categories finals

\textbf{Regles}:
\begin{itemize}
    \item \texttt{refinacio-molt-recomanable}
    \item \texttt{refinacio-adequat}
    \item \texttt{refinacio-parcialment}
\end{itemize}

\textbf{Entrada}: Fets \texttt{punt-positiu}, \texttt{criteri-no-cumplit}

\textbf{Sortida}: Fets \texttt{recomanacio}

\subsection{Mòdul de Presentació}

\textbf{Responsabilitat}: Mostrar resultats a l'usuari

\textbf{Regles}:
\begin{itemize}
    \item \texttt{presentacio-inici}
    \item \texttt{presentacio-recomanacio}
    \item \texttt{presentacio-punts-positius}
    \item \texttt{presentacio-criteris-negatius}
    \item \texttt{presentacio-descartades}
\end{itemize}

\section{Detalls d'Implementació Clau}

\subsection{Càlcul de Distàncies}

\begin{lstlisting}
(deffunction calcular-distancia (?x1 ?y1 ?x2 ?y2)
    (sqrt (+ (** (- ?x2 ?x1) 2) (** (- ?y2 ?y1) 2))))

(deffunction classificar-distancia (?metres)
    (if (< ?metres 500.0) then MoltAProp
    else (if (< ?metres 1000.0) then DistanciaMitjana
    else Lluny)))

(defrule abstraccio-calcular-proximitats
    (declare (salience 100))
    ?hab <- (object (is-a Habitatge) (teLocalitzacio ?locH))
    ?locHab <- (object (is-a Localitzacio) (name ?locH) 
                (coordenadaX ?x1) (coordenadaY ?y1))
    ?serv <- (object (is-a Servei) (teLocalitzacio ?locS))
    ?locServ <- (object (is-a Localitzacio) (name ?locS) 
                 (coordenadaX ?x2) (coordenadaY ?y2))
    (not (proximitat (habitatge ?hab) (servei ?serv)))
    =>
    (bind ?metres (calcular-distancia ?x1 ?y1 ?x2 ?y2))
    (bind ?dist (classificar-distancia ?metres))
    (bind ?cat (class ?serv))
    (assert (proximitat (habitatge ?hab) (servei ?serv) 
            (categoria ?cat) (distancia ?dist) (metres ?metres))))
\end{lstlisting}

\textbf{Optimització}: Càlcul únic al principi, reutilitzat després.

\subsection{Inferència de Requisits}

\begin{lstlisting}
(defrule abstraccio-familia-amb-fills
    (declare (salience 95))
    ?sol <- (object (is-a Solicitant) (numeroFills ?fills))
    (test (> ?fills 0))
    (not (requisit-inferit (solicitant ?sol) 
          (categoria-servei ServeiEducatiu)))
    =>
    (assert (requisit-inferit 
        (solicitant ?sol) 
        (categoria-servei ServeiEducatiu)
        (obligatori si) 
        (motiu "Familia amb fills necessita escoles")))
    (assert (requisit-inferit 
        (solicitant ?sol) 
        (categoria-servei ZonaVerda)
        (obligatori no) 
        (motiu "Zona verda per fills"))))
\end{lstlisting}

\textbf{Patró NOT}: Evita duplicar requisits inferits.

\subsection{Descart amb Justificació}

\begin{lstlisting}
(defrule resolucio-descartar-no-mascotes
    (declare (salience 40))
    (fase-completada (nom abstraccio))
    ?sol <- (object (is-a Solicitant) (teMascotes si))
    ?of <- (object (is-a Oferta) (teHabitatge ?hab))
    ?h <- (object (is-a Habitatge) (name ?hab) (permetMascotes no))
    (not (oferta-descartada (solicitant ?sol) (oferta ?of)))
    =>
    (assert (oferta-descartada 
        (solicitant ?sol) 
        (oferta ?of)
        (motiu "No permet mascotes")))
    (printout t "[RESOLUCIO] DESCARTADA " (instance-name ?of) 
              " - No mascotes" crlf))
\end{lstlisting}

\textbf{Patró}: Comprovar no descartada prèviament evita duplicats.

\subsection{Classificació Multi-criteri}

\begin{lstlisting}
(defrule refinacio-molt-recomanable
    (declare (salience 5))
    (fase-completada (nom resolucio))
    ?sol <- (object (is-a Solicitant))
    ?of <- (object (is-a Oferta))
    (not (oferta-descartada (solicitant ?sol) (oferta ?of)))
    (not (criteri-no-cumplit (solicitant ?sol) (oferta ?of)))
    (punt-positiu (solicitant ?sol) (oferta ?of) (descripcio ?d1))
    (punt-positiu (solicitant ?sol) (oferta ?of) (descripcio ?d2&~?d1))
    (punt-positiu (solicitant ?sol) (oferta ?of) (descripcio ?d3&~?d1&~?d2))
    (not (recomanacio (solicitant ?sol) (oferta ?of)))
    =>
    (assert (recomanacio (solicitant ?sol) (oferta ?of) 
            (grau MoltRecomanable) (puntuacio 100))))
\end{lstlisting}

\textbf{Patró}: \texttt{\&\textasciitilde} assegura que els 3 punts positius són diferents.

\section{Interfície Interactiva}

\subsection{Funcionalitat de main.clp}

El fitxer \texttt{main.clp} proporciona una interfície per crear perfils dinàmicament:

\begin{lstlisting}
(deffunction main ()
    (printout t "=== SISTEMA EXPERT DE RECOMANACIO ===" crlf)
    
    (bind ?crear-nou (pregunta-si-no 
          "Vols crear un nou perfil de solicitant?"))
    
    (if (eq ?crear-nou si) then
        (bind ?perfil (crear-perfil-solicitant))
        (reset)
        (run)
    else
        (printout t "Usa perfils predefinits" crlf)))
\end{lstlisting}

\subsection{Funcions Auxiliars d'Entrada}

\begin{lstlisting}
(deffunction pregunta-si-no (?pregunta)
    (printout t ?pregunta " (si/no): ")
    (bind ?resp (read))
    (while (and (neq ?resp si) (neq ?resp no))
        (printout t "Si us plau, respon 'si' o 'no': ")
        (bind ?resp (read)))
    ?resp)

(deffunction pregunta-numero (?pregunta ?min ?max)
    (printout t ?pregunta " [" ?min "-" ?max "]: ")
    (bind ?resp (read))
    (while (or (not (numberp ?resp)) 
               (< ?resp ?min) (> ?resp ?max))
        (printout t "Numero entre " ?min " i " ?max ": ")
        (bind ?resp (read)))
    ?resp)
\end{lstlisting}

\section{Gestió d'Errors i Casos Extrems}

\subsection{Casos Extrems Considerats}

\begin{enumerate}
    \item \textbf{Pressupost 0}: Sistema demana pressupost vàlid
    \item \textbf{Cap oferta adequada}: Sistema ho indica clarament
    \item \textbf{Totes les ofertes descartades}: Mostra motius
    \item \textbf{Coordenades fora de rang}: Classificat com "Lluny"
    \item \textbf{Atributs nulls}: Regles amb (test ...) per validar
\end{enumerate}

\subsection{Validacions Implementades}

\begin{lstlisting}
;; Validar que l'oferta esta disponible
?of <- (object (is-a Oferta) (disponible si))

;; Validar que no s'ha descartat previament
(not (oferta-descartada (solicitant ?sol) (oferta ?of)))

;; Validar rangs numerics
(test (> ?fills 0))
(test (< ?metres 500.0))

;; Validar existencia d'instancies
?h <- (object (is-a Habitatge) (name ?hab))
\end{lstlisting}

\section{Optimitzacions Realitzades}

\begin{enumerate}
    \item \textbf{Pre-càlcul de proximitats}: Evita recalcular a cada regla
    \item \textbf{Patró NOT per evitar duplicats}: \texttt{(not (requisit-inferit ...))}
    \item \textbf{Salience per ordre eficient}: Fases seqüencials
    \item \textbf{Descart primerenc}: Ofertes inadequades eliminades abans d'avaluar detalls
    \item \textbf{Pattern matching eficient}: Condicions més restrictives primer
\end{enumerate}

\section{Limitacions Tècniques}

\begin{itemize}
    \item \textbf{Escalabilitat}: 100+ ofertes x 50+ sol·licitants funciona bé, 1000+ seria lent
    \item \textbf{Coordenades simplificades}: Sistema cartesià 2D, no GPS real
    \item \textbf{Absència de learning}: No millora amb ús
    \item \textbf{Regles fixes}: Canviar criteris requereix modificar codi
\end{itemize}