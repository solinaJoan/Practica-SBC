% Capítol 4: Implementació

\section{Implementació}
\label{sec:implementacio}

En aquest capítol es descriu com s'ha implementat el sistema basat en el coneixement utilitzant CLIPS com a llenguatge de programació. S'explica l'arquitectura del sistema, com s'ha traduït l'ontologia a estructures CLIPS, com s'han implementat les regles i quina metodologia de desenvolupament s'ha seguit.

\subsection{Arquitectura del sistema}

El sistema s'ha estructurat de manera modular per facilitar el desenvolupament, el manteniment i la comprensió del codi. Aquesta modularitat s'ha aconseguit mitjançant la separació en múltiples fitxers i l'organització clara de les responsabilitats.

\subsubsection{Estructura de fitxers}

El sistema està organitzat en els següents fitxers:

\begin{verbatim}
src/
|-- ontologiaSBC.clp          # Classes de l'ontologia
|-- ontologiaSBC.owl          # Ontologia original en OWL/XML
|-- ontologiaSBC.ttl          # Ontologia en format Turtle
|-- regles.clp                # Regles del sistema expert
|-- instancies.clp            # Instàncies bàsiques de prova
|-- instancies_extended.clp   # Instàncies addicionals (100+)
|-- main.clp                  # Interfície interactiva
`-- run.clp                   # Script de càrrega i execució
\end{verbatim}

Cada fitxer té una responsabilitat específica:

\begin{itemize}
    \item \textbf{ontologiaSBC.clp}: Conté les definicions de totes les classes de l'ontologia traduïdes a CLIPS amb COOL (CLIPS Object-Oriented Language). Aquest fitxer es genera automàticament a partir de l'ontologia OWL utilitzant l'eina owl2clips.
    
    \item \textbf{ontologiaSBC.owl i ontologiaSBC.ttl}: Són versions de l'ontologia en formats estàndard. OWL/XML és el format natiu de Protégé, mentre que Turtle és més llegible per a humans i útil per a documentació.
    
    \item \textbf{regles.clp}: Conté totes les regles del sistema expert organitzades per fases. Aquest és el fitxer més gran i complex, amb més de 30 regles que implementen la lògica de raonament.
    
    \item \textbf{instancies.clp}: Conté un conjunt bàsic d'instàncies de prova (uns 10-15 casos) que cobreixen els escenaris principals. S'utilitza per proves ràpides durant el desenvolupament.
    
    \item \textbf{main.clp}: Implementa una interfície interactiva que permet a l'usuari crear perfils de sol·licitant dinàmicament i obtenir recomanacions personalitzades.
    
    \item \textbf{run.clp}: Script que carrega tots els fitxers necessaris en l'ordre correcte i inicialitza el sistema.
\end{itemize}

\subsubsection{Flux d'execució}

L'execució del sistema segueix sempre la mateixa seqüència:

\begin{enumerate}
    \item \textbf{Càrrega}: L'usuari executa \texttt{(load "run.clp")} que carrega automàticament tots els fitxers necessaris en l'ordre correcte: primer l'ontologia, després les regles i finalment les instàncies.
    
    \item \textbf{Reset}: La comanda \texttt{(reset)} inicialitza la memòria de treball de CLIPS, creant totes les instàncies definides i posant el sistema en l'estat inicial.
    
    \item \textbf{Execució}: Hi ha dues opcions per executar el sistema:
    \begin{itemize}
        \item \textbf{Mode automàtic}: La comanda \texttt{(run)} processa automàticament totes les combinacions de sol·licitants i ofertes definides en els fitxers d'instàncies, generant recomanacions per a cadascuna.
        
        \item \textbf{Mode interactiu}: La comanda \texttt{(main)} llança una interfície que fa preguntes a l'usuari per crear un perfil de sol·licitant personalitzat i després genera recomanacions només per a aquest perfil.
    \end{itemize}
    
    \item \textbf{Processament}: El sistema executa seqüencialment les tres fases de raonament (Abstracció, Resolució, Refinació) per a cada combinació sol·licitant-oferta.
    
    \item \textbf{Presentació de resultats}: Un cop completat el raonament, regles de presentació mostren les recomanacions amb les seves justificacions.
\end{enumerate}

Aquest flux garanteix que el sistema sempre opera de manera consistent i predictible.

\subsection{Traducció de l'ontologia a CLIPS}

L'ontologia dissenyada en Protégé s'ha traduït a CLIPS utilitzant l'extensió COOL (CLIPS Object-Oriented Language), que proporciona capacitats de programació orientada a objectes.

\subsubsection{Procés de generació}

La conversió de l'ontologia OWL a CLIPS s'ha realitzat mitjançant l'eina owl2clips, seguint aquest procés:

\begin{enumerate}
    \item Disseny de l'ontologia en Protégé 5.6, definint classes, atributs, jerarquies i relacions de manera visual.
    
    \item Exportació a format Turtle (TTL), que és més adequat per a la conversió automàtica que el format OWL/XML.
    
    \item Execució de l'eina owl2clips per generar automàticament les definicions CLIPS:
    \begin{verbatim}
    java -jar owl2clips.jar ontologiaSBC.ttl > ontologiaSBC.clp
    \end{verbatim}
    
    \item Revisió i ajust manual del fitxer generat per assegurar que les definicions són correctes i afegir documentació.
\end{enumerate}

Aquest procés automatitzat garanteix que l'ontologia en CLIPS és fidel al disseny original i estalvia la feina d'haver de traduir manualment totes les classes i atributs.

\subsubsection{Estructura de les classes generades}

Les classes de l'ontologia es tradueixen a \texttt{defclass} de COOL. A continuació es mostra un exemple d'una classe generada:

\begin{lstlisting}[caption={Exemple de classe Habitatge en CLIPS}]
(defclass Habitatge
    (is-a USER)
    (role concrete)
    (pattern-match reactive)
    (slot superficieHabitable
        (type FLOAT)
        (create-accessor read-write))
    (slot numeroDormitoris
        (type INTEGER)
        (create-accessor read-write))
    (slot numeroDormitorisDobles
        (type INTEGER)
        (create-accessor read-write))
    (slot numeroDormitorisSimples
        (type INTEGER)
        (create-accessor read-write))
    (slot permetMascotes
        (type SYMBOL)
        (allowed-symbols si no)
        (create-accessor read-write))
    (slot teLocalitzacio
        (type INSTANCE)
        (create-accessor read-write))
    ; ... més slots ...
)
\end{lstlisting}

Els elements clau d'aquesta definició són:

\begin{itemize}
    \item \textbf{(is-a USER)}: Indica que és una classe concreta que hereta de la classe arrel USER de COOL.
    
    \item \textbf{(role concrete)}: Especifica que es poden crear instàncies d'aquesta classe (no és abstracta).
    
    \item \textbf{(pattern-match reactive)}: Activa el pattern matching reactiu, essencial per al forward chaining.
    
    \item \textbf{Slots}: Cada atribut de l'ontologia es tradueix a un slot amb el seu tipus corresponent i accessors.
\end{itemize}

Les subclasses mantenen la jerarquia mitjançant \texttt{is-a}:

\begin{lstlisting}[caption={Exemple de subclasse}]
(defclass Pis
    (is-a Habitatge)
    (role concrete)
    (pattern-match reactive))
\end{lstlisting}

Una subclasse hereta automàticament tots els slots de la seva superclasse, per la qual cosa només cal especificar slots addicionals si n'hi ha.

\subsection{Metodologia de desenvolupament}

El desenvolupament del sistema ha seguit una metodologia de prototipatge ràpid i disseny incremental, tal com es recomana per a sistemes basats en el coneixement. Aquesta aproximació permet validar decisions de disseny aviat i ajustar el sistema de manera iterativa.

\subsubsection{Iteració 1: Prototip mínim funcional}

\textbf{Setmana 3 - Objectiu}: Crear un sistema mínim que demostri la viabilitat de l'aproximació.

\textbf{Característiques implementades}:
\begin{itemize}
    \item Ontologia simplificada amb només 3 classes: Habitatge (sense subclasses), Sol·licitant (sense subclasses) i Oferta.
    
    \item 5 regles bàsiques que implementen els criteris més evidents:
    \begin{itemize}
        \item Descart per preu excessiu
        \item Descart si no permet mascotes
        \item Descart per superfície insuficient
        \item Classificació com a "Recomanat" si passa tots els filtres
        \item Classificació com a "No Recomanat" si es descarta
    \end{itemize}
    
    \item 3 instàncies de prova: un sol·licitant i dues ofertes (una adequada i una inadequada).
    
    \item Sortida molt simple: només "Recomanat" o "No Recomanat" sense explicacions detallades.
\end{itemize}

\textbf{Resultats de la iteració}:

El prototip va funcionar correctament i va demostrar que l'aproximació amb CLIPS era viable. No obstant això, es van identificar diverses limitacions:

\begin{itemize}
    \item La manca de jerarquies de classes feia que les regles fossin massa genèriques i no poguessin capturar diferències importants entre tipus d'habitatges o perfils de sol·licitants.
    
    \item Les recomanacions binàries (recomanat/no recomanat) eren massa simplistes. Calia una classificació més matisada.
    
    \item La falta d'explicacions feia el sistema una "caixa negra" poc útil per a l'usuari.
    
    \item No hi havia cap mecanisme per inferir necessitats implícites del perfil del sol·licitant.
\end{itemize}

\textbf{Decisions preses}:

A partir d'aquestes observacions es va decidir:
\begin{enumerate}
    \item Ampliar l'ontologia amb jerarquies completes de classes.
    \item Implementar un sistema de tres nivells de recomanació (Molt Recomanable, Adequat, Parcialment Adequat).
    \item Afegir mecanismes per capturar justificacions de les decisions.
    \item Implementar regles d'inferència que dedueixin necessitats del perfil.
\end{enumerate}

\subsubsection{Iteració 2: Ampliació de l'ontologia i millora de regles}

\textbf{Setmana 4 - Objectiu}: Implementar una ontologia completa i millorar la qualitat de les recomanacions.

\textbf{Millores implementades}:
\begin{itemize}
    \item Jerarquia completa de Sol·licitant amb 7 subclasses que cobreixen els principals perfils demogràfics.
    
    \item Jerarquia de Servei amb 6 categories principals i més de 20 subclasses.
    
    \item Sistema de proximitat que calcula i classifica distàncies entre habitatges i serveis.
    
    \item 10 instàncies de prova que cobreixen diversos escenaris diferents.
    
    \item Implementació inicial del sistema de tres fases (Abstracció, Resolució, Refinació) encara que amb poques regles en cada fase.
    
    \item Primers mecanismes per capturar punts positius i negatius de les ofertes.
\end{itemize}

\textbf{Problemes detectats durant la iteració}:

Tot i les millores significatives, es van trobar nous problemes:

\begin{itemize}
    \item El càlcul de distàncies es feia múltiples vegades dins de diferents regles, afectant l'eficiència.
    
    \item Algunes regles eren massa específiques (per exemple, una regla diferent per a cada tipus de servei), fent el sistema difícil de mantenir.
    
    \item El control de flux entre fases no era prou robust, i algunes vegades regles de diferents fases es barrejaven.
    
    \item Les explicacions encara eren incompletes i no sempre es generaven correctament.
\end{itemize}

\textbf{Decisions preses}:

Es va decidir:
\begin{enumerate}
    \item Implementar el pre-càlcul de distàncies mitjançant templates de proximitat.
    \item Generalitzar regles utilitzant la jerarquia de classes en lloc de regles específiques per a cada subclasse.
    \item Reforçar el control de fases mitjançant salience i fets de control.
    \item Revisar i millorar el sistema de generació d'explicacions.
\end{enumerate}

\subsubsection{Iteració 3: Refinament i optimització}

\textbf{Setmana 5 - Objectiu}: Optimitzar el sistema i completar totes les regles necessàries.

\textbf{Millores implementades}:
\begin{itemize}
    \item Implementació de templates auxiliars (proximitat, requisit-inferit, oferta-descartada, criteri-no-cumplit, punt-positiu, recomanacio) per estructurar millor el raonament.
    
    \item Més de 30 regles organitzades clarament per fases amb salience assignada adequadament:
    \begin{itemize}
        \item 6 regles d'abstracció per inferir requisits
        \item 15 regles de resolució (7 de descart, 3 de criteris negatius, 5 de punts positius)
        \item 3 regles de refinació per classificar
        \item Regles de presentació
    \end{itemize}
    
    \item Sistema robust de control de flux amb fets \texttt{fase-completada} i salience ben assignada.
    
    \item Explicacions detallades que inclouen:
    \begin{itemize}
        \item Llista de punts positius amb justificació
        \item Llista de criteris no complerts amb gravetat
        \item Motius específics de descart per a ofertes eliminades
    \end{itemize}
    
    \item 20+ instàncies de prova que cobreixen una àmplia varietat d'escenaris.
\end{itemize}

\textbf{Validació de la iteració}:

Aquesta iteració va produir un sistema plenament funcional. Les proves van mostrar que:
\begin{itemize}
    \item Totes les regles s'activaven correctament en l'ordre esperat.
    \item Les recomanacions eren coherents i ben justificades.
    \item L'eficiència era adequada (menys de 2 segons per processar 15 sol·licitants x 40 ofertes).
    \item Les explicacions eren clares i comprensibles.
\end{itemize}

No obstant això, les proves també van revelar alguns casos extrems que no es gestionaven correctament.

\subsubsection{Iteració 4: Proves exhaustives i correcció d'errors}

\textbf{Setmana 6 - Objectiu}: Provar el sistema exhaustivament i corregir tots els errors detectats.

\textbf{Activitats realitzades}:
\begin{itemize}
    \item Creació de més de 100 instàncies de prova que cobreixen:
    \begin{itemize}
        \item Tots els tipus d'habitatge
        \item Tots els perfils de sol·licitant
        \item Combinacions de restriccions complexes
        \item Cases extrems (pressupost mínim, màximes restriccions, etc.)
    \end{itemize}
    
    \item Desenvolupament de la interfície interactiva \texttt{main.clp} que permet crear perfils dinàmicament.
    
    \item Detecció i correcció de casos extrems problemàtics:
    \begin{itemize}
        \item Duplicació de requisits inferits (solucionat amb patró NOT)
        \item Conflictes de salience entre fases (ajustat el rang de salience)
        \item Tractament incorrecte de pressupost amb marge flexible (diferenciades regles per a casos estricte/flexible)
        \item No detecció d'ofertes sospitosament barates (afegit pressupostMinim i regla de descart)
    \end{itemize}
    
    \item Optimització de regles per millorar l'eficiència:
    \begin{itemize}
        \item Ordenació de condicions de més a menys restrictives
        \item Eliminació de redundàncies
        \item Millora del sistema de presentació de resultats
    \end{itemize}
    
    \item Documentació completa del codi amb comentaris explicatius.
\end{itemize}

\textbf{Resultat final}:

Al final d'aquesta iteració el sistema estava completament funcional, provat exhaustivament i documentat adequadament. Les proves van confirmar que:
\begin{itemize}
    \item El sistema genera recomanacions coherents en el 100\% dels casos provats.
    \item Les explicacions són sempre completes i comprensibles.
    \item No hi ha errors detectats en cap dels més de 100 casos de prova.
    \item L'eficiència és més que adequada per a l'ús previst.
\end{itemize}

\subsection{Modularització del codi}

Tot i que CLIPS no té un sistema de mòduls explícits com altres llenguatges de programació, s'ha aconseguit una bona modularitat mitjançant diverses tècniques.

\subsubsection{Estratègies de modularització}

La modularització s'ha aconseguit mitjançant:

\begin{enumerate}
    \item \textbf{Separació per fitxers}: Cada fitxer té una responsabilitat clarament definida. L'ontologia, les regles i les instàncies estan en fitxers separats.
    
    \item \textbf{Separació per fases}: Les regles s'organitzen en tres fases conceptuals (Abstracció, Resolució, Refinació) que s'executen seqüencialment.
    
    \item \textbf{Convenció de noms coherent}: Totes les regles segueixen el patró \texttt{fase-funcio-descripcio}. Per exemple:
    \begin{itemize}
        \item \texttt{abstraccio-familia-amb-fills}
        \item \texttt{resolucio-descartar-preu-excessiu}
        \item \texttt{refinacio-molt-recomanable}
    \end{itemize}
    Aquest patró fa que sigui immediat identificar a quina fase pertany cada regla i quina és la seva funció.
    
    \item \textbf{Templates especialitzats}: Cada tipus de raonament intermedi té el seu template específic, evitant barrejar conceptes diferents en la mateixa estructura de dades.
    
    \item \textbf{Documentació exhaustiva}: Cada regla inclou comentaris que expliquen què fa, per què és necessària i com s'integra amb la resta del sistema.
\end{enumerate}

\subsubsection{Organització de les fases}

Cadascuna de les tres fases està clarament delimitada i té un conjunt específic de regles:

\paragraph{Fase d'Abstracció}

\textbf{Responsabilitat}: Inferir requisits implícits i calcular proximitats.

\textbf{Regles incloses}:
\begin{itemize}
    \item \texttt{abstraccio-calcular-proximitats}: Calcula i classifica totes les distàncies entre habitatges i serveis.
    \item \texttt{abstraccio-familia-amb-fills}: Infereix necessitat d'escoles i zones verdes per a famílies.
    \item \texttt{abstraccio-persona-gran}: Infereix necessitat de serveis de salut i comerços propers.
    \item \texttt{abstraccio-estudiants}: Infereix necessitat de transport públic.
    \item \texttt{abstraccio-prefereix-transport}: Detecta preferència explícita per transport públic.
    \item \texttt{abstraccio-parella-futurs-fills}: Infereix preferència per zones educatives i verdes.
    \item \texttt{abstraccio-finalitzar}: Marca la fase com a completada.
\end{itemize}

\textbf{Entrada}: Instàncies de Sol·licitant, Habitatge i Servei.

\textbf{Sortida}: Fets \texttt{proximitat} i \texttt{requisit-inferit}.

\paragraph{Fase de Resolució}

\textbf{Responsabilitat}: Avaluar ofertes, descartar inadequades i detectar punts forts i dèbils.

Aquesta fase es subdivideix conceptualment en tres sub-mòduls:

\textbf{Sub-mòdul de Filtratge} (salience 40-45):
\begin{itemize}
    \item \texttt{resolucio-descartar-preu-excessiu}: Descarta ofertes que superen el pressupost màxim si aquest és estricte.
    \item \texttt{resolucio-descartar-preu-sospitos}: Descarta ofertes amb preus anormalment baixos.
    \item \texttt{resolucio-descartar-no-mascotes}: Descarta ofertes que no permeten mascotes quan el sol·licitant en té.
    \item \texttt{resolucio-descartar-no-accessible}: Descarta ofertes sense ascensor en plantes altes per a persones amb necessitats d'accessibilitat.
    \item \texttt{resolucio-descartar-servei-evitat}: Descarta ofertes molt properes a serveis que el sol·licitant vol evitar.
    \item \texttt{resolucio-descartar-falta-requisit-inferit}: Descarta ofertes que no tenen serveis obligatoris a prop.
    \item \texttt{resolucio-descartar-superficie-insuficient}: Descarta ofertes massa petites per al nombre de persones.
\end{itemize}

\textbf{Sub-mòdul de Criteris Negatius} (salience 35):
\begin{itemize}
    \item \texttt{resolucio-criteri-preu-alt}: Detecta preus lleugerament per sobre del pressupost.
    \item \texttt{resolucio-criteri-soroll-alt}: Detecta habitatges amb nivell de soroll alt.
    \item \texttt{resolucio-criteri-sense-ascensor}: Detecta absència d'ascensor quan és preferible.
\end{itemize}

\textbf{Sub-mòdul de Punts Positius} (salience 30-32):
\begin{itemize}
    \item \texttt{resolucio-punt-bon-preu}: Detecta preus significativament per sota del pressupost.
    \item \texttt{resolucio-punt-terrassa}: Valora la presència de terrassa o balcó.
    \item \texttt{resolucio-punt-assolellat}: Valora orientació solar favorable.
    \item \texttt{resolucio-punt-transport-aprop}: Valora proximitat a transport públic.
    \item \texttt{resolucio-punt-eficiencia-energetica}: Valora eficiència energètica alta.
    \item \texttt{resolucio-punt-parking}: Valora presència de parking quan el sol·licitant té vehicle.
    \item \texttt{resolucio-punt-vistes}: Valora bones vistes.
    \item \texttt{resolucio-punt-cobreix-requisit}: Valora compliment de requisits inferits.
    \item Diverses regles més per altres aspectes positius.
\end{itemize}

\paragraph{Fase de Refinació}

\textbf{Responsabilitat}: Classificar ofertes en categories finals.

\textbf{Regles incloses}:
\begin{itemize}
    \item \texttt{refinacio-molt-recomanable}: Classifica ofertes sense criteris negatius i amb 3 o més punts positius.
    \item \texttt{refinacio-adequat}: Classifica ofertes sense criteris negatius però amb menys de 3 punts positius.
    \item \texttt{refinacio-parcialment}: Classifica ofertes amb 1-2 criteris negatius lleus o moderats.
\end{itemize}

\paragraph{Mòdul de Presentació}

\textbf{Responsabilitat}: Mostrar resultats a l'usuari de manera clara i organitzada.

\textbf{Regles incloses} (salience negativa per executar-se al final):
\begin{itemize}
    \item \texttt{presentacio-inici}: Mostra capçalera per a cada sol·licitant.
    \item \texttt{presentacio-recomanacio}: Mostra detalls de cada recomanació.
    \item \texttt{presentacio-punts-positius}: Llista els punts forts de l'oferta.
    \item \texttt{presentacio-criteris-negatius}: Llista els aspectes negatius de l'oferta.
    \item \texttt{presentacio-descartades}: Mostra ofertes descartades amb els seus motius.
\end{itemize}

\subsection{Implementació de regles clau}

A continuació es detallen alguns exemples de regles representatives que il·lustren les tècniques d'implementació utilitzades.

\subsubsection{Càlcul de distàncies i proximitats}

El càlcul de distàncies és una operació fonamental que s'utilitza en moltes parts del raonament. Per eficiència, es fa una sola vegada al principi:

\begin{lstlisting}[caption={Funcions de càlcul de distància}]
(deffunction calcular-distancia (?x1 ?y1 ?x2 ?y2)
    "Calcula la distancia de Manhattan entre dos punts"
    (+ (abs (- ?x2 ?x1)) (abs (- ?y2 ?y1))))

(deffunction classificar-distancia (?metres)
    "Classifica una distancia en categories qualitatives"
    (if (< ?metres 500.0) then MoltAProp
    else (if (< ?metres 1000.0) then DistanciaMitjana
    else Lluny)))
\end{lstlisting}

La regla de pre-càlcul s'executa primer de tot (salience 100):

\begin{lstlisting}[caption={Regla de càlcul de proximitats}]
(defrule abstraccio-calcular-proximitats
    "Calcula totes les proximitats entre habitatges i serveis"
    (declare (salience 100))
    ?hab <- (object (is-a Habitatge) (teLocalitzacio ?locH))
    ?locHab <- (object (is-a Localitzacio) (name ?locH) 
                (coordenadaX ?x1) (coordenadaY ?y1))
    ?serv <- (object (is-a Servei) (teLocalitzacio ?locS))
    ?locServ <- (object (is-a Localitzacio) (name ?locS) 
                 (coordenadaX ?x2) (coordenadaY ?y2))
    (not (proximitat (habitatge ?hab) (servei ?serv)))
    =>
    (bind ?metres (calcular-distancia ?x1 ?y1 ?x2 ?y2))
    (bind ?dist (classificar-distancia ?metres))
    (bind ?cat (class ?serv))
    (assert (proximitat 
        (habitatge ?hab) 
        (servei ?serv) 
        (categoria ?cat) 
        (distancia ?dist) 
        (metres ?metres))))
\end{lstlisting}

\textbf{Aspectes destacables d'aquesta implementació}:

\begin{itemize}
    \item El patró NOT \texttt{(not (proximitat ...))} evita calcular la mateixa distància més d'una vegada.
    \item S'emmagatzema tant la categoria qualitativa (MoltAProp, DistanciaMitjana, Lluny) com el valor numèric exacte per si alguna regla el necessita.
    \item S'emmagatzema també la categoria del servei per facilitar consultes posteriors.
\end{itemize}

\subsubsection{Inferència de requisits}

Les regles d'inferència detecten automàticament necessitats basant-se en el perfil del sol·licitant:

\begin{lstlisting}[caption={Exemple de regla d'inferència}]
(defrule abstraccio-familia-amb-fills
    "Les families amb fills necessiten escoles i zones verdes"
    (declare (salience 95))
    ?sol <- (object (is-a Solicitant) (numeroFills ?fills))
    (test (> ?fills 0))
    (not (requisit-inferit (solicitant ?sol) 
          (categoria-servei ServeiEducatiu)))
    =>
    (assert (requisit-inferit 
        (solicitant ?sol) 
        (categoria-servei ServeiEducatiu)
        (obligatori si) 
        (motiu "Familia amb fills necessita escoles properes")))
    (assert (requisit-inferit 
        (solicitant ?sol) 
        (categoria-servei ZonaVerda)
        (obligatori no) 
        (motiu "Zona verda recomanable per a fills")))
    (printout t "[ABSTRACCIO] Inferits requisits per " 
              (instance-name ?sol) crlf))
\end{lstlisting}

\textbf{Aspectes destacables}:

\begin{itemize}
    \item El patró NOT evita inferir el mateix requisit múltiples vegades.
    \item Es distingeix entre requisits obligatoris (escoles) i preferibles (zones verdes).
    \item Cada requisit inclou un motiu textual que s'utilitzarà en les explicacions.
    \item S'imprimeix un missatge de traça per facilitar el debugging.
\end{itemize}

\subsubsection{Descart amb justificació}

Les regles de descart eliminen ofertes inadequades i registren el motiu:

\begin{lstlisting}[caption={Exemple de regla de descart}]
(defrule resolucio-descartar-no-mascotes
    "Descarta ofertes que no permeten mascotes"
    (declare (salience 40))
    (fase-completada (nom abstraccio))
    ?sol <- (object (is-a Solicitant) (teMascotes si))
    ?of <- (object (is-a Oferta) (teHabitatge ?hab))
    ?h <- (object (is-a Habitatge) (name ?hab) 
          (permetMascotes no))
    (not (oferta-descartada (solicitant ?sol) (oferta ?of)))
    =>
    (assert (oferta-descartada 
        (solicitant ?sol) 
        (oferta ?of)
        (motiu "No permet mascotes")))
    (printout t "[RESOLUCIO] DESCARTADA " (instance-name ?of) 
              " per " (instance-name ?sol) 
              " - No mascotes" crlf))
\end{lstlisting}

\textbf{Aspectes destacables}:

\begin{itemize}
    \item La condició \texttt{(fase-completada (nom abstraccio))} assegura que aquesta regla només s'executa després de la fase d'abstracció.
    \item El patró NOT evita marcar com a descartada una oferta que ja ho està.
    \item El motiu del descart queda registrat per poder mostrar-lo a l'usuari.
    \item El missatge de traça ajuda a seguir l'execució del sistema.
\end{itemize}

\subsubsection{Classificació multi-criteri}

Les regles de classificació utilitzen patrons complexos per comptar punts positius i negatius:

\begin{lstlisting}[caption={Regla de classificació}]
(defrule refinacio-molt-recomanable
    "Ofertes sense negatius i amb 3+ positius"
    (declare (salience 5))
    (fase-completada (nom resolucio))
    ?sol <- (object (is-a Solicitant))
    ?of <- (object (is-a Oferta))
    (not (oferta-descartada (solicitant ?sol) (oferta ?of)))
    (not (criteri-no-cumplit (solicitant ?sol) (oferta ?of)))
    (punt-positiu (solicitant ?sol) (oferta ?of) 
                  (descripcio ?d1))
    (punt-positiu (solicitant ?sol) (oferta ?of) 
                  (descripcio ?d2&~?d1))
    (punt-positiu (solicitant ?sol) (oferta ?of) 
                  (descripcio ?d3&~?d1&~?d2))
    (not (recomanacio (solicitant ?sol) (oferta ?of)))
    =>
    (assert (recomanacio 
        (solicitant ?sol) 
        (oferta ?of) 
        (grau MoltRecomanable) 
        (puntuacio 100))))
\end{lstlisting}

\textbf{Aspectes destacables}:

\begin{itemize}
    \item El patró \texttt{\&\textasciitilde} (\&~) assegura que els tres punts positius són diferents entre si.
    \item Les múltiples condicions NOT garanteixen que només ofertes adequades reben aquesta classificació.
    \item La condició final NOT evita classificar dues vegades la mateixa oferta.
\end{itemize}

\subsection{Control de flux amb salience}

El sistema utilitza salience (prioritat de regles) per controlar l'ordre d'execució i garantir que les fases s'executen seqüencialment. La següent taula mostra l'estratègia de salience utilitzada:

\begin{table}[H]
\centering
\small
\begin{tabular}{|l|c|p{8cm}|}
\hline
\textbf{Tipus de Regla} & \textbf{Salience} & \textbf{Justificació} \\
\hline
Càlcul de proximitats & 100 & S'ha d'executar abans que res per tenir totes les distàncies disponibles \\
Inferència de requisits & 95 & S'executa després de proximitats però abans d'avaluar ofertes \\
Fi fase abstracció & 50 & Marca el final de la fase quan no queden més inferències \\
Descart d'ofertes & 40-45 & Prioritari dins de resolució per eliminar ofertes clarament inadequades \\
Detecció de criteris negatius & 35 & Després de descarts però abans de positius \\
Detecció de punts positius & 30-32 & S'executa quan ja sabem que l'oferta no s'ha descartat \\
Fi fase resolució & 10 & Marca el final de la fase d'avaluació \\
Classificació final & 3-5 & Assigna categories finals després que tots els criteris s'hagin avaluat \\
Presentació de resultats & -10 a -100 & S'executa al final per mostrar els resultats \\
\hline
\end{tabular}
\caption{Estratègia de salience per controlar el flux d'execució}
\end{table}

Aquesta estratègia garanteix que:
\begin{enumerate}
    \item Les proximitats es calculen una sola vegada abans de res.
    \item Els requisits s'infereixen abans d'avaluar cap oferta.
    \item Els descarts es fan abans de calcular punts positius (no té sentit calcular punts positius d'ofertes que es descartaran).
    \item La classificació final només es fa quan tots els criteris s'han avaluat.
    \item La presentació de resultats és l'últim que passa.
\end{enumerate}