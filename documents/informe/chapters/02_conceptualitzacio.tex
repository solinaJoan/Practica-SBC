% 02_conceptualitzacio.tex
% Capítol 2: Conceptualització

\section{Conceptualització}
\label{sec:conceptualitzacio}

\vspace{0.5cm}

\subsection{Introducció a la fase de conceptualització}

La fase de conceptualització és el pont entre la identificació del problema i la seva formalització tècnica. En aquesta etapa, transformem el coneixement expert identificat en una estructura conceptual clara i coherent, definint els elements principals del domini i les seves interrelacions. L'objectiu és obtenir una visió compartida i no ambigua del problema que serveixi de base per a la posterior formalització en una ontologia i implementació en regles.

Aquesta fase és crítica perquè determina l'abast i la qualitat del sistema final. Una conceptualització inadequada (incompleta, ambigua o mal estructurada) propagarà problemes a totes les fases posteriors, mentre que una conceptualització sòlida facilitarà tant la formalització com la implementació i el manteniment.

Per dur a terme aquesta conceptualització, hem combinat diverses fonts de coneixement. Primer, hem analitzat plataformes immobiliàries existents per identificar les característiques que el mercat considera rellevants. Segon, hem consultat models de llenguatge actuals (ChatGPT i Claude) com a "experts virtuals" per explorar el domini de manera sistemàtica. Tercer, hem aplicat el nostre propi coneixement i sentit comú per validar i completar la informació obtinguda. Finalment, hem contrastat el resultat amb la literatura sobre sistemes de recomanació d'habitatge i amb agents immobiliaris reals quan ha estat possible.

\vspace{0.5cm}

\subsection{Conceptes principals del domini}

\subsubsection{Habitatge}

El concepte central del nostre domini és l'habitatge, que representa una vivenda disponible per llogar. Un habitatge no és simplement un conjunt de metres quadrats, sinó una entitat complexa amb múltiples facetes que determinen la seva adequació per a diferents perfils d'usuaris.

Hem identificat diversos grups d'atributs que caracteritzen un habitatge. Els atributs físics bàsics inclouen la superfície habitable (expressada en metres quadrats), el nombre total de dormitoris, la distinció entre dormitoris dobles i simples (rellevant per determinar la capacitat real), i el nombre de banys (que afecta la comoditat, especialment per a famílies nombroses).

La tipologia de l'habitatge és un atribut fonamental que afecta molts altres aspectes. Distingim entre pis (la opció més comuna), àtic (generalment preferit per la llum i les vistes), dúplex (amb dos nivells), estudi (opció compacta per a individus o parelles), i habitatge unifamiliar (ideal per a famílies grans que busquen independència). Cada tipologia té implicacions sobre altres característiques: per exemple, els àtics solen tenir terrassa gran, mentre que els estudis són compactes i econòmics.

L'equipament determina el nivell de comoditat i afecta especialment perfils com estudiants o persones que es traslladen temporalment. Considerem si l'habitatge està moblat (essencial per a estudiants i expatriats), si inclou electrodomèstics (nevera, rentadora, forn), si disposa de calefacció i aire condicionat (cada cop més valorats), i l'estat de conservació general (nou, bon estat, necessita reformes).

Les característiques arquitectòniques i de situació dins l'edifici també són rellevants. La planta on es troba l'habitatge (entresòl, primer, segon, àtic) i la presència o absència d'ascensor determinen l'accessibilitat. L'orientació solar (matí, tarda, tot el dia) afecta la lluminositat i l'eficiència energètica. Si l'habitatge és exterior o interior condiciona el soroll i la llum natural. La presència de terrassa o balcó i la seva superfície augmenten significativament l'atractiu. Les vistes (mar, muntanya, ciutat, o cap) són un plus valorat.

Les restriccions i condicions de lloguer inclouen si l'habitatge permet mascotes (eliminator per a moltes famílies), el consum energètic (de A a G, afectant les factures mensuals), i el nivell de soroll de la zona (baix, mitjà, alt).

Finalment, els extres o amenities poden ser determinants per a certs perfils: plaça d'aparcament (essencial si tens vehicle), piscina comunitària (atractiva per a famílies amb fills), traster (útil per a emmagatzematge addicional), i armaris encastats (alliberen espai habitable).

\subsubsection{Localització}

La localització és un concepte aparentment simple però fonamental, ja que connecta l'habitatge amb els serveis urbans i determina gran part de la seva adequació.

Una localització es caracteritza per diversos nivells de granularitat. Al nivell més general, tenim el districte i el barri, que proporcionen context urbà i socioeconòmic. A nivell més específic, tenim l'adreça completa (carrer i número) i el codi postal. Finalment, per a càlculs de distància, utilitzem coordenades geogràfiques.

En la nostra implementació simplificada, utilitzem un sistema de coordenades cartesià 2D (X, Y) que representa metres des d'un punt d'origen arbitrari a la ciutat. Això facilita els càlculs de distància sense la complexitat de treballar amb latitud/longitud. No obstant això, el disseny és prou flexible per adaptar-se fàcilment a coordenades GPS reals si es desitja integrar amb dades reals.

La localització és el punt de contacte entre l'habitatge i els serveis: cada habitatge té una localització, cada servei té una localització, i calculem la distància entre elles per determinar la proximitat.

\subsubsection{Servei}

Els serveis urbans són elements de la ciutat que afecten la qualitat de vida dels residents i, per tant, l'adequació d'un habitatge per a un perfil determinat. La seva proximitat a l'habitatge pot ser un factor obligatori, preferent, o fins i tot negatiu segons el tipus de servei i el perfil del sol·licitant.

Hem identificat sis categories principals de serveis, cadascuna amb múltiples subtipus.

Els serveis de transport públic són crucials en una ciutat gran. Inclouen estacions de metro (la forma més ràpida de transport urbà), parades de bus (donen més capilaritat), estacions de tren (per a connexions interurbanes), i en alguns casos, estacions de tramvia. La proximitat al transport públic és especialment important per a estudiants, persones grans, i qualsevol que no tingui vehicle propi.

Els serveis educatius són fonamentals per a famílies amb fills. Distingim entre llars d'infants (0-3 anys), escoles primàries (6-12 anys), instituts (12-18 anys), i universitats (18+ anys). No totes les famílies necessiten tots els tipus: una família amb un nadó prioritzarà llars d'infants, mentre que una amb adolescents buscarà instituts propers.

Els serveis de salut inclouen hospitals (per a urgències i especialitats), centres de salut o CAPs (per a atenció primària rutinària), i farmàcies (per a medicació i consultes menors). Són especialment importants per a persones grans i famílies amb fills petits.

Els serveis comercials són necessaris per a la vida quotidiana. Inclouen supermercats (compra setmanal o diària), hipermercats (compra mensual de gran volum), mercats municipals (productes frescos), i centres comercials (compres i oci). Les persones grans prefereixen supermercats propers on fer compra diària, mentre que les famílies poden preferir hipermercats per a compra mensual.

Les zones verdes són espais per a l'esbarjo i l'exercici. Inclouen parcs (grans extensions per a passejada i esport), jardins (espais més petits i tranquils), i zones esportives (amb instal·lacions específiques). Són particularment rellevants per a famílies amb fills (necessiten espais de joc) i persones amb mascotes.

Els serveis d'oci enriqueixen la vida social i cultural. Inclouen gimnasos, biblioteques, centres culturals, cinemes, teatres, restaurants i bars. Alguns d'aquests (bars, discoteques) poden generar soroll i ser considerats negatius per certs perfils.

Finalment, hem identificat una categoria especial de serveis potencialment molestos que alguns perfils voldran evitar: discoteques (soroll nocturn), estadis esportius (aglomeracions i soroll en dies de partit), zones industrials (contaminació i trànsit pesat), autopistes (soroll constant), i aeroports (soroll de vols). La proximitat a aquests serveis pot ser un factor d'exclusió per a certs sol·licitants.

\subsubsection{Sol·licitant}

El sol·licitant és la persona o grup de persones que busca habitatge. És el concepte més complex i divers, ja que diferents perfils tenen necessitats radicalment diferents.

Hem identificat set tipologies principals de sol·licitants, organitzades jeràrquicament.

L'individu representa una persona sola, que pot ser un professional jove, un divorçat, o qualsevol que visqui independentment. Generalment tenen necessitats senzilles (superfície moderada, preu ajustat) però poden ser exigents amb la localització segons si treballen o estudien.

La parella sense fills és una unitat de dues persones adultes. Solen prioritzar la qualitat de vida (terrassa, bona zona) sobre l'espai, ja que no necessiten múltiples dormitoris. Poden permetre's gastar més per persona que un individu sol.

La parella amb plans de fills és un cas especial: actualment són dues persones, però preveuen ampliar la família en un futur pròxim (1-2 anys). Busquen habitatges amb "potencial de créixer", preferint 3 dormitoris a 2, i valoren la proximitat a escoles i zones verdes tot i que encara no les necessiten.

La família amb fills és una de les tipologies més exigents. Distingim entre família biparental (dos adults) i monoparental (un adult), ja que aquest factor afecta el pressupost disponible. El nombre i les edats dels fills determinen necessitats específiques: famílies amb nadons necessiten llars d'infants, amb infants necessiten escoles primàries i zones de joc, amb adolescents necessiten instituts i centres culturals.

El grup d'estudiants representa dues o més persones joves que comparteixen habitatge mentre estudien. Tenen requisits molt específics: pressupost ajustat (repartit entre tots), habitatge moblat (per no haver d'invertir en mobles), proximitat a la universitat i transport públic, i tolerància al soroll (sovint viuen en zones d'ambient jove).

La persona gran (>65 anys) té necessitats molt específiques relacionades amb la mobilitat i la salut. Requereixen accessibilitat (ascensor si no és planta baixa, absència de barreres arquitectòniques), proximitat a serveis de salut (hospitals, CAPs, farmàcies), i comerços propers per a compra diària (sovint no tenen vehicle). També valoren la tranquil·litat i la seguretat del barri.

Per a tots els perfils, considerem atributs comuns que afecten les necessitats. El pressupost (màxim i mínim) és sempre un factor crític, així com si el marge és estricte o flexible. El nombre de persones determina l'espai mínim necessari. La possessió de mascotes (tipus i nombre) és un factor eliminator si l'habitatge no les permet. La possessió de vehicle fa valorar el parking inclòs. Les necessitats d'accessibilitat són crítiques per a persones amb mobilitat reduïda. La preferència per transport públic afecta la importància de la seva proximitat. El fet de treballar o estudiar a la ciutat condiciona la necessitat de bona connexió.

\subsubsection{Oferta}

Una oferta vincula un habitatge concret amb condicions específiques de lloguer i amb l'avaluació que el sistema en fa per a cada sol·licitant.

Els atributs bàsics d'una oferta inclouen l'habitatge al qual fa referència (amb totes les seves característiques), el preu mensual de lloguer, la disponibilitat (si està llogada o disponible), i la data de publicació (opcionalment, per identificar ofertes antigues).

Els atributs calculats pel sistema són els més interessants. El grau de recomanació (Molt Recomanable, Adequat, Parcialment Adequat, Descartat) és l'avaluació principal que el sistema fa de l'oferta per a un sol·licitant concret. Els motius de recomanació expliquen per què s'ha assignat aquest grau: quins són els punts forts (per a Molt Recomanables), quins criteris no es compleixen (per a Parcialment Adequats), o per què s'ha descartat (per a Descartats).

És important notar que el grau de recomanació i els motius són específics per a cada parella sol·licitant-oferta: el mateix habitatge pot ser Molt Recomanable per a una família amb fills i Descartat per a estudiants, o viceversa.

\vspace{0.5cm}

\subsection{Relacions entre conceptes}

Una vegada identificats els conceptes principals, cal definir com es relacionen entre ells. Aquestes relacions estructuren el coneixement del domini i determinaran la navegació i el raonament del sistema.

\subsubsection{Relacions d'associació bàsiques}

La relació més simple és teLocalitzacio, que connecta tant habitatges com serveis amb la seva localització geogràfica. És una relació funcional (cada habitatge/servei té exactament una localització) i obligatòria (no pot haver-hi habitatges o serveis sense localització).

La relació teHabitatge connecta cada oferta amb l'habitatge al qual fa referència. També és funcional i obligatòria: cada oferta referencia exactament un habitatge.

\subsubsection{Relacions de proximitat}

La relació aPropDe connecta localitzacions amb serveis i és fonamental per al raonament del sistema. No és binària (proper/lluny), sinó que distingim tres nivells de proximitat basats en la distància: moltAPropDe (menys de 500 metres, caminable en 5-7 minuts), aDistanciaMitjana (500-1000 metres, caminable en 10-15 minuts), i llunyde (més de 1000 metres, generalment requereix transport).

Aquesta classificació en tres nivells permet un raonament més ric que una simple dicotomia. Per exemple, una escola a distància mitjana pot ser acceptable per a una família, mentre que un hospital lluny pot ser problemàtic per a una persona gran.

La relació de proximitat es calcula automàticament a partir de les coordenades de les localitzacions, utilitzant la distància de Manhattan o Euclidiana. No es pre-calcula per a totes les parelles possibles (seria ineficient), sinó que es computa dinàmicament quan es necessita.

\subsubsection{Relacions de preferència i restricció}

Els sol·licitants poden expressar relacions amb serveis de tres tipus, que tenen interpretacions diferents en el raonament.

La relació requereixServei indica que el sol·licitant necessita obligatòriament aquest tipus de servei proper. Si un habitatge no té aquest servei a distància proper o mitjana, quedarà descartat. Per exemple, una família amb fills petits requereix escoles; un estudiant requereix transport públic.

La relació prefereixServei indica que el sol·licitant valora positivament aquest servei, però no és eliminator si no està disponible. La seva proximitat millorarà la valoració de l'oferta, però la seva absència no la descartarà. Per exemple, una parella sense fills pot preferir zones verdes, però no és obligatori.

La relació evitaServei indica que el sol·licitant no vol viure prop d'aquest tipus de servei (generalment serveis molestos). Si un habitatge està molt a prop d'un servei evitat, quedarà descartat o severament penalitzat. Per exemple, una persona gran pot voler evitar discoteques; una família amb nadós pot voler evitar estadis.

\subsubsection{Relacions espacials addicionals}

Opcionalment, els sol·licitants poden especificar llocTreball i llocEstudi, que són localitzacions específiques on passen temps regularment. Si s'especifiquen, el sistema pot valorar positivament habitatges ben connectats amb aquests llocs, ja sigui per proximitat directa o per proximitat a transport públic que els connecti.

\vspace{0.5cm}

\subsection{Ús de models de llenguatge per a la conceptualització}

Una de les innovacions metodològiques d'aquest projecte ha estat l'ús de models de llenguatge (ChatGPT 4 i Claude 3.5 Sonnet) com a "experts virtuals" per a la fase de conceptualització. Aquesta aproximació mereix una explicació detallada del procés, els resultats i les limitacions.

\subsubsection{Metodologia d'ús dels models de llenguatge}

Per utilitzar efectivament un model de llenguatge com a expert, cal primer posar-lo en context adequadament. Hem utilitzat el següent prompt de contextualització:

\begin{lstlisting}[language=bash, basicstyle=\ttfamily\footnotesize, breaklines=true]
You are an experienced real estate agent specializing in rental 
properties in Barcelona. You help match different types of people 
(families with children, students, couples, elderly) with suitable 
homes based on their needs and the neighborhood characteristics.

The knowledge engineer will ask you about the characteristics, 
criteria, and knowledge you use for making decisions. Your answers 
should be direct and precise, helpful for building rules in an 
expert system. Focus on providing concrete, actionable knowledge 
rather than general statements.
\end{lstlisting}

Aquest prompt estableix clarament el rol (agent immobiliari expert), el domini (lloguer d'habitatges a Barcelona), l'audiència (enginyer del coneixement), i el format de resposta desitjat (coneixement concret i accionable).

Un cop establert el context, hem fet preguntes progressivament més específiques per extreure coneixement estructurat. Aquest procés s'anomena "elicitació de coneixement" i segueix principis similars als utilitzats en entrevistes amb experts humans.

\subsubsection{Exemples de preguntes i respostes}

A continuació presentem alguns exemples representatius del diàleg amb el model, il·lustrant el tipus de coneixement extret.

\paragraph{Pregunta 1: Característiques del barri per a famílies}

\textit{Question: What are the most important neighborhood characteristics for a family with two young children (ages 3 and 6)?}

\textbf{Resposta sintetitzada del model:}
\begin{itemize}
    \item Primary schools within 500 metres (walking distance)
    \item Parks or playgrounds within 500 metres
    \item Pedestrian-friendly streets (low traffic, wide sidewalks)
    \item Healthcare centres (at least a CAP) within 1 km
    \item Supermarkets for daily shopping within 500 metres
    \item Low noise levels (avoid nightlife areas, stadiums, major roads)
    \item Public transport accessible but not mandatory if they have a car
    \item Daycare centres (llar d'infants) within 500 metres if both parents work
\end{itemize}

Aquesta resposta ens proporciona diversos inputs valuosos: distàncies concretes (500m per a serveis quotidians, 1km per a serveis menys freqüents), una jerarquia de necessitats (escola i zones verdes són prioritàries), i factors negatius a evitar (soroll, trànsit intens). També identifica factors condicionals (la llar d'infants només és rellevant si ambdós pares treballen).

\paragraph{Pregunta 2: Requisits d'espai segons composició familiar}

\textit{Question: What minimum size requirements would you recommend for different household sizes?}

\textbf{Resposta sintetitzada del model:}
\begin{itemize}
    \item Single person: 30-40 m² (studio acceptable, 1 bedroom preferred)
    \item Couple: 50-60 m² (1-2 bedrooms)
    \item Couple + 1 child: 65-80 m² (minimum 2 bedrooms)
    \item Couple + 2 children: 80-100 m² (minimum 3 bedrooms)
    \item Family of 5+: 100+ m² (minimum 4 bedrooms)
    \item Group of 3 students: 70-90 m² (3 bedrooms, shared spaces OK)
\end{itemize}

Aquestes xifres concretes són extremadament útils per definir regles de descart (un habitatge massa petit per al nombre de persones). També identifiquen diferències subtils: per exemple, estudiants toleren menys espai per persona que famílies perquè passen menys temps a casa.

\paragraph{Pregunta 3: Priorització amb pressupost ajustat}

\textit{Question: When a family has a strict budget and cannot afford everything they want, how do you advise them to prioritize between location, size, and amenities?}

\textbf{Resposta sintetitzada del model:}

The general priority order with tight budgets is:
\begin{enumerate}
    \item Basic space requirements (minimum bedrooms and total size)
    \item Essential services for their profile (schools for families, transport for students)
    \item Location/neighborhood safety
    \item Extras and amenities (these can wait)
\end{enumerate}

Budget flexibility of 10-15\% is reasonable if location and essential services are significantly better. Beyond 15\%, the financial strain outweighs the benefits.

Aquesta resposta ens dóna una jerarquia de prioritats que podem traduir en pesos relatius en la nostra funció d'avaluació. També proporciona un threshold concret (15\%) per a la flexibilitat de pressupost.

\paragraph{Pregunta 4: Criteris per a persones grans}

\textit{Question: What are the non-negotiable requirements for elderly people (65+) when renting?}

\textbf{Resposta sintetitzada del model:}
\begin{itemize}
    \item Elevator if above ground floor (absolutely critical)
    \item No architectural barriers (steps, narrow doors)
    \item Healthcare centres within walking distance (500-700m max)
    \item Daily shopping (small supermarket or market) within 500m
    \item Ground floor or low floor with elevator strongly preferred
    \item Quiet neighborhood (they're home more hours than working people)
    \item Good natural light (affects mood and health)
    \item Avoid: steep hills, stairs to building entrance, isolated areas
\end{itemize}

Aquesta resposta identifica requisits obligatoris (ascensor si no és planta baixa) versus preferències fortes (planta baixa), i també factors negatius específics per aquest col·lectiu (pendents pronunciats, escales a l'entrada).

\subsubsection{Validació i integració del coneixement obtingut}

El coneixement extret dels models de llenguatge no s'ha acceptat acríticament, sinó que s'ha validat i integrat seguint diverses estratègies.

Primer, hem contrastat les respostes entre diferents models (ChatGPT i Claude) per identificar consens i discrepàncies. El coneixement consistent entre models té més probabilitats de ser correcte.

Segon, hem comparat el coneixement extret amb fonts autoritzades: normatives d'accessibilitat per a persones grans, estudis sociològics sobre preferències habitacionals, i informació de plataformes immobiliàries reals.

Tercer, hem aplicat el sentit comú i l'experiència personal per detectar recomanacions poc realistes o específiques d'altres contextos geogràfics (per exemple, alguns models donaven consells més adients per a ciutats americanes que per a Barcelona).

Quart, hem simplificat i adaptat el coneixement al nivell de granularitat adequat per al nostre sistema. Per exemple, el model podria distingir entre 5 tipus de zones verdes, però nosaltres hem decidit treballar només amb 3 categories.

Finalment, hem documentat clarament en l'informe quan una decisió prové d'un model de llenguatge, permetent la traçabilitat i facilitant futures revisions si es detecten errors.

\subsubsection{Limitacions i consideracions crítiques}

L'ús de models de llenguatge per a conceptualització té limitacions que cal reconèixer explícitament.

Primer, els models poden "al·lucinar" (inventar) informació plausible però incorrecta. Per exemple, poden citar estudis inexistents o proporcionar estadístiques fabricades. Per això, tota informació factual o quantitativa s'ha de verificar amb fonts fiables.

Segon, els models poden reflectir biaixos presents en les seves dades d'entrenament. Per exemple, poden assumir estructures familiars tradicionals o fer suposicions culturalment específiques. Hem estat atents a aquestes possibles biaixos i els hem corregit quan els hem detectat.

Tercer, els models no tenen experiència real ni comprensió profunda del domini. Poden proporcionar coneixement "de llibre" que un expert real amb anys d'experiència matisaria o contradria. Per això, el coneixement extret s'ha complementat amb altres fonts.

Quart, els models són millors proporcionant coneixement general que coneixement específic d'un context local. Per exemple, poden donar bones recomanacions sobre necessitats de famílies amb fills (universal), però ser menys fiables sobre característiques específiques de barris de Barcelona.

Malgrat aquestes limitacions, l'ús de models de llenguatge ha estat valuós per a la fase de conceptualització, especialment per:
\begin{itemize}
    \item Accelerar l'exploració inicial del domini
    \item Identificar factors que podríem haver passat per alt
    \item Proporcionar estructuracions alternatives del coneixement
    \item Generar exemples concrets i casos d'ús
    \item Validar la coherència de les regles proposades
\end{itemize}

\vspace{0.5cm}

\subsection{Regles heurístiques identificades}

A partir de l'elicitació de coneixement (tant de models de llenguatge com d'altres fonts), hem identificat un conjunt de regles heurístiques que guiaran la implementació del sistema. Aquestes regles es poden agrupar en diverses categories.

\subsubsection{Regles basades en perfil demogràfic}

Aquestes regles infereixen necessitats a partir de la tipologia i característiques del sol·licitant.

\textbf{Regles per a famílies amb fills:}
\begin{itemize}
    \item SI numeroFills > 0 LLAVORS necessita escoles a distància propera o mitjana (obligatori)
    \item SI numeroFills > 0 LLAVORS prefereix zones verdes properes
    \item SI edatFills conté valors < 3 LLAVORS necessita llar d'infants propera
    \item SI numeroFills ≥ 2 LLAVORS evita zones de soroll nocturn
    \item SI numeroFills > 0 LLAVORS necessita mínim (1 + numeroFills) dormitoris
    \item SI numeroFills > 2 LLAVORS prefereix habitatge unifamiliar o dúplex
\end{itemize}

\textbf{Regles per a estudiants:}
\begin{itemize}
    \item SI és grup d'estudiants LLAVORS necessita transport públic molt proper (obligatori)
    \item SI és grup d'estudiants LLAVORS necessita habitatge moblat (obligatori)
    \item SI és grup d'estudiants LLAVORS prioritza pressupost per sobre d'altres factors
    \item SI és grup d'estudiants LLAVORS prefereix zones amb ambient jove
    \item SI és grup d'estudiants LLAVORS tolera nivell de soroll mitjà-alt
\end{itemize}

\textbf{Regles per a persones grans:}
\begin{itemize}
    \item SI edat > 65 I plantaPis > 0 LLAVORS necessita ascensor (obligatori)
    \item SI edat > 65 LLAVORS necessita serveis de salut propers (obligatori)
    \item SI edat > 65 LLAVORS necessita comerç d'alimentació proper (obligatori)
    \item SI edat > 65 LLAVORS evita zones de soroll alt
    \item SI edat > 65 LLAVORS prefereix planta baixa o primera
    \item SI edat > 65 LLAVORS valora molt la llum natural
\end{itemize}

\textbf{Regles per a parelles sense fills:}
\begin{itemize}
    \item SI és parella sense fills LLAVORS flexibilitat en localització (no hi ha restriccions escolars)
    \item SI és parella sense fills LLAVORS valora qualitat de vida (terrassa, vistes)
    \item SI és parella sense fills LLAVORS pot prioritzar preu sobre espai
\end{itemize}

\textbf{Regles per a parelles amb plans de fills:}
\begin{itemize}
    \item SI és parella futurs fills LLAVORS prefereix escoles properes (no obligatori, però valora positivament)
    \item SI és parella futurs fills LLAVORS prefereix mínim 3 dormitoris (espai per créixer)
    \item SI és parella futurs fills LLAVORS prefereix zones verdes properes
\end{itemize}

\subsubsection{Regles de sentit comú sobre habitatges}

Aquestes regles expressen coneixement general sobre la qualitat relativa dels habitatges, independent del perfil del sol·licitant.

\textbf{Regles sobre tipologia i planta:}
\begin{itemize}
    \item Àtic > Pis alt > Pis intermedi > Pis baix > Entresòl (en termes de llum i vistes)
    \item Habitatge amb ascensor > Sense ascensor (especialment en plantes altes)
    \item Habitatge exterior > Interior (llum natural i ventilació)
\end{itemize}

\textbf{Regles sobre orientació i llum:}
\begin{itemize}
    \item Orientació tot el dia > Orientació tarda > Orientació matí
    \item Habitatge amb terrassa/balcó > Sense terrassa
    \item Habitatge amb vistes > Sense vistes
\end{itemize}

\textbf{Regles sobre estat i equipament:}
\begin{itemize}
    \item Habitatge nou o reformat > Bon estat > Necessita reformes
    \item Amb calefacció i aire condicionat > Només un > Cap dels dos
    \item Consum energètic A o B > C o D > E, F o G
\end{itemize}

\textbf{Regles sobre extras:}
\begin{itemize}
    \item SI teVehicle LLAVORS valora molt positivament parking inclòs
    \item Piscina comunitària és un plus per a famílies amb fills
    \item Traster és útil per a famílies grans o emmagatzematge
\end{itemize}

\subsubsection{Regles de proximitat i distància}

Aquestes regles tradueixen distàncies físiques en qualificacions qualitatives.

\textbf{Classificació de distàncies:}
\begin{itemize}
    \item Molt a prop: < 500 m (caminable en 5-7 minuts)
    \item Distància mitjana: 500-1000 m (caminable en 10-15 minuts)
    \item Lluny: > 1000 m (generalment requereix transport)
\end{itemize}

\textbf{Adequació segons tipus de servei i perfil:}
\begin{itemize}
    \item Per a famílies amb fills petits: escola ha d'estar molt a prop o distància mitjana
    \item Per a persones grans: serveis bàsics (salut, comerç) han d'estar molt a prop
    \item Per a estudiants: transport públic ha d'estar molt a prop
    \item Serveis molestos: si estan molt a prop, descarten l'habitatge
\end{itemize}

\subsubsection{Regles de pressupost}

\textbf{Gestió de pressupost estricte:}
\begin{itemize}
    \item SI margeEstricte = si I preu > pressupostMaxim LLAVORS descarta
    \item SI margeEstricte = si I preu < pressupostMinim LLAVORS descarta (sospitós)
\end{itemize}

\textbf{Gestió de pressupost flexible:}
\begin{itemize}
    \item SI margeEstricte = no I preu ≤ pressupostMaxim * 1.15 LLAVORS acceptable amb advertència
    \item SI margeEstricte = no I preu > pressupostMaxim * 1.15 LLAVORS descarta
    \item SI preu < pressupostMaxim * 0.8 LLAVORS punt molt positiu (estalvi significatiu)
\end{itemize}

\vspace{0.5cm}

\subsection{Descomposició del problema en subproblemes}

Un cop identificats els conceptes, relacions i regles del domini, cal estructurar el procés de resolució del problema. Seguint la metodologia de sistemes experts, descomponem el problema principal en subproblemes més tractables que es resoldran seqüencialment.

\subsubsection{Problema principal}

\textbf{Entrada:} Conjunt de sol·licitants amb les seves característiques, conjunt d'ofertes d'habitatges amb les seves característiques, conjunt de serveis urbans amb les seves localitzacions.

\textbf{Sortida:} Per a cada sol·licitant, una llista d'ofertes classificades segons el seu grau d'adequació (Molt Recomanable, Adequat, Parcialment Adequat, Descartat), amb explicacions detallades per a cada classificació.

\textbf{Procés:} Avaluar cada parella (sol·licitant, oferta) considerant tots els factors rellevants i aplicant el coneixement expert codificat en forma de regles.

\subsubsection{Subproblema 1: Abstracció (Inferència de necessitats)}

\textbf{Entrada:} Perfil d'un sol·licitant amb les seves característiques bàsiques (edat, nombre de persones, fills, mascotes, vehicle, etc.)

\textbf{Tasca:} Inferir necessitats i preferències implícites a partir del perfil demogràfic, aplicant regles heurístiques del domini.

\textbf{Sortida:} Conjunt de requisits inferits, cadascun amb una categoria de servei, un nivell d'obligatorietat (obligatori vs preferent), i una justificació.

\textbf{Mètode:} Aplicació de regles de classificació heurística seguida de regles d'inferència específiques per cada tipologia identificada.

\textbf{Exemple:}
\begin{itemize}
    \item Entrada: Sol·licitant família biparental, 4 persones, 2 fills (edats 6 i 10 anys), té gos
    \item Procés: Regla detecta numeroFills > 0 → infereix necessitat d'escoles (obligatori)
    \item Procés: Regla detecta numeroFills > 0 → infereix preferència per zones verdes
    \item Procés: Regla detecta teMascotes = si → afegeix restricció permetMascotes obligatori
    \item Sortida: {(ServeiEducatiu, obligatori, "Família amb fills necessita escoles"), (ZonaVerda, preferent, "Família amb fills prefereix zones verdes"), (permetMascotes, obligatori, "Té mascota tipus Gos")}
\end{itemize}

\subsubsection{Subproblema 2: Resolució (Avaluació i filtratge)}

\textbf{Entrada:} Una parella (sol·licitant amb requisits inferits, oferta), localitzacions de tots els serveis.

\textbf{Tasca:} Avaluar l'oferta contra els requisits del sol·licitant, identificant incompatibilitats eliminadores, criteris no complerts, i punts positius destacables.

\textbf{Sortida:} O bé un motiu de descart (si hi ha incompatibilitat eliminadora), o bé dues llistes: criteris no complerts (amb gravetat) i punts positius.

\textbf{Mètode:} Aplicació seqüencial de tres tipus de regles:
\begin{enumerate}
    \item Regles de descart (eliminen ofertes clarament inadequades)
    \item Regles de detecció de criteris no complerts (identifiquen febleses no eliminadores)
    \item Regles de detecció de punts positius (identifiquen fortaleses)
\end{enumerate}

Aquest subproblema es subdivideix en tres fases:

\paragraph{Fase 2.1: Càlcul de proximitats}
\begin{itemize}
    \item Calcular distàncies entre l'habitatge de l'oferta i tots els serveis rellevants
    \item Classificar cada distància en: MoltAProp, DistanciaMitjana, Lluny
    \item Emmagatzemar aquestes proximitats per a ús posterior
\end{itemize}

\paragraph{Fase 2.2: Filtratge obligatori}
Aplicar regles de descart en aquest ordre de prioritat:
\begin{enumerate}
    \item Pressupost: Si el preu excedeix els límits establerts → descarta
    \item Mascotes: Si el sol·licitant té mascotes i l'habitatge no les permet → descarta
    \item Accessibilitat: Si el sol·licitant necessita accessibilitat i l'habitatge no la té → descarta
    \item Superfície: Si l'habitatge és massa petit per al nombre de persones → descarta
    \item Serveis obligatoris: Si falta algun servei requerit a distància adequada → descarta
    \item Serveis evitats: Si hi ha algun servei evitat molt a prop → descarta
    \item Mobilat: Si és grup d'estudiants i l'habitatge no està moblat → descarta
\end{enumerate}

\paragraph{Fase 2.3: Avaluació de criteris}
Per a ofertes no descartades, detectar:
\begin{itemize}
    \item Criteris no complerts: Preu lleugerament alt, soroll, sense ascensor (persona gran), servei preferent lluny, etc.
    \item Punts positius: Bon preu, terrassa, assolellat, alta eficiència energètica, transport molt proper, compleix requisits inferits, etc.
\end{itemize}

\textbf{Exemple:}
\begin{itemize}
    \item Entrada: Família Garcia (requisits inferits del Subproblema 1), Oferta Pis Eixample (95m², 3 dorm, 1350€, permite mascotes, escola a 300m, parc a 450m, metro a 250m)
    \item Fase 2.1: Calcula proximitats → escola MoltAProp, parc MoltAProp, metro MoltAProp
    \item Fase 2.2: Verifica descartos → Preu OK (1350 < 1500), Mascotes OK, Superfície OK, Serveis obligatoris OK → No descarta
    \item Fase 2.3: Detecta criteris → Cap negatiu
    \item Fase 2.3: Detecta positius → {Escola molt propera (compleix requisit inferit), Zona verda molt propera (compleix requisit inferit), Transport molt proper, Terrassa, Preu bo}
    \item Sortida: CriterisNoComplerts=[], PuntsPositius=[5 elements]
\end{itemize}

\subsubsection{Subproblema 3: Refinació (Classificació final)}

\textbf{Entrada:} Una parella (sol·licitant, oferta) amb el resultat de l'avaluació (criteris no complerts i punts positius).

\textbf{Tasca:} Assignar un grau de recomanació global basant-se en l'evidència acumulada.

\textbf{Sortida:} Grau de recomanació (Molt Recomanable, Adequat, Parcialment Adequat) i puntuació numèrica opcional per ordenació dins de cada categoria.

\textbf{Mètode:} Aplicació de regles de classificació basades en comptadors d'evidència positiva i negativa.

\textbf{Regles de classificació:}
\begin{itemize}
    \item SI (criteris no complerts = 0) I (punts positius ≥ 3) LLAVORS Molt Recomanable
    \item SI (criteris no complerts = 0) I (punts positius < 3) LLAVORS Adequat
    \item SI (criteris no complerts ≥ 1) I (criteris no complerts ≤ 2) LLAVORS Parcialment Adequat
    \item SI (criteris no complerts > 2) LLAVORS Descartat (reclassificat)
\end{itemize}

\textbf{Exemple:}
\begin{itemize}
    \item Entrada: Família Garcia + Oferta Eixample + Avaluació (0 criteris negatius, 5 punts positius)
    \item Procés: Aplica regla: 0 negatius I 5 positius (≥3) → Molt Recomanable
    \item Sortida: (MoltRecomanable, puntuació=100)
\end{itemize}

\vspace{0.5cm}

\subsection{Exemples de resolució manual}

Per validar la conceptualització i il·lustrar com funciona el raonament expert, presentem dos exemples de resolució manual completa seguint la descomposició en subproblemes.

\subsubsection{Exemple 1: Família Garcia}

\textbf{Perfil del sol·licitant:}
\begin{itemize}
    \item Nom: Família Garcia
    \item Tipologia: FamíliaBiparental
    \item Nombre de persones: 4 (2 adults + 2 fills edats 6 i 10 anys)
    \item Pressupost: màxim 1500 EUR/mes, mínim 600 EUR/mes, marge no estricte
    \item Mascotes: sí, 1 gos
    \item Vehicle: sí
\end{itemize}

\textbf{Ofertes disponibles:}
\begin{enumerate}
    \item Pis Eixample: 95m², 3 dorm, 2 banys, 1350 EUR, permet mascotes, escola 300m, parc 400m, metro 250m
    \item Àtic Gràcia: 120m², 3 dorm, 2 banys, 1800 EUR, NO permet mascotes, escola 600m
    \item Estudi Sants: 35m², 1 dorm, 650 EUR, permet mascotes
\end{enumerate}

\textbf{Resolució Oferta 1 (Pis Eixample):}

\textit{Subproblema 1 - Abstracció:}
\begin{itemize}
    \item Detecta: numeroFills = 2 > 0 → Infereix: {(ServeiEducatiu, obligatori, "Fills necessiten escola"), (ZonaVerda, preferent, "Fills prefereixen zones verdes")}
    \item Detecta: teMascotes = si → Afegeix: (permetMascotes, obligatori, "Té gos")
\end{itemize}

\textit{Subproblema 2 - Resolució:}
\begin{itemize}
    \item Fase 2.1: Calcula proximitats → escola MoltAProp (300m), parc MoltAProp (400m), metro MoltAProp (250m)
    \item Fase 2.2: Verifica descartos:
    \begin{itemize}
        \item Preu: 1350 < 1500 → OK
        \item Mascotes: permet = si → OK
        \item Superfície: 95m² ≥ 4*10 = 40m² → OK (àmpliament suficient)
        \item Serveis obligatoris: escola MoltAProp → OK
    \end{itemize}
    \item Fase 2.3: Detecta punts positius:
    \begin{itemize}
        \item Escola molt propera (compleix requisit obligatori inferit)
        \item Zona verda molt propera (compleix preferència inferida)
        \item Transport molt proper
        \item Té terrassa (assumit per la descripció)
        \item Preu bo (1350 / 1500 = 90\%, estalvi del 10\%)
    \end{itemize}
    \item No detecta criteris negatius
\end{itemize}

\textit{Subproblema 3 - Refinació:}
\begin{itemize}
    \item Avaluació: 0 criteris negatius, 5 punts positius
    \item Regla: 0 negatius I 5 ≥ 3 → Molt Recomanable
    \item Resultat: (MoltRecomanable, 100)
\end{itemize}

\textbf{Resolució Oferta 2 (Àtic Gràcia):}

\textit{Subproblema 2 - Resolució (usa els mateixos requisits inferits):}
\begin{itemize}
    \item Fase 2.2: Verifica descartos:
    \begin{itemize}
        \item Preu: 1800 > 1500, però marge no estricte. 1800/1500 = 1.2 (20\% excés). Si límit flexible és 15\%, 1500*1.15 = 1725 < 1800 → Supera límit flexible → DESCARTA
        \item [Alternativament, si arribés aquí] Mascotes: NO permet i sol·licitant té gos → DESCARTA
    \end{itemize}
    \item Motiu de descart: "Preu supera en un 20\% el pressupost màxim (límit flexible 15\%)" o "No permet mascotes i el sol·licitant té un gos"
\end{itemize}

\textit{No cal Subproblema 3 perquè ja està descartada.}

\textbf{Resolució Oferta 3 (Estudi Sants):}

\textit{Subproblema 2 - Resolució:}
\begin{itemize}
    \item Fase 2.2: Verifica descartos:
    \begin{itemize}
        \item Superfície: 35m² < 4*10 = 40m² → MASSA PETIT → DESCARTA
    \end{itemize}
    \item Motiu de descart: "Superfície insuficient per a 4 persones (35m² < 40m² mínim estimat)"
\end{itemize}

\textbf{Resultats finals per Família Garcia:}
\begin{itemize}
    \item Molt Recomanable: Pis Eixample (100 punts)
    \item Descartades: Àtic Gràcia (preu excessiu i no mascotes), Estudi Sants (massa petit)
\end{itemize}

\subsubsection{Exemple 2: Grup d'estudiants Marc i companys}

\textbf{Perfil del sol·licitant:}
\begin{itemize}
    \item Nom: Marc i companys
    \item Tipologia: GrupEstudiants
    \item Nombre de persones: 3
    \item Pressupost: màxim 900 EUR/mes, marge estricte
    \item Vehicle: no
    \item Prefereix transport públic: sí
    \item Estudia a la ciutat: sí
\end{itemize}

\textbf{Ofertes disponibles:}
\begin{enumerate}
    \item Pis compartit Gràcia: 95m², 4 dorm, 1400 EUR, moblat, metro 200m
    \item Estudi Gràcia: 30m², 1 dorm, 750 EUR, moblat, metro 150m
\end{enumerate}

\textbf{Resolució Oferta 1 (Pis compartit):}

\textit{Subproblema 1 - Abstracció:}
\begin{itemize}
    \item Detecta: GrupEstudiants → Infereix: {(Transport, obligatori, "Estudiants necessiten transport"), (moblat, obligatori, "Estudiants necessiten moblat")}
\end{itemize}

\textit{Subproblema 2 - Resolució:}
\begin{itemize}
    \item Fase 2.2: Verifica descartos:
    \begin{itemize}
        \item Preu: 1400 > 900 i marge estricte → DESCARTA
    \end{itemize}
    \item Motiu: "Preu supera pressupost màxim amb marge estricte"
\end{itemize}

\textbf{Resolució Oferta 2 (Estudi):}

\textit{Subproblema 2 - Resolució:}
\begin{itemize}
    \item Fase 2.1: metro MoltAProp (150m)
    \item Fase 2.2: Verifica descartos:
    \begin{itemize}
        \item Preu: 750 < 900 → OK
        \item Moblat: sí → OK
        \item Superfície: 30m² vs 3 persones. Mínim estimat: 3*10=30m² → JUST però OK
        \item Transport: metro MoltAProp → OK
    \end{itemize}
    \item Fase 2.3: Punts positius:
    \begin{itemize}
        \item Transport molt proper (compleix requisit)
        \item Moblat (compleix requisit)
        \item Preu bo (750/900 = 83\%, dins pressupost)
    \end{itemize}
    \item Possible criteri negatiu: Espai just per 3 persones (30m² és el mínim)
\end{itemize}

\textit{Subproblema 3 - Refinació:}
\begin{itemize}
    \item Avaluació: 0-1 criteris negatius (segons si considerem l'espai just com a negatiu), 3 punts positius
    \item Si 0 negatius: Molt Recomanable
    \item Si 1 negatiu (espai just): Parcialment Adequat
    \item Decisió: Considerem que 30m² per 3 estudiants és acceptable (passen poc temps a casa) → 0 negatius → Adequat (3 punts positius, però no excepcionals per Molt Recomanable)
\end{itemize}

\textbf{Resultats finals per Marc i companys:}
\begin{itemize}
    \item Adequat: Estudi Gràcia (70 punts)
    \item Descartades: Pis compartit (preu excessiu amb marge estricte)
\end{itemize}

\vspace{0.5cm}

\subsection{Organització de la resolució}

Hem establert que el procés de resolució segueix una estructura en pipeline de tres fases seqüencials: Abstracció, Resolució i Refinació. Aquesta organització no és arbitrària, sinó que respon a principis d'enginyeria del coneixement que faciliten el desenvolupament, manteniment i comprensió del sistema.

\subsubsection{Avantatges de l'organització en fases}

La separació en fases proporciona diversos beneficis tècnics i metodològics.

Primer, cada fase té una responsabilitat clarament delimitada, seguint el principi de separació de responsabilitats. L'Abstracció infereix necessitats, la Resolució avalua ofertes, i la Refinació classifica resultats. Aquesta separació fa que cada fase sigui més senzilla de desenvolupar i depurar.

Segon, les fases són seqüencials i unidireccionals, evitant dependències circulars complexes. L'Abstracció genera requisits que usa la Resolució, i la Resolució genera avaluacions que usa la Refinació, però no hi ha retroalimentació inversa. Això simplifica el flux de control i el raonament sobre el comportament del sistema.

Tercer, cada fase pot desenvolupar-se i provar-se de manera independent. Podem validar que l'Abstracció infereix correctament les necessitats sense haver implementat encara la Resolució. Podem provar la Resolució amb requisits manuals sense dependre de l'Abstracció. Aquest desenvolupament incremental redueix la complexitat.

Quart, la modularitat facilita el manteniment i l'extensió. Si volem afegir una nova regla d'inferència, només cal modificar l'Abstracció. Si volem canviar els criteris de classificació, només cal modificar la Refinació. Les altres fases no es veuen afectades.

\subsubsection{Flux de dades entre fases}

El flux de dades és clarament definit i traceable:

\textbf{Fase 1 (Abstracció):}
\begin{itemize}
    \item Entrada: Sol·licitant amb atributs bàsics
    \item Sortida: Requisits inferits (serveis obligatoris, serveis preferents, restriccions)
\end{itemize}

\textbf{Fase 2 (Resolució):}
\begin{itemize}
    \item Entrada: Sol·licitant + Requisits inferits + Oferta + Serveis
    \item Sortida: O descart amb motiu, o (CriterisNoComplerts, PuntsPositius)
\end{itemize}

\textbf{Fase 3 (Refinació):}
\begin{itemize}
    \item Entrada: (CriterisNoComplerts, PuntsPositius)
    \item Sortida: (GrauRecomanació, Puntuació)
\end{itemize}

Aquesta estructura en pipeline assegura que la informació flueix de manera predictible i que cada fase afegeix valor de manera incremental.

\vspace{0.5cm}

\subsection{Conclusions de la fase de conceptualització}

La fase de conceptualització ha permès transformar el problema inicial (recomanar habitatges) en una estructura conceptual clara i implementable. Hem identificat els conceptes principals del domini (Habitatge, Sol·licitant, Servei, Oferta, Localització), les seves característiques rellevants, i les relacions que els connecten.

Hem extret coneixement expert de diverses fonts, incloent models de llenguatge utilitzats com a experts virtuals, que hem validat i integrat de manera crítica. Aquest coneixement s'ha estructurat en forma de regles heurístiques organitzades per categories: regles basades en perfil demogràfic, regles de sentit comú, regles de proximitat, i regles de pressupost.

Hem descompost el problema principal en tres subproblemes seqüencials (Abstracció, Resolució, Refinació), cadascun amb entrada, sortida i mètode clarament definits. Aquesta descomposició facilita tant la implementació com el raonament sobre el comportament del sistema.

Hem validat la conceptualització mitjançant exemples de resolució manual, demostrant que el coneixement estructurat permet arribar a conclusions raonables i justificades per a casos concrets.

Amb aquesta base conceptual sòlida, estem preparats per avançar a la fase de formalització, on crearem una ontologia explícita que capturi tot aquest coneixement de manera precisa i computacionalment tractable.