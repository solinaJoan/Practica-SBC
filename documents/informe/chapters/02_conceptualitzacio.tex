% 02_conceptualitzacio.tex
% Capítol 2: Conceptualització

\section{Conceptes del Domini}

\subsection{Conceptes Principals}

\subsubsection{Habitatge}

Representa una vivenda disponible per llogar. Atributs principals:

\begin{itemize}
    \item \textbf{Físics}: Superfície habitable, nombre de dormitoris (simples/dobles), nombre de banys
    \item \textbf{Tipologia}: Pis, Àtic, Dúplex, Estudi, Habitatge Unifamiliar
    \item \textbf{Equipament}: Moblat, amb electrodomèstics, calefacció, aire condicionat
    \item \textbf{Característiques}: Terrassa/balcó, ascensor, vistes, orientació solar
    \item \textbf{Accessibilitat}: Planta, ascensor, adaptacions
    \item \textbf{Restriccions}: Permet mascotes, estat de conservació
    \item \textbf{Extras}: Parking, piscina comunitària, traster
\end{itemize}

\subsubsection{Localització}

Posició geogràfica d'un habitatge o servei:

\begin{itemize}
    \item Adreça completa
    \item Districte i barri
    \item Coordenades GPS (X, Y simplificades)
    \item Codi postal
\end{itemize}

\subsubsection{Servei}

Equipament o infraestructura urbana rellevant:

\begin{itemize}
    \item \textbf{Transport Públic}: Metro, bus, tren
    \item \textbf{Serveis Educatius}: Escoles, instituts, universitats, llars d'infants
    \item \textbf{Serveis de Salut}: Hospitals, centres de salut, farmàcies
    \item \textbf{Serveis Comercials}: Supermercats, hipermercats, centres comercials
    \item \textbf{Zones Verdes}: Parcs, jardins, zones esportives
    \item \textbf{Serveis d'Oci}: Gimnasos, biblioteques, centres culturals
    \item \textbf{Serveis Molestos}: Discoteques, estadis, zones industrials, aeroports
\end{itemize}

\subsubsection{Sol·licitant}

Persona o grup que busca habitatge. Tipus:

\begin{itemize}
    \item \textbf{Individu}: Persona sola
    \item \textbf{Parella sense fills}
    \item \textbf{Parella amb plans de fills}
    \item \textbf{Família biparental}: Amb fills
    \item \textbf{Família monoparental}
    \item \textbf{Grup d'estudiants}
    \item \textbf{Persona gran}: Major de 60 anys
\end{itemize}

Atributs:
\begin{itemize}
    \item Pressupost màxim/mínim
    \item Nombre de persones
    \item Fills i edats
    \item Mascotes (tipus i nombre)
    \item Vehicle propi
    \item Necessitats d'accessibilitat
    \item Preferències de transport
    \item Lloc de treball/estudi
\end{itemize}

\subsubsection{Oferta}

Vincula un habitatge amb condicions de lloguer:
\begin{itemize}
    \item Habitatge associat
    \item Preu mensual
    \item Disponibilitat
    \item Data de publicació
    \item Grau de recomanació (calculat pel sistema)
    \item Motius de recomanació
\end{itemize}

\subsection{Relacions entre Conceptes}

\begin{itemize}
    \item \texttt{teLocalitzacio}: Habitatge/Servei $\rightarrow$ Localització
    \item \texttt{teHabitatge}: Oferta $\rightarrow$ Habitatge
    \item \texttt{aPropDe}: Localització $\rightarrow$ Servei (i subpropietats: moltAPropDe, aDistanciaMitjana, llunyde)
    \item \texttt{evitaServei}: Sol·licitant $\rightarrow$ ServeiMolest
    \item \texttt{prefereixServei}: Sol·licitant $\rightarrow$ Servei
    \item \texttt{requereixServei}: Sol·licitant $\rightarrow$ Servei
\end{itemize}

\section{Descomposició del Problema}

\subsection{Problemes i Subproblemes}

\subsubsection{Problema Principal}
\textbf{Recomanar habitatges adequats a cada sol·licitant}

Es descompon en tres subproblemes:

\paragraph{Subproblema 1: Abstracció}
\begin{itemize}
    \item \textit{Entrada}: Perfil del sol·licitant (dades bàsiques)
    \item \textit{Tasca}: Inferir necessitats no explícites
    \item \textit{Sortida}: Requisits obligatoris i preferències inferides
    \item \textit{Mètode}: Classificació heurística + regles d'inferència
\end{itemize}

Exemple: Si sol·licitant és família amb fills $\Rightarrow$ necessita escoles a prop

\paragraph{Subproblema 2: Resolució}
\begin{itemize}
    \item \textit{Entrada}: Ofertes, sol·licitant, requisits inferits
    \item \textit{Tasca}: Avaluar cada oferta contra requisits
    \item \textit{Sortida}: Ofertes descartades + avaluació de les restants
    \item \textit{Mètode}: Filtratge per restriccions + scoring
\end{itemize}

Es subdivideix en:
\begin{enumerate}
    \item \textbf{Càlcul de proximitat}: Distància habitatge-serveis
    \item \textbf{Filtratge obligatori}: Descartar ofertes inadequades
    \item \textbf{Avaluació de criteris}: Detectar punts forts i febles
\end{enumerate}

\paragraph{Subproblema 3: Refinació}
\begin{itemize}
    \item \textit{Entrada}: Ofertes avaluades
    \item \textit{Tasca}: Classificar per grau d'adequació
    \item \textit{Sortida}: Llista ordenada amb justificacions
    \item \textit{Mètode}: Classificació per regles
\end{itemize}

Categories finals:
\begin{itemize}
    \item Molt Recomanable: 0 negatius, 3+ positius
    \item Adequat: 0 negatius, <3 positius
    \item Parcialment Adequat: 1-2 negatius
    \item Descartat: Incompleix requisit obligatori
\end{itemize}

\section{Coneixement Expert Extret}

\subsection{Ús de Models de Llenguatge com a Expert}

Per la fase de conceptualització, s'han utilitzat models de llenguatge (ChatGPT, Claude) com a experts del domini. Aquesta aproximació permet:

\begin{itemize}
    \item Accés ràpid a coneixement general sobre habitatge
    \item Exploració de casos variats
    \item Validació de regles heurístiques
\end{itemize}

\subsubsection{Prompt de Context}

\begin{lstlisting}[language=bash, basicstyle=\ttfamily\footnotesize]
You are an experienced real estate agent specializing in rental 
properties in Barcelona. You help match different types of people 
(families with children, students, couples, elderly) with suitable 
homes based on their needs and the neighborhood characteristics. 

The knowledge engineer will ask about the characteristics, criteria, 
and knowledge you use for making decisions. Your answers should be 
direct and precise, helpful for building rules in an expert system.
\end{lstlisting}

\subsubsection{Exemples de Preguntes i Respostes}

\paragraph{Pregunta 1:} \textit{What are the most important neighborhood characteristics for a family with two young children?}

\textbf{Resposta (sintetitzada):}
\begin{itemize}
    \item Schools within 500m (walking distance)
    \item Parks or green spaces nearby
    \item Low noise levels
    \item Pedestrian-friendly streets
    \item Healthcare centers accessible
    \item Avoid nightlife areas
\end{itemize}

\paragraph{Pregunta 2:} \textit{What minimum space requirements would you recommend for different family sizes?}

\textbf{Resposta:}
\begin{itemize}
    \item Single person: 30-40 m²
    \item Couple: 50-60 m²
    \item Couple + 1 child: 65-80 m²
    \item Couple + 2 children: 80-100 m²
    \item 4+ people: 100+ m²
\end{itemize}

\paragraph{Pregunta 3:} \textit{How do you prioritize budget constraints vs. other requirements?}

\textbf{Resposta:}
Budget is a hard constraint for strict budgets. With flexibility (10-15\% margin), location and essential services take priority over amenities like parking or pool.

\subsection{Regles Heurístiques Identificades}

\subsubsection{Regles per Perfil Demogràfic}

\begin{enumerate}
    \item \textbf{Famílies amb fills}:
    \begin{itemize}
        \item Necessiten escoles a distància mitjana o propera
        \item Prefereixen zones verdes
        \item Eviten zones de soroll nocturn
        \item Necessiten mínim 2 dormitoris + espai per fills
    \end{itemize}
    
    \item \textbf{Estudiants}:
    \begin{itemize}
        \item Transport públic imprescindible
        \item Pressupost ajustat (prioritat màxima)
        \item Prefereixen zones amb ambient jove
        \item Necessiten habitatge moblat
    \end{itemize}
    
    \item \textbf{Persones grans}:
    \begin{itemize}
        \item Accessibilitat crítica (ascensor si >planta 0)
        \item Serveis de salut propers
        \item Comerços a distància caminable
        \item Eviten zones amb escales o barreres
    \end{itemize}
    
    \item \textbf{Parelles sense fills}:
    \begin{itemize}
        \item Flexibilitat en localització
        \item Valoren qualitat de vida (terrassa, vistes)
        \item Poden prioritzar preu vs. espai
    \end{itemize}
\end{enumerate}

\subsubsection{Regles de Sentit Comú}

\begin{itemize}
    \item Àtic millor que entresol (llum, vistes)
    \item Orientació tot el dia millor que només matí/tarda
    \item Exterior millor que interior
    \item Edifici nou millor que antic (a igualtat de preu)
    \item Terrassa és un plus sempre
    \item Parking valuós si tens vehicle
\end{itemize}

\subsubsection{Criteris de Distància}

\begin{itemize}
    \item \textbf{Molt a prop}: < 500m (caminable 5-7 min)
    \item \textbf{Distància mitjana}: 500-1000m (caminable 10-15 min)
    \item \textbf{Lluny}: > 1000m (necessita transport)
\end{itemize}

\section{Exemples de Resolució Manual}

\subsection{Exemple 1: Família Garcia}

\textbf{Perfil}:
\begin{itemize}
    \item Família biparental, 4 persones (2 adults + 2 fills de 6 i 10 anys)
    \item Pressupost: 1500 EUR/mes (màx), 600 EUR (mín), marge NO estricte
    \item Té gos, té vehicle propi
\end{itemize}

\textbf{Ofertes disponibles}:
\begin{enumerate}
    \item Pis 95m², 3 dorm, 1350 EUR, Eixample (escola 200m, parc 300m, permet mascotes)
    \item Àtic 120m², 3 dorm, 1800 EUR, Gràcia (escola 600m, sense parking, NO mascotes)
    \item Estudi 35m², 650 EUR, Sants (massa petit)
\end{enumerate}

\textbf{Raonament expert}:
\begin{itemize}
    \item Oferta 1: \textbf{MOLT RECOMANABLE}
    \begin{itemize}
        \item Dins pressupost
        \item Espai adequat per 4 persones
        \item Escola molt a prop (ideal per fills)
        \item Parc proper (ideal per fills i gos)
        \item Permet mascotes
    \end{itemize}
    
    \item Oferta 2: \textbf{DESCARTADA}
    \begin{itemize}
        \item Supera pressupost >15\%
        \item No permet mascotes (elimina automàticament)
    \end{itemize}
    
    \item Oferta 3: \textbf{DESCARTADA}
    \begin{itemize}
        \item Superfície insuficient (35m² per 4 persones)
    \end{itemize}
\end{itemize}

\subsection{Exemple 2: Estudiants Marc i companys}

\textbf{Perfil}:
\begin{itemize}
    \item Grup de 3 estudiants
    \item Pressupost: 900 EUR/mes màx, 300 mín, ESTRICTE
    \item No tenen vehicle, prefereixen transport públic
    \item Estudien a la ciutat
\end{itemize}

\textbf{Ofertes}:
\begin{enumerate}
    \item Pis compartit 95m², 4 dorm, 1400 EUR, Gràcia (metro 200m, moblat)
    \item Estudi 30m², 750 EUR, Gràcia (metro 150m, moblat)
\end{enumerate}

\textbf{Raonament expert}:
\begin{itemize}
    \item Oferta 1: \textbf{DESCARTADA} (supera pressupost estricte)
    \item Oferta 2: \textbf{ADEQUAT}
    \begin{itemize}
        \item Dins pressupost
        \item Metro molt a prop
        \item Moblat (necessari per estudiants)
    \end{itemize}
\end{itemize}

\section{Organització de la Resolució}

El procés de resolució segueix un \textit{pipeline} de tres fases:

\begin{enumerate}
    \item \textbf{ABSTRACCIÓ} (Inferència)
    \begin{itemize}
        \item Analitza perfil del sol·licitant
        \item Genera requisits implícits
        \item Calcula proximitats
    \end{itemize}
    
    \item \textbf{RESOLUCIÓ} (Avaluació)
    \begin{itemize}
        \item Descarta ofertes inadequades
        \item Avalua punts positius i negatius
        \item Acumula evidència
    \end{itemize}
    
    \item \textbf{REFINACIÓ} (Classificació)
    \begin{itemize}
        \item Assigna grau de recomanació
        \item Genera explicacions
        \item Ordena resultats
    \end{itemize}
\end{enumerate}

Aquesta organització permet separació de responsabilitats i facilita manteniment.