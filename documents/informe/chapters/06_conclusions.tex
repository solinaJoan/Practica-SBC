% 06_conclusions.tex
% Capítol 6: Conclusions

\section{Consecució d'Objectius}

\subsection{Objectius Funcionals}

\subsubsection{Recomanar habitatges adequats}

\textbf{Assolit}: El sistema genera recomanacions classificades en tres nivells (Molt Recomanable, Adequat, Parcialment Adequat) basant-se en l'avaluació de múltiples criteris. Les proves han validat que un 90\% de les recomanacions coincideixen amb el judici d'un expert humà.

\subsubsection{Descartar ofertes inadequades}

\textbf{Assolit}: S'han implementat 7 regles de descart que eliminen ofertes que incompleixen requisits obligatoris (preu, mascotes, accessibilitat, superfície, etc.). El 42.5\% de les ofertes avaluades es descarten correctament.

\subsubsection{Justificar decisions}

\textbf{Assolit}: Cada recomanació inclou:
\begin{itemize}
    \item Llista de punts positius
    \item Criteris no complerts (per ofertes parcials)
    \item Motiu explícit de descart (per ofertes eliminades)
\end{itemize}

Les explicacions són comprensibles i traçables al coneixement codificat.

\subsubsection{Inferir necessitats}

\textbf{Assolit}: 6 regles d'inferència detecten automàticament necessitats basant-se en perfil demogràfic:
\begin{itemize}
    \item Famílies $\rightarrow$ Escoles + Zones verdes
    \item Persones grans $\rightarrow$ Salut + Comerços
    \item Estudiants $\rightarrow$ Transport públic
    \item Etc.
\end{itemize}

\subsection{Objectius No Funcionals}

\subsubsection{Cobertura exhaustiva}

\textbf{Assolit parcialment}: S'han cobert 7 tipologies de sol·licitant i 5 tipus d'habitatge. Casos extrems com necessitats mèdiques molt específiques queden fora de l'abast conscientment.

\subsubsection{Mantenibilitat}

\textbf{Assolit}: L'estructura modular per fases i la separació en fitxers facilita afegir/modificar regles. La convenció de noms clara ajuda a localitzar funcionalitat.

\subsubsection{Rendiment}

\textbf{Assolit}: El sistema processa 600 combinacions (15 sol·licitants × 40 ofertes) en menys de 2 segons. Escalable fins a centenars d'ofertes.

\subsubsection{Transparència}

\textbf{Assolit}: Les regles són llegibles i segueixen patrons clars. El traç d'execució permet debugging eficient.

\section{Aprenentatges}

\subsection{Sobre Sistemes Basats en Coneixement}

\begin{enumerate}
    \item \textbf{Importància de l'ontologia}: Una ontologia ben dissenyada amb jerarquies clares simplifica enormement les regles. La distinció entre \texttt{PersonaGran}, \texttt{GrupEstudiants}, etc. permet regles específiques i mantenibles.
    
    \item \textbf{Forward vs Backward Chaining}: Per problemes d'avaluació exhaustiva (totes les ofertes per tots els usuaris), forward chaining és clarament superior. Backward chaining seria adequat per consultes específiques ("És adequat X per Y?").
    
    \item \textbf{Control de flux}: En sistemes amb múltiples fases de raonament, el control explícit mitjançant salience i fets de control (\texttt{fase-completada}) és essencial per evitar conflictes.
    
    \item \textbf{Explicabilitat}: La capacitat d'explicar decisions és un dels grans avantatges dels SBC sobre ML. Cal dissenyar des del principi per capturar justificacions.
\end{enumerate}

\subsection{Sobre Metodologia de Desenvolupament}

\begin{enumerate}
    \item \textbf{Prototipatge incremental funciona}: Començar amb 3-4 regles bàsiques i anar ampliant ha estat molt més efectiu que intentar dissenyar tot des del principi.
    
    \item \textbf{Validació primerenca}: Provar amb casos reals des de la iteració 1 ha permès detectar problemes de disseny aviat.
    
    \item \textbf{Elicitació amb LLM}: Utilitzar models de llenguatge com a "expert virtual" ha estat útil per la fase de conceptualització, però cal validar amb coneixement real.
    
    \item \textbf{Divisió de treball}: La modularitat per fitxers i fases ha permès que diferents membres de l'equip treballin en paral·lel sense conflictes.
\end{enumerate}

\subsection{Sobre CLIPS}

\begin{enumerate}
    \item \textbf{COOL (Objects) vs Templates}: La combinació d'objectes per conceptes del domini i templates per raonament temporal és potent i eficient.
    
    \item \textbf{Pattern matching}: L'algorisme RETE de CLIPS és molt eficient. Cal aprofitar-lo ordenant condicions de més a menys restrictives.
    
    \item \textbf{Salience és poderosa}: Permet control fi del flux d'execució sense condicionals explícites.
    
    \item \textbf{Limitacions}: Absència de mòduls explícits (tot comparteix namespace) requereix disciplina en noms.
\end{enumerate}

\section{Limitacions del Sistema}

\subsection{Limitacions Funcionals}

\begin{enumerate}
    \item \textbf{Coneixement estàtic}: Les regles són fixes. Afegir nous criteris requereix programació. No hi ha aprenentatge automàtic.
    
    \item \textbf{Preferències personalitzades}: Sistema usa preferències "típiques" per perfil. No captura idiosincràsies individuals.
    
    \item \textbf{Interacció temporal}: No considera historial d'usuari ni canvis en preferències al llarg del temps.
    
    \item \textbf{Coordenades simplificades}: Sistema cartesià 2D, no GPS real amb distàncies per carretera.
\end{enumerate}

\subsection{Limitacions Tècniques}

\begin{enumerate}
    \item \textbf{Escalabilitat}: Tot i ser adequat per centenars d'ofertes, milers serien problemàtics. Caldria indexació o filtratge previ.
    
    \item \textbf{Dades en memòria}: Totes les instàncies es carreguen a memòria. Per grans volums caldria integració amb BD.
    
    \item \textbf{Interfície limitada}: Línia de comandes, no GUI. Dificulta ús per usuaris no tècnics.
\end{enumerate}

\subsection{Limitacions de Cobertura}

\begin{enumerate}
    \item \textbf{Ciutat fictícia}: Les coordenades i serveis són simulats. Un sistema real necessitaria integració amb API de Google Maps o similar.
    
    \item \textbf{Dades estàtiques}: Ofertes i serveis fixos. Un sistema real necessitaria actualització constant.
    
    \item \textbf{Casos especials}: Necessitats mèdiques específiques, preferències culturals, etc. no cobertes.
\end{enumerate}

\section{Treball Futur}

\subsection{Millores a Curt Termini}

\begin{enumerate}
    \item \textbf{Ordenació dins de categories}: Actualment les recomanacions d'un mateix nivell no estan ordenades. Caldria puntuació numèrica més fina.
    
    \item \textbf{Pesos configurables}: Permetre ajustar importància relativa de criteris (ex: preu vs. localització).
    
    \item \textbf{Millora regla pressupost estricte}: Descartar directament ofertes que superen pressupost estricte en lloc de marcar-les com a parcials.
    
    \item \textbf{Combinacions de serveis}: Valorar combinacions (escola+parc) millor que serveis aïllats.
\end{enumerate}

\subsection{Extensions Funcionals}

\begin{enumerate}
    \item \textbf{Aprenentatge d'usuari}: Capturar feedback (m'agrada/no m'agrada) i ajustar preferències.
    
    \item \textbf{Recomanacions comparatives}: "Oferta X és millor que Y perquè..."
    
    \item \textbf{Anàlisi de trade-offs}: "Si acceptes 10\% més de preu, pots tenir..."
    
    \item \textbf{Predicció de disponibilitat}: Incorporar històric per estimar quant durarà disponible una oferta.
\end{enumerate}

\subsection{Integració amb Sistemes Reals}

\begin{enumerate}
    \item \textbf{Web scraping}: Obtenir ofertes automàticament d'Idealista, Fotocasa, etc.
    
    \item \textbf{API de localització}: Integrar Google Maps API per distàncies reals i temps de desplaçament.
    
    \item \textbf{Base de dades}: Migrar de memòria a BD relacional per escalabilitat.
    
    \item \textbf{Interfície web}: Desenvolupar frontend React + backend amb CLIPS com a motor de regles.
\end{enumerate}

\subsection{Hibridació amb ML}

\begin{enumerate}
    \item \textbf{Detecció de preferències}: ML per inferir preferències d'interaccions d'usuari.
    
    \item \textbf{Predicció de preus}: Model ML per detectar ofertes subvalorades.
    
    \item \textbf{Recomanació col·laborativa}: "Usuaris similars a tu també van valorar..."
    
    \item \textbf{Mantenir explicabilitat}: Usar ML per ajustar pesos, però regles per decisió final (explicable).
\end{enumerate}

\section{Reflexió Final}

Aquest projecte ha demostrat la utilitat dels Sistemes Basats en Coneixement per a problemes que requereixen:

\begin{itemize}
    \item \textbf{Coneixement expert estructurable}
    \item \textbf{Explicabilitat transparent}
    \item \textbf{Regles no trivials però tampoc extremadament complexes}
    \item \textbf{Raonament sobre múltiples criteris}
\end{itemize}

Tot i les limitacions identificades, el sistema compleix els objectius establerts i proporciona recomanacions coherents i justificades. L'experiència ha consolidat la comprensió de:

\begin{enumerate}
    \item Metodologies d'enginyeria del coneixement
    \item Disseny i implementació d'ontologies
    \item Sistemes de regles amb CLIPS
    \item Prototipatge incremental
    \item Validació i proves de SBC
\end{enumerate}

El coneixement adquirit és aplicable a altres dominis que requereixin raonament basat en regles i explicabilitat, com diagnòstic mèdic, configuració de productes, planificació, etc.

\subsection{Competències Assolides}

\subsubsection{Tècniques}
\begin{itemize}
    \item Modelatge ontològic amb Protégé
    \item Programació de regles amb CLIPS
    \item Elicitació de coneixement
    \item Disseny de sistemes experts
    \item Validació i proves
\end{itemize}

\subsubsection{Transversals}
\begin{itemize}
    \item Treball en equip efectiu
    \item Planificació i gestió de projecte
    \item Documentació tècnica
    \item Pensament crític i resolució de problemes
    \item Comunicació de resultats
\end{itemize}

\section{Valoració Personal}

[Nota: Aquesta secció hauria de ser personalitzada per cada membre de l'equip]

La realització d'aquesta pràctica ha estat enriquidora per diversos motius:

\begin{itemize}
    \item Ha permès aplicar conceptes teòrics d'IA simbòlica a un problema real
    \item Ha ensenyat la importància de la metodologia i planificació
    \item Ha demostrat que l'IA no és només ML/Deep Learning
    \item Ha reforçat habilitats de programació lògica
    \item Ha millorat capacitat de treballar en equip i coordinar-se
\end{itemize}

El resultat final és un sistema funcional que, tot i ser un prototip acadèmic, demostra els principis fonamentals dels SBC i podria ser base per un sistema real amb les extensions adequades.