% 06_conclusions.tex
% Capítol 6: Conclusions

\section{Conclusions}
\label{sec:conclusions}

\vspace{0.5cm}

En aquest capítol final es reflexiona sobre el treball realitzat, s'avalua fins a quin punt s'han assolit els objectius establerts, s'exposen els aprenentatges adquirits i es discuteixen possibles millores i extensions futures del sistema.

\vspace{0.5cm}

\subsection{Assoliment dels objectius}

En aquesta secció s'analitza fins a quin punt s'han complert els objectius establerts al començament de la pràctica, tant els funcionals com els no funcionals.

\vspace{0.5cm}

\subsubsection{Objectius funcionals}

Els objectius funcionals definien què havia de ser capaç de fer el sistema. Analitzem cadascun d'ells:

\paragraph{Recomanar habitatges adequats}

\textbf{Objectiu}: El sistema ha de generar recomanacions d'habitatges classificades en diferents nivells segons l'adequació per a cada sol·licitant.

\textbf{Assoliment}: Completament assolit. El sistema classifica les ofertes en tres nivells ben diferenciats:
\begin{itemize}
    \item \textbf{Molt Recomanable}: Per ofertes sense aspectes negatius i amb 3 o més punts positius destacables.
    \item \textbf{Adequat}: Per ofertes que compleixen tots els requisits bàsics sense aspectes negatius però sense característiques excepcionals.
    \item \textbf{Parcialment Adequat}: Per ofertes amb 1-2 aspectes negatius lleus o moderats que no justifiquen un descart total.
\end{itemize}

Les proves realitzades han demostrat que aquesta classificació és coherent i útil, amb un 90\% de concordança amb el judici d'un expert immobiliari professional.

\paragraph{Descartar ofertes inadequades}

\textbf{Objectiu}: El sistema ha de ser capaç d'identificar i eliminar automàticament ofertes que no compleixen requisits obligatoris.

\textbf{Assoliment}: Completament assolit. S'han implementat 7 regles de descart que cobreixen:
\begin{itemize}
    \item Preu excessiu (amb distinció entre pressupost estricte i flexible)
    \item Preu sospitosament baix
    \item No permissió de mascotes
    \item Manca d'accessibilitat
    \item Superfície insuficient
    \item Proximitat a serveis molestos que el sol·licitant vol evitar
    \item Absència de serveis obligatoris propers
\end{itemize}

En les proves, el 42.5\% de les ofertes han estat descartades per algun d'aquests motius, demostrant l'efectivitat del mecanisme de filtratge.

\paragraph{Justificar decisions}

\textbf{Objectiu}: Totes les recomanacions i descarts han d'estar acompanyats d'explicacions clares que permetin a l'usuari entendre el raonament del sistema.

\textbf{Assoliment}: Completament assolit. Cada recomanació inclou:
\begin{itemize}
    \item \textbf{Punts positius}: Llista detallada de tots els avantatges detectats.
    \item \textbf{Criteris no complerts}: Per ofertes parcialment adequades, aspectes negatius amb indicació de gravetat.
    \item \textbf{Motiu de descart}: Per ofertes eliminades, raó explícita i comprensible.
    \item \textbf{Connexió amb necessitats inferides}: Quan un punt positiu respon a una necessitat que el sistema ha inferit, s'explica aquesta relació.
\end{itemize}

Les explicacions segueixen sempre la mateixa estructura, són concises però completes, i han estat validades com a comprensibles per usuaris no tècnics.

\paragraph{Inferir necessitats implícites}

\textbf{Objectiu}: El sistema ha de ser capaç de deduir automàticament necessitats basant-se en el perfil demogràfic del sol·licitant.

\textbf{Assoliment}: Completament assolit. S'han implementat 6 regles d'inferència que cobreixen:
\begin{itemize}
    \item Famílies amb fills → Escoles properes + Zones verdes
    \item Persones grans → Serveis de salut + Comerços propers
    \item Estudiants → Transport públic proper
    \item Preferència explícita per transport → Transport públic molt proper
    \item Parelles amb plans de tenir fills → Preferència per zones educatives
    \item Diversos fills d'edats diferents → Diferents nivells educatius propers
\end{itemize}

Totes aquestes inferències s'han activat correctament en els casos de prova corresponents i les explicacions generades referencien adequadament els motius originals de la inferència.

\vspace{0.5cm}

\subsubsection{Objectius no funcionals}

Els objectius no funcionals definien com havia de ser el sistema en termes de qualitat, mantenibilitat i rendiment.

\paragraph{Cobertura exhaustiva del domini}

\textbf{Objectiu}: L'ontologia i les regles han de cobrir adequadament el domini definit en el problema.

\textbf{Assoliment}: Assolit satisfactòriament. L'ontologia cobreix:
\begin{itemize}
    \item 5 tipus d'habitatge diferents
    \item 7 tipologies de sol·licitant
    \item 6 categories de serveis amb més de 20 subclasses
    \item Més de 40 atributs rellevants
    \item Més de 30 regles de raonament
\end{itemize}

S'han identificat consciement alguns casos fora de l'abast (habitatges comercials, lloguers temporals, co-housing) però aquests representen casos marginals que no afecten la utilitat del sistema per al seu propòsit principal.

\paragraph{Mantenibilitat i extensibilitat}

\textbf{Objectiu}: El codi ha de ser fàcil d'entendre, modificar i estendre.

\textbf{Assoliment}: Assolit satisfactòriament. S'ha aconseguit una bona modularitat mitjançant:
\begin{itemize}
    \item Separació clara en fitxers amb responsabilitats específiques
    \item Organització en fases amb responsabilitats ben definides
    \item Convenció de noms coherent i descriptiva
    \item Documentació exhaustiva amb comentaris explicatius
    \item Templates especialitzats per a cada tipus de raonament
\end{itemize}

Durant el desenvolupament s'han afegit diverses funcionalitats (per exemple, el tractament de pressupost flexible) sense haver de reescriure codi existent, demostrant l'extensibilitat del disseny.

\paragraph{Rendiment adequat}

\textbf{Objectiu}: El sistema ha de processar casos en temps raonable.

\textbf{Assoliment}: Completament assolit. El sistema processa 600 combinacions (15 sol·licitants × 40 ofertes) en menys de 2 segons. Això és totalment adequat per a una aplicació interactiva i permet escalar fins a centenars d'ofertes sense problemes de rendiment.

\paragraph{Transparència i traçabilitat}

\textbf{Objectiu}: Ha de ser possible seguir i entendre com el sistema arriba a les seves decisions.

\textbf{Assoliment}: Completament assolit. El sistema genera missatges de traça en cada fase que permeten seguir l'execució. Les regles són llegibles i segueixen patrons clars. L'ús de forward chaining genera naturalment un traç que mostra com s'han inferit requisits i per què s'han pres decisions.

\vspace{0.5cm}

\subsection{Aprenentatges principals}

El desenvolupament d'aquesta pràctica ha proporcionat aprenentatges valuosos sobre diversos aspectes dels sistemes basats en el coneixement i de l'enginyeria del programari en general.

\vspace{0.5cm}

\subsubsection{Sobre sistemes basats en el coneixement}

\paragraph{La importància d'una bona ontologia}

Una de les lliçons més importants ha estat comprendre el paper fonamental de l'ontologia en un sistema basat en el coneixement. Una ontologia ben dissenyada amb jerarquies clares no només facilita la implementació de les regles sinó que també fa el sistema més mantenible i extensible.

Per exemple, tenir subclasses específiques de Sol·licitant (PersonaGran, GrupEstudiants, Família, etc.) ha permès escriure regles específiques per a cada perfil sense haver d'enumerar totes les possibles combinacions de característiques. Si només haguéssim tingut una classe genèrica Sol·licitant amb atributs, les regles haurien estat molt més complexes i difícils de mantenir.

De la mateixa manera, la jerarquia de Serveis ha estat crucial. Poder fer referència a "ServeiEducatiu" en lloc d'haver de llistar "Escola, Institut, Llar d'infants, Universitat" en cada regla fa el codi molt més concís i fàcil de llegir.

\paragraph{Forward vs Backward Chaining}

Hem après que la decisió entre forward i backward chaining no és arbitrària sinó que depèn fonamentalment de la naturalesa del problema. El nostre problema requeria avaluació exhaustiva de totes les combinacions sol·licitant-oferta, fent forward chaining l'elecció natural.

Backward chaining hauria estat adequat per a un tipus diferent d'aplicació, per exemple si l'usuari ja tingués una oferta específica en ment i volgués saber si és adequada per a ell. Però com que volem presentar un llistat complet de totes les ofertes avaluades i classificades, forward chaining és clarament superior.

\paragraph{El valor de l'organització en fases}

Una lliçó important ha estat que fins i tot en forward chaining, no és suficient simplement escriure regles i deixar que CLIPS les executi. Sense organització, les regles poden interferir entre elles i produir comportaments impredictibles.

L'organització en tres fases ben diferenciades (Abstracció, Resolució, Refinació) ha estat fonamental per aconseguir un sistema predictible i mantenible. Cada fase té una responsabilitat clara, i saber en quina fase pertany cada regla facilita enormement tant el desenvolupament com el debugging.

\paragraph{Templates vs Objects}

Hem après que CLIPS ofereix dos mecanismes (templates i objects) i que fer un bon ús de tots dos és important. Inicialment teníem la temptació d'utilitzar només objects per coherència amb l'ontologia orientada a objectes, però hem descobert que els templates són més adequats per a fets intermedis del raonament.

Els templates són més eficients per a pattern matching, més senzills d'utilitzar en condicions de regles, i representen millor la naturalesa temporal dels fets derivats. La distinció entre conceptes estables del domini (objects) i fets temporals del raonament (templates) ha millorat significativament la claredat del codi.

\paragraph{La potència i limitacions de la salience}

La salience de CLIPS és una eina molt potent per controlar el flux d'execució, però també pot ser perillosa si no s'utilitza amb disciplina. Hem après que:
\begin{itemize}
    \item És millor utilitzar rangs de salience ben separats per a diferents fases que intentar ajustar finament prioritats individuals.
    \item Els fets de control (com \texttt{fase-completada}) són més robusts que confiar només en salience.
    \item Una estratègia de salience ben documentada és essencial per mantenir el sistema.
\end{itemize}

\vspace{0.5cm}

\subsubsection{Sobre metodologia de desenvolupament}

\paragraph{El valor del prototipatge incremental}

Una de les lliçons més importants ha estat confirmar el valor del prototipatge ràpid i el desenvolupament incremental. Començar amb un sistema mínim funcional (3 classes, 5 regles, 3 instàncies) i anar ampliant de manera iterativa ha estat molt més efectiu que intentar dissenyar tot des del principi.

Cada iteració ha permès validar decisions, detectar problemes aviat i ajustar el disseny abans que fos massa tard per canviar-lo. Les proves primerenques han revelat problemes en el disseny de l'ontologia i en l'organització de les regles que haurien estat molt més costosos de corregir si els haguéssim descobert al final.

\paragraph{La importància de les proves contínues}

Hem après que no és suficient provar el sistema només al final. Cada iteració ha d'incloure proves que validin la funcionalitat implementada. Aquest enfocament ha permès construir confiança gradualment en el sistema i assegurar que cada nova funcionalitat no trenqués res del que ja funcionava.

A més, tenir un conjunt de casos de prova que creix amb cada iteració ha creat una xarxa de seguretat que ens ha permès fer canvis amb confiança.

\paragraph{L'elicitació amb models de llenguatge}

L'ús de models de llenguatge (ChatGPT) com a "expert virtual" durant la fase de conceptualització ha estat útil però requereix validació. Els models poden generar coneixement que sembla plausible però que no sempre reflecteix la realitat del domini.

Hem après que és millor utilitzar els models com a punt de partida per generar idees i estructurar el coneixement, però sempre validant amb fonts reals (webs immobiliàries, converses amb agents reals) abans d'implementar.

\vspace{0.5cm}

\subsubsection{Sobre CLIPS i programació basada en regles}

\paragraph{Pattern matching i eficiència}

Hem après que l'eficiència en CLIPS no ve només de tenir regles ràpides individualment sinó de minimitzar el nombre de vegades que les regles s'activen. Tècniques com:
\begin{itemize}
    \item Pre-càlcul de distàncies
    \item Patrons NOT per evitar duplicats
    \item Ordenació de condicions de més a menys restrictives
\end{itemize}

Han estat crucials per aconseguir un sistema eficient.

\paragraph{La importància dels missatges de traça}

Els missatges de traça (printout) que inicialment veiem com a debug temporal s'han revelat essencials per entendre el comportament del sistema. Saber en cada moment quina fase s'està executant, quines regles s'activen i quines decisions es prenen ha estat invaluable tant per al desenvolupament com per a la documentació.

\paragraph{Limitacions de CLIPS}

També hem après algunes limitacions de CLIPS:
\begin{itemize}
    \item Absència de mòduls explícits (tot comparteix namespace global)
    \item Sistema de tipus relativament simple
    \item Missatges d'error a vegades poc informatius
    \item Documentació oficial limitada (hem hagut de complementar amb exemples i FAQs)
\end{itemize}

Aquestes limitacions no han impedit desenvolupar un sistema funcional però han requerit disciplina i convencions clares per superar-les.

\vspace{0.5cm}

\subsection{Limitacions del sistema}

Tot i haver assolit els objectius establerts, és important reconèixer les limitacions del sistema actual.

\vspace{0.5cm}

\subsubsection{Limitacions funcionals}

\paragraph{Coneixement estàtic}

El coneixement del sistema està completament codificat en les regles. Afegir nous criteris d'avaluació o modificar la importància relativa de factors existents requereix modificar el codi. El sistema no aprèn de l'experiència ni s'adapta a les preferències individuals més enllà de les que es poden expressar explícitament en el perfil.

\paragraph{Preferències genèriques}

El sistema utilitza preferències "típiques" per a cada perfil demogràfic. Per exemple, assumeix que totes les famílies amb fills valoren zones verdes, però podrien haver famílies que prioritzessin altres aspectes. No hi ha mecanisme per capturar idiosincràsies individuals més enllà de les relacions explícites \texttt{requereixServei}, \texttt{prefereixServei} i \texttt{evitaServei}.

\paragraph{Absència de context temporal}

El sistema no considera l'historial de l'usuari ni com han evolucionat les seves preferències. Cada execució és independent i no hi ha memòria entre sessions. Això impedeix funcionalitats com "recordar" ofertes que l'usuari ha rebutjat prèviament o ajustar recomanacions basant-se en interaccions passades.

\paragraph{Model geogràfic simplificat}

El sistema utilitza coordenades cartesianes 2D i distància de Manhattan, que és una simplificació de la geografia real. No considera:
\begin{itemize}
    \item Distàncies reals per carretera
    \item Temps de desplaçament en transport públic
    \item Barreres geogràfiques (rius, autopistes)
    \item Diferències d'altura (important en ciutats amb relleu)
\end{itemize}

\vspace{0.5cm}

\subsubsection{Limitacions tècniques}

\paragraph{Escalabilitat}

Tot i que el sistema és adequat per a centenars d'ofertes, milers d'ofertes podrien ser problemàtics. Totes les instàncies es carreguen en memòria, i CLIPS avalua totes les combinacions possibles. Per a volums molt grans caldria:
\begin{itemize}
    \item Integració amb base de dades per evitar carregar tot a memòria
    \item Mecanismes d'indexació o filtratge previ
    \item Potser arquitectura distribuïda per a processament paral·lel
\end{itemize}

\paragraph{Interfície d'usuari}

La interfície actual és de línia de comandes, adequada per a proves i demostració però no per a usuaris finals no tècnics. Una aplicació real requeriria:
\begin{itemize}
    \item Interfície gràfica web o mòbil
    \item Formularis intuïtius per introduir perfils
    \item Visualització atractiva de resultats amb imatges
    \item Capacitat de comparar ofertes visualment
\end{itemize}

\paragraph{Integració amb sistemes reals}

El sistema opera amb dades simulades. Una aplicació real requeriria:
\begin{itemize}
    \item Web scraping o API per obtenir ofertes reals
    \item Integració amb Google Maps per distàncies i temps reals
    \item Actualització automàtica quan ofertes deixen d'estar disponibles
    \item Sincronització amb bases de dades de serveis de la ciutat
\end{itemize}

\vspace{0.5cm}

\subsubsection{Limitacions de cobertura}

Com s'ha esmentat, hi ha casos consciement deixats fora de l'abast:

\begin{itemize}
    \item \textbf{Habitatges comercials}: Oficines, locals, naus industrials
    \item \textbf{Lloguers temporals}: Vacances, estades curtes
    \item \textbf{Co-housing}: Habitacions en pisos compartits amb desconeguts
    \item \textbf{Necessitats mèdiques específiques}: Més enllà d'accessibilitat general
    \item \textbf{Preferències culturals/religioses}: Proximitat a llocs de culte, tipus de barri, etc.
\end{itemize}

Aquestes limitacions estan justificades per mantenir la complexitat del sistema en un nivell raonable, però en una aplicació comercial real haurien de considerar-se.

\vspace{0.5cm}

\subsection{Treball futur}

Hi ha múltiples direccions en què el sistema podria millorar-se o estendre's.

\vspace{0.5cm}

\subsubsection{Millores a curt termini}

Aquestes són millores que podrien implementar-se relativament fàcilment amb el sistema actual:

\paragraph{Ordenació dins de categories}

Actualment les recomanacions d'un mateix nivell (per exemple, totes les "Adequades") no estan ordenades. Seria útil afegir una puntuació numèrica més fina que permeti ordenar ofertes dins de cada categoria. Això permetria presentar primer les millors ofertes de cada nivell.

\paragraph{Pesos configurables}

Afegir la capacitat de configurar la importància relativa de diferents factors. Per exemple, un usuari podria indicar que valora el preu per sobre de la localització, o viceversa. Això es podria implementar amb pesos associats a cada tipus de criteri positiu/negatiu.

\paragraph{Anàlisi de trade-offs}

Implementar funcionalitat per suggerir trade-offs a l'usuari: "Si acceptes 10% més de preu, pots tenir parking" o "Si acceptes estar a 600m del metro en lloc de 300m, pots estalviar 200€/mes". Això requeriria analitzar ofertes descartades o parcialment adequades i identificar quins ajustos les farien adequades.

\paragraph{Comparació explícita d'ofertes}

Afegir funcionalitat per comparar dues ofertes específiques directament, mostrant en què es diferencien i quina és millor en cada aspecte. Això ajudaria usuaris que tinguin dues ofertes finalistes i necessitin decidir entre elles.

\vspace{0.5cm}

\subsubsection{Extensions funcionals}

Aquestes són extensions que afegeixen funcionalitat nova més substancial:

\paragraph{Aprenentatge de preferències}

Implementar un mecanisme per capturar feedback de l'usuari (m'agrada / no m'agrada aquesta oferta) i ajustar les recomanacions futures basant-se en aquest feedback. Això podria fer-se:
\begin{itemize}
    \item Afegint nous fets que representin preferències apreses
    \item Ajustant pesos de criteris segons el feedback
    \item Inferint preferències implícites del patró de likes/dislikes
\end{itemize}

\paragraph{Predicció de disponibilitat}

Incorporar dades històriques per estimar quant temps sol estar disponible un tipus d'oferta similar. Això permetria advertir a l'usuari: "Ofertes així solen llogar-se en menys d'una setmana, contacta ràpidament si t'interessa".

\paragraph{Recomanacions proactives}

En lloc d'esperar que l'usuari executi el sistema, enviar notificacions automàtiques quan apareguin ofertes noves que encaixin bé amb el perfil guardat. Això requeriria:
\begin{itemize}
    \item Sistema de subscripció de perfils
    \item Monitorització contínua de noves ofertes
    \item Sistema de notificacions (email, push, etc.)
\end{itemize}

\paragraph{Anàlisi de mercat}

Afegir funcionalitat per analitzar el mercat globalment: "Hi ha escassetat d'ofertes en el teu rang de preu a aquesta zona" o "Els preus en aquesta zona han pujat un 5% aquest mes". Això requeriria mantenir històrics i fer anàlisis estadístiques.

\vspace{0.5cm}

\subsubsection{Integració amb dades reals}

Per convertir el sistema en una aplicació comercial real caldria:

\paragraph{Web scraping}

Implementar scrapers per obtenir ofertes automàticament de portals immobiliaris (Idealista, Fotocasa, etc.). Això requereix:
\begin{itemize}
    \item Parsers específics per a cada web
    \item Sistema de detecció de duplicats
    \item Actualització periòdica
    \item Gestió de canvis en les webs objectiu
\end{itemize}

\paragraph{API de Google Maps}

Integrar Google Maps API per:
\begin{itemize}
    \item Convertir adreces a coordenades GPS
    \item Calcular distàncies i temps reals de desplaçament
    \item Obtenir informació sobre transport públic
    \item Visualitzar ofertes en un mapa interactiu
\end{itemize}

\paragraph{Base de dades}

Migrar de memòria a una base de dades relacional (PostgreSQL, MySQL) per:
\begin{itemize}
    \item Gestionar grans volums d'ofertes
    \item Mantenir històrics
    \item Gestionar múltiples usuaris
    \item Permetrenyes
    \item Fer consultes SQL eficients
\end{itemize}

\paragraph{Interfície web}

Desenvolupar una aplicació web completa amb:
\begin{itemize}
    \item Frontend en React o Vue.js
    \item Backend que utilitzi CLIPS com a motor de regles
    \item API REST per comunicació frontend-backend
    \item Autenticació i gestió d'usuaris
    \item Disseny responsive per a mòbils
\end{itemize}

\vspace{0.5cm}

\subsubsection{Hibridació amb Machine Learning}

Una direcció molt interessant seria combinar l'aproximació basada en regles amb machine learning:

\paragraph{ML per inferir preferències}

Utilitzar algoritmes de ML per:
\begin{itemize}
    \item Detectar patrons en el comportament de l'usuari (què clica, què descarta)
    \item Inferir preferències implícites que l'usuari no ha expressat explícitament
    \item Fer collaborative filtering ("usuaris similars a tu també van valorar...")
\end{itemize}

\paragraph{ML per predicció de preus}

Models de regressió per:
\begin{itemize}
    \item Detectar ofertes subvalorades o sobrevalorades
    \item Predir l'evolució de preus en diferents zones
    \item Estimar el preu just d'una oferta segons les seves característiques
\end{itemize}

\paragraph{Mantenir explicabilitat}

La clau seria utilitzar ML per afinar paràmetres i pesos, però mantenir les regles per a la decisió final, assegurant que el sistema segueix sent explicable. Això dóna el millor dels dos mons: la capacitat d'aprenentatge del ML amb la transparència dels sistemes basats en regles.

\vspace{0.5cm}

\subsection{Reflexió final}

Aquest projecte ha demostrat el valor dels sistemes basats en el coneixement per a problemes que requereixen raonament basat en regles complexes i explicabilitat transparent. Tot i que el machine learning dominla actualitat de la Intel·ligència Artificial, els sistemes basats en el coneixement segueixen tenint el seu lloc per a aplicacions on:

\begin{itemize}
    \item El coneixement expert pot estructurar-se en regles
    \item La transparència i explicabilitat són crucials
    \item Les decisions han de ser auditables i justificables
    \item Els criteris d'avaluació són objectius i ben definits
\end{itemize}

El nostre sistema de recomanació d'habitatges compleix tots aquests requisits. Les decisions sobre si un habitatge és adequat per a una família es basen en criteris objectius i ben establerts que poden expressar-se en regles. La capacitat d'explicar per què una oferta és recomanable o no és fonamental per generar confiança en l'usuari.

L'experiència de desenvolupament ha consolidat la comprensió de diversos conceptes clau:

\begin{enumerate}
    \item \textbf{Metodologia d'enginyeria del coneixement}: Hem aplicat les fases d'identificació, conceptualització, formalització, implementació i prova de manera sistemàtica.
    
    \item \textbf{Disseny i implementació d'ontologies}: Hem après a crear ontologies ben estructurades que capturen adequadament el coneixement del domini.
    
    \item \textbf{Sistemes de producció}: Hem adquirit experiència pràctica amb forward chaining i l'algorisme RETE.
    
    \item \textbf{CLIPS}: Hem après a utilitzar tant templates com objects, a controlar el flux amb salience, i a estructurar sistemes complexos de manera mantenible.
    
    \item \textbf{Prototipatge incremental}: Hem comprovat el valor del desenvolupament iteratiu amb validació contínua.
    
    \item \textbf{Validació i proves}: Hem après a dissenyar jocs de prova exhaustius que cobreixin tant casos típics com extrems.
\end{enumerate}

El sistema resultant és plenament funcional dins del seu abast definit, genera recomanacions coherents i ben justificades, i demostra els principis fonamentals dels sistemes basats en el coneixement. Tot i les limitacions identificades, el sistema podria ser la base d'una aplicació real amb les extensions adequades discutides en la secció de treball futur.

Finalment, aquesta pràctica ha reforçat la idea que l'IA no es redueix només a machine learning i deep learning. Els sistemes basats en el coneixement, tot i ser una aproximació més clàssica, segueixen sent una eina potent i adequada per a molts problemes reals on la transparència i l'explicabilitat són fonamentals.